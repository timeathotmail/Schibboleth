\documentclass[fontsize=12pt,paper=a4,twoside]{scrartcl}

\newcommand{\grad}{\ensuremath{^{\circ}} }
\renewcommand{\strut}{\vrule width 0pt height5mm depth2mm}

\usepackage[utf8]{inputenc}
\usepackage[final]{pdfpages}
% obere Seitenränder gestalten können
\usepackage{fancyhdr}
\usepackage{moreverb}
% Graphiken als jpg, png etc. einbinden können
\usepackage{graphicx}
\usepackage{stmaryrd}
% Floats Objekte mit [H] festsetzen
\usepackage{float}
% setzt URL's schön mit \url{http://bla.laber.com/~mypage}
\usepackage{url}
% Externe PDF's einbinden können
\usepackage{pdflscape}
% Verweise innerhalb des Dokuments schick mit " ... auf Seite ... "
% automatisch versehen. Dazu \vref{labelname} benutzen
\usepackage[ngerman]{varioref}
\usepackage[ngerman]{babel}
\usepackage{ngerman}
% Bibliographie
\usepackage{bibgerm}
% Tabellen
\usepackage{tabularx}
\usepackage{supertabular}
\usepackage[colorlinks=true, pdfstartview=FitV, linkcolor=blue,
            citecolor=blue, urlcolor=blue, hyperfigures=true,
            pdftex=true]{hyperref}
\usepackage{bookmark}

% Damit Latex nicht zu lange Zeilen produziert:
\sloppy
%Uneinheitlicher unterer Seitenrand:
%\raggedbottom

% Kein Erstzeileneinzug beim Absatzanfang
% Sieht aber nur gut aus, wenn man zwischen Absätzen viel Platz einbaut
\setlength{\parindent}{0ex}

% Abstand zwischen zwei Absätzen
\setlength{\parskip}{1ex}

% Seitenränder für Korrekturen verändern
\addtolength{\evensidemargin}{-1cm}
\addtolength{\oddsidemargin}{1cm}

\bibliographystyle{gerapali}

% Lustige Header auf den Seiten
  \pagestyle{fancy}
  \setlength{\headheight}{70.55003pt}
  \fancyhead{}
  \fancyhead[LO,RE]{Software--Projekt 2\\ SoSe 2014
  \\Projektplan}
  \fancyhead[LE,RO]{Seite \thepage\\\slshape \leftmark\\\slshape \rightmark}

%
% Und jetzt geht das Dokument los....
%

\begin{document}

% Lustige Header nur auf dieser Seite
  \thispagestyle{fancy}
  \fancyhead[LO,RE]{ }
  \fancyhead[LE,RO]{Universität Bremen\\FB 3 -- Informatik\\
  Dr.\ Karsten Hölscher }
  \fancyfoot[C]{}

% Start Titelseite
  \vspace{3cm}

  \begin{minipage}[H]{\textwidth}
  \begin{center}
  \bf
  \Large
  Software--Projekt 2 2014\\
  \smallskip
  \small
  VAK 03-BA-901.02\\
  \vspace{3cm}
  \end{center}
  \end{minipage}
  \begin{minipage}[H]{\textwidth}
  \begin{center}
  \vspace{1cm}
  \bf
  \Large Projektplan\\
  \vfill
  \end{center}
  \end{minipage}
  \vfill
  \begin{minipage}[H]{\textwidth}
  \begin{center}
  \sf
  \begin{tabular}{lrr}
  xxxxxx xxxxxxx & xxxxxxxx@tzi.de & 1234567\\
  xxxx xxxxxxxx & xxxx@tzi.de & 2345678\\
  \end{tabular}
  \\ ~
  \vspace{2cm}
  \\
  \it Abgabe: Tag. Monat. Jahr --- Version 1.2\\ ~
  \end{center}
  \end{minipage}

% Ende Titelseite

% Start Leerseite

\newpage

  \thispagestyle{fancy}
  \fancyhead{}
  \fancyhead[LO,RE]{Software--Projekt \\  2014
  \\Projektplan}
  \fancyhead[LE,RO]{Seite \thepage\\\slshape \leftmark\\~}
  \fancyfoot{}
  \renewcommand{\headrulewidth}{0.4pt}
  \tableofcontents

\newpage

  \fancyhead[LE,RO]{Seite \thepage\\\slshape \leftmark\\\slshape \rightmark}


%%%%%%%%%%%%%%%%%%%%%%%%%%%%%%%%%%%%%%%%%%%%%%%%%%%%%%%%%%%%%%%%%%%%%%%%
\section*{Version und Änderungsgeschichte}

{\em Die aktuelle Versionsnummer des Dokumentes sollte eindeutig und gut zu
identifizieren sein, hier und optimalerweise auf dem Titelblatt.}

\begin{tabular}{ccl}
Version & Datum & Änderungen \\
\hline
1.0 & TT.MM.JJJJ & Erste veröffentlichte Version. \\
1.1 & TT.MM.JJJJ & Zeitplanung für die Anforderungsspezifikation hinzugefügt. \\
1.2 & TT.MM.JJJJ & .... 
\end{tabular}


%%%%%%%%%%%%%%%%%%%%%%%%%%%%%%%%%%%%%%%%%%%%%%%%%%%%%%%%%%%%%%%%%%%%%%%%
\section{Einleitung}

\subsection{Projektübersicht}

\subsubsection{Ziele}

{\em Hier folgt die Kurzbeschreibung der Aufgabe, soweit sie bisher
  bekannt ist. Auch: was ist {\rm nicht} Teil der Aufgabe.}

\subsubsection{Hauptarbeitsaktivitäten und --produkte}


\subsubsection{Haupt--Meilensteine und grober Zeitplan}

{\em Meilensteine, jeweils mit konkretem Datum,
 Kriterien für die Erfüllung der Meilensteine.}


\subsubsection{Benötigte Ressourcen}

\begin{itemize}
\item \textbf{Menschliche Ressourcen}

\item \textbf{Hardware}

\item \textbf{Räume}

\dots

\end{itemize}

\subsubsection{Budget}

{\em Beinhaltet auch konkrete Angaben zu Entwicklerstunden und Kosten in Euro.}


\subsubsection{Kontaktdaten des Kunden}

\subsubsection{Mitarbeiter}
{\em Hier finden sich alle Mitarbeitenden der Gruppe mit Kontaktdaten und Foto.}

\subsection{Auszuliefernde Produkte}


\subsection{Evolution des Plans}

{\em Wird der Plan verändert? Wann? Wie oft? Von wem? Wenn bereits Aktualisierungen vorgesehen sind, welche sind das? Möglicherweise betrifft das die Zeitplanung, die Risikobewertung, oder andere Teile des Plans. Gibt es möglicherweise auch unvorhergesehene Aktualisierungen?}

\subsection{Referenzen}
% mit \nocite kann man Literatur auflisten, die im Text nicht explizit
% erwähnt ist. \nocite{*} zitiert dann das ganze .bib-File
%
% Die Bibliographie erzeugt man indem man erst
%
% pdflatex bericht.tex
% bibtex bericht
% pdflatex bericht.tex
% pdflatex bericht.tex
%
% benutzt
%\nocite{Knudsen1}
%\nocite{*}
%\bibliography{literatur}

% Das renewcommand verhindert dass für die Literatur eine section* angelegt wird.
% auftaucht
{\renewcommand\section[2]{}
\bibliography{referenzen}
}

\subsection{Definitionen und Akronyme}

{\em Hier sollen Begriffe definiert werden, die nötig sind, um den
  Projektplan zu verstehen. Diese kommen insbesondere aus der Welt des
  Kunden (Projektdomäne) und der Welt des Softwareproduzenten.}

\section{Projektorganisation}

\subsection{Prozessmodell}

\subsection{Organisationsstruktur}

{\em Genaue Beschreibung der Rollen, Rechte und Pflichten!}

{\em z.B. auch regelmäßiges Treffen im Chat, Einrichtung einer
  Groupware oder eines Forums, o.ä. \dots}

\subsection{Organisationsgrenzen und --schnittstellen}

{\em Hierher gehören auch evtl. Kontaktpersonen für Fremdbibliotheken u.ä.}

\subsection{Verantwortlichkeiten}

\section{Managementprozess}

\subsection{Managementprozess und --prioritäten}

\subsection{Annahmen, Abhängigkeiten und Einschränkungen}

\subsection{Risikomanagement}\label{riskmanagement}

{\em Wenn Ihr Euch entschieden habt, bestimmte vorbeugende Maßnahmen 
     durchzuführen, solltet Ihr dies deutlich kennzeichnen. Hoffentlich
     haben diese Maßnahmen dann einen Einfluss auf Eintrittswahrscheinlichkeit oder Schadenshöhe (zum Beispiel
     ist die Eintrittswahrscheinlichkeit von komplettem Datenverlust durch regelmäßige Backups deutlich 
     geringer). Daher solltet Ihr für diese Fälle dann die verringerten Werte für Eintrittswahrscheinlichkeit, 
     Schadenshöhe und Risikopotential zusätzlich angeben. }

{\em Wie werden neue Risiken erkannt/erfasst? Wer ist für was
  zuständig? Wie ist der Informationsfluss? \ldots 

Dieser Teil ist ein
  wichtiger Schwerpunkt des Projektplans und sollte daher ausführlich
  behandelt werden.}

\subsection{Projektüberwachung}\label{3.4-controlling}
{\em Wie wird der Projektstatus verfolgt? Wie stellt Ihr sicher, dass
  der Phasenleiter jederzeit über den Stand der Entwicklung informiert
  ist? Wie werden Probleme bzw. Verzögerungen frühzeitig erkannt und
  angegangen?}

\subsection{Mitarbeiter}
{\em Kompetenzen der und Anforderungen an die Mitarbeiter.}

%%%%%%%%%%%%%%%%%%%%%%%%%%%%%%%%%%%%%%%%%%%%%%%%%%%%%%%%%%%%%%%%%%%%%%%%

\section{Technische Prozesse}
\subsection{Methoden, Werkzeuge und Techniken}
\subsubsection{Entwicklungsplattform}

\subsubsection{Entwicklungsmethode}
{\em Ist der Einsatz spezieller Methoden vorgesehen?}

\subsubsection{Programmiersprache und Bibliotheken}

\subsection{Dokumentationsplan}

\subsubsection{Codingstyle}

\subsubsection{Kommentarsprache}

\subsubsection{JavaDoc}

\subsubsection{Begleitende Dokumentation}

\subsection{Unterstützende Projektfunktionen}
{\em Wie wird Euer Konfigurationsmanagement funktionieren? Wer ist verantwortlich? Benötigt Ihr dazu Ressourcen oder Zeit? Plant Ihr Datensicherung?}

{\em Gibt es Maßnahmen zur Qualitätssicherung? Wer ist zuständig?
  Wieviel Zeit ist dafür vorgesehen?}


%%%%%%%%%%%%%%%%%%%%%%%%%%%%%%%%%%%%%%%%%%%%%%%%%%%%%%%%%%%%%%%%%%%%%%%%

\section{Arbeitspakete, Zeitplan und Budget}

{\em Dieser Teil ist ein zweiter Schwerpunkt des Projektplans. Hier sollt Ihr die nächste Phase detailliert planen (siehe Arbeitspakete). Die weiteren Phasen sollen ebenfalls wenigstens grob geplant werden. Ein Gantt-Diagramm ist zwingend! 

Ihr sollt den Plan in der kommenden Phase auch tatsächlich benutzen -- und so
  Erfahrungen sammeln, was evtl. bei der Planung unberücksichtigt
  blieb. Bei der nächsten Zeitplanung (für die nächste Phase) bekommt
  Ihr dann evtl.\ eine noch bessere Planung hin.}

\subsection{Arbeitspakete}\label{aps}


{\em Besonderen Wert legen wir auf die Granularität der APs. Diese
  sollten von 1-2 Personen in max. einer Woche Zeitdauer (kalendarisch, nicht
  Aufwand) bearbeitbar sein. Die Beschreibungen sollten so genau sein,
  dass der Bearbeiter damit genau weiß, was zu tun ist.}

\subsection{Zeitplan und Abhängigkeiten}

{\em Die Abhängigkeiten zwischen Arbeitspaketen oder Meilensteinen müssen genannt werden, sowie im
  Gantt-Diagramm eingezeichnet werden. Der kritische Pfad soll
  angegeben und/oder eingezeichnet werden!}

\subsection{Ressourcenanforderung}

{\em Jedem Arbeitspaket muss mind.\ ein Bearbeiter zugeordnet
  werden. Die Zuordnung der ganzen Gruppe sollte nur in Ausnahmefällen
  erfolgen -- und dann vermutlich begründet werden!}


%%%%%%%%%%%%%%%%%%%%%%%%%%%%%%%%%%%%%%%%%%%%%%%%%%%%%%%%%%%%%%%%%%%%%%%%
\section{Sonstige Elemente}

\subsection{Pläne für die Konvertierung von Daten}


\subsection{Managementpläne für Unterauftragsnehmer}

{\em Wenn Fremdbibliotheken benutzt werden\dots}

\subsection{Ausbildungspläne}

{\em Hierunter fallen z.B. auch interne Schulungen, die Ihr
  durchführen wollt.}

\subsection{Raumpläne}
\dots

\subsection{Installationspläne}
\dots

\subsection{Pläne für die Übergabe des Systems}
\dots

\subsection{Beschaffungspläne für Hardware}
\dots


\end{document}
