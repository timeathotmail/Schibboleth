\documentclass[fontsize=12pt,paper=a4,twoside]{scrartcl}

\newcommand{\grad}{\ensuremath{^{\circ}} }
\renewcommand{\strut}{\vrule width 0pt height5mm depth2mm}

\usepackage[utf8]{inputenc}
\usepackage[final]{pdfpages}
% obere Seitenränder gestalten können
\usepackage{fancyhdr}
\usepackage{moreverb}
% Graphiken als jpg, png etc. einbinden können
\usepackage{graphicx}
\usepackage{stmaryrd}
% Floats Objekte mit [H] festsetzen
\usepackage{float}
% setzt URL's schön mit \url{http://bla.laber.com/~mypage}
\usepackage{url}
% Externe PDF's einbinden können
\usepackage{pdflscape}
% Verweise innerhalb des Dokuments schick mit " ... auf Seite ... "
% automatisch versehen. Dazu \vref{labelname} benutzen
\usepackage[ngerman]{varioref}
\usepackage[ngerman]{babel}
\usepackage{ngerman}
% Bibliographie
\usepackage{bibgerm}
% Tabellen
\usepackage{tabularx}
\usepackage{supertabular}
\usepackage[colorlinks=true, pdfstartview=FitV, linkcolor=blue,
            citecolor=blue, urlcolor=blue, hyperfigures=true,
            pdftex=true]{hyperref}
\usepackage{bookmark}


\newboolean{langversion} %Deklaration
\setboolean{langversion}{true} %Zuweisung ist 'false' für Blockkurs
\newcommand{\highlight}[1]{\textcolor{blue}{\textbf{#1}}}
\newcommand{\nurlangversion}[0]{%
\ifthenelse{\boolean{langversion}}{\highlight{}}{\highlight{Entfällt in SWP-1}}}

% erstes Argument: SWP-2, zweites SWP-1
\newcommand{\variante}[2]{\ifthenelse{\boolean{langversion}}{#1}{#2}}


% Damit Latex nicht zu lange Zeilen produziert:
\sloppy
%Uneinheitlicher unterer Seitenrand:
%\raggedbottom

% Kein Erstzeileneinzug beim Absatzanfang
% Sieht aber nur gut aus, wenn man zwischen Absätzen viel Platz einbaut
\setlength{\parindent}{0ex}

% Abstand zwischen zwei Absätzen
\setlength{\parskip}{1ex}

% Seitenränder für Korrekturen verändern
\addtolength{\evensidemargin}{-1cm}
\addtolength{\oddsidemargin}{1cm}

\bibliographystyle{gerapali}

% Lustige Header auf den Seiten
  \pagestyle{fancy}
  \setlength{\headheight}{70.55003pt}
  \fancyhead{}
  \fancyhead[LO,RE]{Software--Projekt 2\\  2014
  \\Testplan}
  \fancyhead[LE,RO]{Seite \thepage\\\slshape \leftmark\\\slshape \rightmark}

%
% Und jetzt geht das Dokument los....
%

\begin{document}

% Lustige Header nur auf dieser Seite
  \thispagestyle{fancy}
  \fancyhead[LO,RE]{ }
  \fancyhead[LE,RO]{Universität Bremen\\FB 3 -- Informatik\\
  Dr.\ Karsten Hölscher \\Tutor: Karsten Hölscher}
  \fancyfoot[C]{}

% Start Titelseite
  \vspace{3cm}

  \begin{minipage}[H]{\textwidth}
  \begin{center}
  \bf
  \Large
  Software--Projekt 2 2014\\
  \smallskip
  \small
  VAK 03-BA-901.02\\
  \vspace{3cm}
  \end{center}
  \end{minipage}
  \begin{minipage}[H]{\textwidth}
  \begin{center}
  \vspace{1cm}
  \bf
  \Large Testplan\\
  \includegraphics[width=0.6\textwidth]{Bilder/Logo.png}
  \vfill
  \end{center}
  \end{minipage}
  \vfill
  \begin{minipage}[H]{\textwidth}
  \begin{center}
  \sf
  \begin{tabular}{lrr}
  Patrick Hollatz & phollatz@tzi.de & 2596537 \\
  Tobias Dellert & tode@tzi.de & 2936941 \\
  Tim Ellhoff & tellhoff@tzi.de & 2520913\\
  Daniel Pupat & dpupat@tzi.de & 2703053 \\
  Olga Miloevich & halfelv@tzi.de  & 2586817\\  
  Tim Wiechers & tim3@tzi.de & 2925222 \\

  \end{tabular}
  \\ ~
  \vspace{2cm}
  \\
  \it Abgabe: 06. Juli 2014 --- Version 1.1\\ ~
  \end{center}
  \end{minipage}

% Ende Titelseite

% Start Leerseite

\newpage

  \thispagestyle{fancy}
  \fancyhead{}
  \fancyhead[LO,RE]{Software--Projekt \\  2014
  \\Testplan}
  \fancyhead[LE,RO]{Seite \thepage\\\slshape \leftmark\\~}
  \fancyfoot{}
  \renewcommand{\headrulewidth}{0.4pt}
  \tableofcontents

\newpage

  \fancyhead[LE,RO]{Seite \thepage\\\slshape \leftmark\\\slshape \rightmark}


%%%%%%%%%%%%%%%%%%%%%%%%%%%%%%%%%%%%%%%%%%%%%%%%%%%%%%%%%%%%%%%%%%%%%%%%
\section*{Version und Änderungsgeschichte}

\begin{tabular}{ccl}
Version & Datum & Änderungen \\
\hline
1.0 & 04.07.2014 & Dokumentvorlage als initiale Fassung kopiert \\
1.1 & 05.07.2014 & Einführung, Systemüberblick, Merkmale, Abnahme- und Testendekriterien\\
1.2 & 06.07.2014 & Testplan abgabefertig
\end{tabular}

\paragraph{Wichtiger Hinweis:} Dieser Testplan wurde zu einem Teil aus verschiedenen Dokumententeilen von dem Testplan unserer Gruppenmitglieder aus dem Wintersemester 2013/14 erstellt (Gruppe \textit{IT\_R3V0LUTION}). Einige Teile wurden komplett übernommen, andere überarbeitet bzw. angepasst. \\
Diese Vereinbarung haben wir in der Kick-Off-Veranstaltung für RE SWP 2014 mit dem Veranstalter Dr. Karsten Hölscher getroffen. \\

%%%%%%%%%%%%%%%%%%%%%%%%%%%%%%%%%%%%%%%%%%%%%%%%%%%%%%%%%%%%%%%%%%%%%%%%
\section{Einführung}\label{c01}
\textit{bearbeitet von: Daniel Pupat }

\subsection{Zweck}
Der Testplan bietet einen Überblick über die geplanten Tests und dient u.a. als Anleitung für die Tester. Die Software soll dabei ausführlich auf Funktionalität getestet werden. 

Im Testplan wird festgelegt, wie man welche Komponenten testet. Dazu wird außerdem definiert, welchen Umfang die Tests haben sollen und wann ein Test erfolgreich ist und wann nicht.

Während der Implementierungsphase werden wir uns nach dem Testplan richten und ihn gegebenenfalls weiterführen und vervollständigen.


\subsection{Umfang}
Der Testplan entspricht der vereinfachten Form des \emph{IEEE Standard for Software Test Documentation 829-1998}. 

\subsection{Beziehungen zu anderen Dokumenten}
Dieser Testplan bezieht sich zum einen auf die Anforderungsspezifikation, da dort die Systemeigenschaften und Systemattribute spezifiziert wurden. Die Testfälle werden auf Grundlage der dortigen Anwendungsfälle entwickelt.

Außerdem gibt es Referenzen zur Architekturbeschreibung, da in dieser die Module und Komponenten definiert wurden, die in diesem Dokument getestet werden sollen.


\subsection{Aufbau der Testbezeichner}
\label{sec:aufb-der-testb}

Der Aufbau der Testbezeichner richtet sich nach folgendem Schema:
\begin{itemize}
\item Die ersten beiden Buchstaben geben die Art des Tests vor. Dabei unterscheiden wir zwischen vier verschiedenen Testarten:
\begin{itemize}
	\item Komponententests = KT
	\item Integrationstests = IT
	\item Funktionstests = FT
	\item Leistungstests = LT
\end{itemize}
\item Die Nummer steht für die jeweilige Testfallnummer

\item Optional: in alphabetischer Reihenfolge werden hier Variationen oder untergeordnete Testfälle definiert
\end{itemize}
Nach diesem Schema sieht ein Testbezeichner nun folgendermaßen aus:

\textbf{\emph{IT-3-A}}: Integrationstest, Nr. 3, Variante 1

%%
%% Dokumentation der Testergebnisse 
%%
\subsection{Dokumentation der Testergebnisse}

Zu jedem Testfall wird ein kurzes Testprotokoll angefertigt. Dieses beinhaltet den Ablauf des Testfalls und die möglichen Komplikationen, die bei der Durchführung entstehen können. Dann werden die Resultate des Testfalls bestimmt und eventuell gefundene Fehler beschrieben.

%%
%% Definitionen und Referenzen 
%%
\subsection{Definitionen und Akronyme}
\label{c00b}


\subsection{Referenzen}
\emph{IEEE Standard for Software Test Documentation 829-1998}\\
\url{http://standards.ieee.org/findstds/standard/829-1998.html}

\bibliographystyle{plain}
\bibliography{literatur}

% Systemüberblick
\section{Systemüberblick}\label{c02}
\textit{bearbeitet von: Daniel Pupat }\\

Das System besteht aus der Server- und der Clientkomponente. Die konzeptionelle Sicht der Architekturbeschreibung (vgl. Abschnitt 3 der Architekturbeschreibung) dient als Grundlage für den Testplan, da dort die verschiedenen Komponenten  beschrieben werden.

Auf der Serverseite gibt es die Komponenten \texttt{Communication, BusinessLogic} und \texttt{Persistence} (vgl. Abbildung 3: Konzeptionelle Sicht Server; Architekturbeschreibung).

Die Clientseite besteht aus den Komponenten \texttt{Communication, Model} und \texttt{User Interface} (vgl. Abbildung 4: Konzeptionelle Sicht Client; Architekturbeschreibung).

Da starke Abhängigkeiten zwischen all diesen Komponenten bestehen, ist es wichtig, dass diese Komponenten fehlerfrei funktionieren. 

\subsection{Module der Anwendungsschicht und deren Funktionen}
\label{mod-controller}

In der nachfolgenden Tabelle werden die Module verfeinert, die in Punkt 5 visualisiert sind.

\begin{tabular}{|l|p{12cm}|}
\hline
	GUI & webapp(xhtml)\\
\hline
	AndroidApp & com.mygdx.game, com.mygdx.game\\
\hline
	Communication & common.net\\
\hline
	UserInterface & GUI AndroidApp\\
\hline
	Server & Persistence, ServerInbox, ServerDirectory, Server, \\
\hline
	Common & entities, common.net\\
\hline
	Persistence & Data.java, IPersistence.java\\
\hline
\end{tabular}


\clearpage

\section{Merkmale}
\textit{bearbeitet von: Daniel Pupat }\\
\label{c04}

Zu testende Merkmale sind in erster Linie Funktionen, welche alle Anwendungsfälle abdecken, die die Kundin sich gewünscht haben. Dabei muss sowohl die App, als auch die Website getestet werden.\\

1. Online\\
\begin{itemize}
\item[1.1]Benutzer registrieren
\item[1.2]Benutzer anmelden 
\item[1.3]Benutzer abmelden 
\item[1.4]Gegner herausfordern
\item[1.5]Angefangenes Spiel weiterspielen
\item[1.6]Frage beantworten 
\item[1.7]Einstellungen ändern
\end{itemize}
\bigskip
2. Offline\\
\begin{itemize}
\item[2.1]Neues Spiel starten
\item[2.2]Frage beantworten
\end{itemize}
\bigskip
3.Website\\
\begin{itemize}
\item[3.1]Admin anmelden
\item[3.2]Admin abmelden
\item[3.3]Frage hinzufügen
\item[3.4]Frage bearbeiten
\item[3.5]Frage löschen
\item[3.6]Frageliste importieren
\item[3.7]Frageliste exportieren
\item[3.8]User löschen
\item[3.9]Passwort ändern
\end{itemize}
\bigskip


\subsubsection{Funktionale Anforderungen} 
Besonders wichtig ist der Spielvorgang bei der App, da die Nutzer diesen hauptsächlich nutzen werden. Dabei muss darauf geachtet werden, dass der Nutzer sowohl online als auch offline spielen kann. Auch das Spielen gegen eine andere Person muss getestet werden.\\

\subsection{Nicht zu testende Merkmale}\label{c05}
Da wir ein komplett neues Projekt erstellen, ist es wichtig, dass alle Merkmale getestet werden. Sollte es triviale Funktionen geben, müssen diese nicht getestet werden.

% Abnahme- und Testkriterien
\section{Abnahme- und Testendekriterien}\label{c07}
\textit{bearbeitet von: Daniel Pupat }\\

Fehler werden in eine Kategorie eingeordnet und erhalten entsprechende Fehlerwerte. Aus diesen Fehlerwerten ergeben sich Prioritäten, welche die Reihenfolge der Fehlerbehandlung angeben. Das Testen wird beendet, wenn der berechnete Fehlerwert aller Fehler pro 1000 Zeilen Code unter dem Wert 10 liegt und die Software nicht beeinträchtigt wird, d.h. es keinen Fehler der Fehlerklasse \texttt{Mittel} oder höher gibt.

\textbf{Testabdeckung}
Die Testabdeckung soll so hoch wie möglich sein. Für ein stabiles System spricht, dass die Testabdeckung in systemkritischen Bereichen soweit vollständig ist. Jeder Fehler in diesem Bereich kann das System zum Absturz bringen und muss somit verhindert werden. In anderen Bereichen, die das laufende System bei einem Fehler weniger beeinträchtigen, wird die Testabdeckung nicht so vollständig sein, wie in kritischen Bereichen.

\textbf{Fehlerbewertung:}\\
Die nachfolgende Tabelle spezifiziert die Auswirkung eines Fehlers, durch die man diese nach Priorität einordnen kann.\\

\begin{tabularx}{\textwidth}{|p{2cm}|p{11.53cm}|c|}
\hline
	\textbf{Fehlerkl.\footnote{=Fehlerklasse}} & \textbf{Beschreibung} & \textbf{Wert}\\
\hline
	Leicht & Unwesentliche Fehler, die den Programmablauf nicht beeinträchtigen, aber trotzdem behandelt werden sollten. & 1\\
\hline
	Mittel & Fehler in dieser Art haben Auswirkungen auf den Programmablauf. Dieser beeinträchtigt aber nicht die grundlegenden Funktionen. & 10\\
\hline
	Schwer & Fehler der Klasse ,,Schwer'' beeinträchtigen die Funktionsfähigkeit des Systems sehr stark und müssen sofort behandelt werden. & 20\\
\hline
	Fatal & Diese Fehler machen den Programmablauf unmöglich und können zum Absturz des Systems führen.  & 100\\
\hline
\end{tabularx}

% Vorgehensweise
\section{Vorgehensweise}\label{c06}
\textit{bearbeitet von: Daniel Pupat }\\

\subsection{Komponenten- und Integrationstest}


{\em Hier findet sich das konkrete Vorgehen bei der Integration: Welche
  Klassen werden zunächst zusammen getestet, welche kommen dann hinzu?
 Das kann man z.B.\ geeignet in Form eines Baumes aufzeigen.}


\subsection{Funktionstest}

Die Funktionstests sind durch jene Anwendungsfälle aus der Anforderungsspezifikation vorgegeben. Jede dieser Funktionen muss durch Tests gedeckt sein.


\section{Aufhebung und Wiederaufnahme}\label{c08}
\textit{bearbeitet von: Daniel Pupat }\\

Wir werden Tests unterbrechen, wenn ein gewisser Wert überschritten wird, welcher über die Tabelle in Abschnitt  \ref{c07} berechnet wird. In diesem Fall werden wir sofort wieder mit der Implementierung anfangen. Da wir mit der Bottom-up Strategie testen, werden wir bei Fehlern in der unteren Schicht einen niedrigeren Wert nehmen.\\
Bei Fehlern der Data setzen wir einen Wert von 10, bei Fehlern in der Logik einen Wert von 20 und bei Fehlern, welche die GUI betreffen, einen von 40 und bei den restlichen Faktoren einen von 100.\\
Sollten die Fehler behoben sein, testen wir noch einmal alle Komponenten, die mit den veränderten interagieren.

\section{Hardware- und Softwareanforderungen}\label{c09}
\textit{bearbeitet von: Daniel Pupat }\\


\subsection{Hardware}

Als Hardware stehen uns unsere Notebooks und Smartphones, sowie die Unirechner zur Verfügung. Dabei haben wir alle geforderten Betriebssysteme mindestens einmal auf unseren Notebooks installiert, sodass wir auf jeden Gerät testen können. Da Android Unterstützung gefordert ist, werden wir die App über unseren vorhandenen Smartphones, die Android haben, testen. Andere Systeme (iOS) werden wir testen, falls noch genügend Zeit vorhanden ist.

\subsection{Software}

Als Software benutzen wir in der Eclipse Umgebung JUnit-Tests. Diese werden in Form von BlackBox- und WhiteBox-Tests implementiert. Die App werden wir mithilfe eines Android Emulators und unseren Smartphones testen.

% Testfälle
\section{Testfälle}\label{c10}
\textit{bearbeitet von: Daniel Pupat }\\

\subsection{Komponententest}\label{c10-0}

Wir haben hier alle Klassen aufgelistet, welche wir testen wollen. Dabei werden wir keine abstrakten Klassen und Interfaces testen. Exceptions testen wir nicht einzeln, sondern diese werden mit den zugehörigen Methoden getestet.

\begin{table}[h]
\centering
\begin{tabular}{|l|p{3cm}|p{3cm}|l|}
\hline
Klasse & Implementierer & Tester & Testart \\
\hline
Match & Patrick  & Tim E.   & Whitebox \\
Question      & Tim W.   & Daniel   & Whitebox \\
User     & Patrick  & Olga   & Whitebox \\
Client      & Daniel  & Tobias    & Whitebox \\
ClientInbox      & Patrick  & Tim E.   & Whitebox \\
Data      & Tim W.  & Olga    & Blackbox \\
Server      & Tim W. & Patrick    & Whitebox \\
ServerDirectory      & Tim E. & Tobias    & Whitebox \\
ServerInbox      & Tobias & Daniel    & Whitebox \\
AuthRequest      & Tim E. & Olga    & Whitebox \\
LogoutRequest      & Tobias  & Patrick    & Whitebox \\
AuthResponse      & Daniel  & Tim W.   & Whitebox \\
UserListChangedResponse      & Daniel  & Tobias    & Whitebox \\
NetUtils      & Olga  & Tobias    & Whitebox \\
QuizGame      & Daniel  & Tim W.   & Whitebox \\
\hline
\end{tabular}
\caption{Komponententests}
\end{table}

\clearpage
\subsection{Integrationstest}\label{c10a}

\begin{tabular}{|l|p{12cm}|}
\hline
	Testfallbezeichner & IT-1-a Frage hinzufügen\\
\hline
	Testobjekte & Persistence, entities\\
\hline
	Eingabe & Frage(Frage, Antwort)\\
\hline
	Ausgabe & Erfolgreich Frage hinzugefügt\\
\hline
	Umgebungserfordernisse & Server läuft, Datenbank existiert, Admin angemeldet\\
\hline
	Anforderungen & keine\\
\hline
	Abhängigkeiten & keine \\
\hline
\end{tabular}

\begin{tabular}{|l|p{12cm}|}
\hline
	Testfallbezeichner & IT-1-b Frage löschen\\
\hline
	Testobjekte & Persistence, entities\\
\hline
	Eingabe & Frage\\
\hline
	Ausgabe & Erfolgreich Frage gelöscht\\
\hline
	Umgebungserfordernisse & Server läuft, Datenbank existiert, Admin angemeldet\\
\hline
	Anforderungen & Frage ist in Datenbank vorhanden\\
\hline
	Abhängigkeiten & IT-1-a \\
\hline
\end{tabular}


\begin{tabular}{|l|p{12cm}|}
\hline
	Testfallbezeichner & IT-2 User löschen\\
\hline
	Testobjekte & Persistence, entities\\
\hline
	Eingabe & User\\
\hline
	Ausgabe & Erfolgreich User gelöscht\\
\hline
	Umgebungserfordernisse & Server läuft, Datenbank existiert, Admin angemeldet\\
\hline
	Anforderungen & User ist in Datenbank vorhanden\\
\hline
	Abhängigkeiten & keine \\
\hline
\end{tabular}

\begin{tabular}{|l|p{12cm}|}
\hline
	Testfallbezeichner & IT-3 User registrieren\\
\hline
	Testobjekte & entities, Persistence, Communication\\
\hline
	Eingabe &  Registrierdaten(Nutzername, Passwort)\\
\hline
	Ausgabe & Registrierung erfolgreich\\
\hline
	Umgebungserfordernisse & App ist gestartet, Server läuft, Datenbank existiert\\
\hline
	Anforderungen & korrekte Eingabedaten\\
\hline
	Abhängigkeiten & keine \\
\hline
\end{tabular}

\begin{tabular}{|l|p{12cm}|}
\hline
	Testfallbezeichner & IT-4 Anmelden\\
\hline
	Testobjekte & entities, Persistence, Communication\\
\hline
	Eingabe &  Anmeldedaten\\
\hline
	Ausgabe & Anmeldung erfolgreich\\
\hline
	Umgebungserfordernisse & App ist gestartet, Server läuft, Datenbank existiert\\
\hline
	Anforderungen & User existiert in Datenbank\\
\hline
	Abhängigkeiten & IT-3 \\
\hline
\end{tabular}


\subsection{Funktionstest}\label{c10b}

\begin{tabular}{|l|p{13.75cm}|}
\hline
	Bezeichner: & FT-1\\
\hline
	Anwendungsfall & App starten\\
\hline
	Eingabe & Start-Button wird gedrückt\\
\hline
	Ausgabe & App ist gestartet\\ &
	Bei Fehlerfall: Rückmeldung Probleme beim starten der App\\
\hline
	Umsetzung & Start-Button wird gedrückt, Benutzer hat die App bereits heruntergeladen\\
\hline
\end{tabular}

\begin{tabular}{|l|p{13.75cm}|}
\hline
	Bezeichner: & FT-2\\
\hline
	Anwendungsfall & Offline-Modus starten\\
\hline
	Eingabe & Offline Modus starten wird gedrückt\\
\hline
	Ausgabe & Der Modus wurde gestartet, spielen nun möglich\\ &
	Bei Fehlerfall: Rückmeldung starten nicht erfolgreich\\
\hline
	Umsetzung & Auf Offline Modus starten drücken\\
\hline
\end{tabular}

\begin{tabular}{|l|p{13.75cm}|}
\hline
	Bezeichner: & FT-3\\
\hline
	Anwendungsfall & Benutzer registrieren\\
\hline
	Eingabe & Informationen des Nutzers(Nickname, Passwort)\\ 
\hline
	Ausgabe & Rückmeldung über erfolgreiches Anmelden. Nutzer ist nun in Datenbank gespeichert\\ &
	Bei Fehlerfall: Fehlerrückmeldung\\
\hline
	Umsetzung & Manuelles eintragen der Daten, dann auf registrieren klicken\\
\hline
\end{tabular}

\begin{tabular}{|l|p{13.75cm}|}
\hline
	Bezeichner: & FT-4\\
\hline
	Anwendungsfall & Benutzer anmelden\\
\hline
	Eingabe & Anmeldedaten(Nutzername, Passwort)\\
\hline
	Ausgabe & Rückmeldung über erfolgreiche Anmeldung\\ &
	Bei Fehlerfall: Rückmeldung Anmeldung nicht erfolgreich\\
\hline
	Umsetzung & Manuelle Eingabe der Nutzerdaten, Benutzer ist registriert\\
\hline
\end{tabular}

\begin{tabular}{|l|p{13.75cm}|}
\hline
	Bezeichner: & FT-5\\
\hline
	Anwendungsfall & Benutzer abmelden\\
\hline
	Eingabe & Logout-Button wird gedrückt\\
\hline
	Ausgabe & Rückmeldung über erfolgreiche Abmeldung\\ &
	Bei Fehlerfall: Rückmeldung Abmeldung nicht erfolgreich\\
\hline
	Umsetzung & Manuelles ausloggen, Benutzer ist angemeldet\\
\hline
\end{tabular}

\begin{tabular}{|l|p{13.75cm}|}
\hline
	Bezeichner: & FT-6\\
\hline
	Anwendungsfall & Spielen(Offline)\\
\hline
	Eingabe & Spielen-Button drücken\\
\hline
	Ausgabe & Spiel wird gestartet\\ &
	Bei Fehlerfall: Rückmeldung Spielen nicht möglich\\
\hline
	Umsetzung & Manuell, Benutzer spielt Offline\\
\hline
\end{tabular}

\begin{tabular}{|l|p{13.75cm}|}
\hline
	Bezeichner: & FT-7\\
\hline
	Anwendungsfall & Spielen(Online)\\
\hline
	Eingabe & Spielen-Button wird gedrückt\\
\hline
	Ausgabe & Auswahl zwischen Neues Spiel starten und offenes Spiel wird angezeigt\\ &
	Bei Fehlerfall: Rückmeldung Spielen nicht möglich\\
\hline
	Umsetzung & Manuell, Benutzer ist angemeldet\\
\hline
\end{tabular}

\begin{tabular}{|l|p{13.75cm}|}
\hline
	Bezeichner: & FT-8\\
\hline
	Anwendungsfall & Neues Spiel\\
\hline
	Eingabe & Neues Spiel wird gedrückt\\
\hline
	Ausgabe & Liste aller Spieler die Online sind erscheint\\ &
	Bei Fehlerfall: Rückmeldung Neues Spiel nicht erfolgreich\\
\hline
	Umsetzung & Manuell, Benutzer war auf Spielen\\
\hline
\end{tabular}

\begin{tabular}{|l|p{13.75cm}|}
\hline
	Bezeichner: & FT-9\\
\hline
	Anwendungsfall & Gegner herausfordern\\
\hline
	Eingabe & Gegner der herausgefordert werden soll\\
\hline
	Ausgabe & Spiel wird gestartet\\ &
	Bei Fehlerfall: Rückmeldung herausfordern nicht erfolgreich\\
\hline
	Umsetzung & Manuell, Gegner ist Online\\
\hline
\end{tabular}

\begin{tabular}{|l|p{13.75cm}|}
\hline
	Bezeichner: & FT-10\\
\hline
	Anwendungsfall & Offenes Spiel spielen\\
\hline
	Eingabe & Antwort auf die Frage\\
\hline
	Ausgabe & Nachricht ob richtige oder falsche Antwort\\ &
	Bei Fehlerfall: Rückmeldung Fehler\\
\hline
	Umsetzung & Manuell, Spiel wurde bereits angefangen\\
\hline
\end{tabular}

\begin{tabular}{|l|p{13.75cm}|}
\hline
	Bezeichner: & FT-11\\
\hline
	Anwendungsfall & Einstellungen ändern\\
\hline
	Eingabe & Benutzername, Passwort\\
\hline
	Ausgabe & Rückmeldung über erfolgreiches Ändern\\ &
	Bei Fehlerfall: Rückmeldung Ändern war nicht erfolgreich\\
\hline
	Umsetzung & Manuell, Benutzer ist angemeldet\\
\hline
\end{tabular}

\begin{tabular}{|l|p{13.75cm}|}
\hline
	Bezeichner: & FT-12\\
\hline
	Anwendungsfall & Website wird aufgerufen\\
\hline
	Eingabe & URL\\
\hline
	Ausgabe & Website wird angezeigt\\ &
	Bei Fehlerfall: Rückmeldung anzeigen war nicht erfolgreich\\
\hline
	Umsetzung & Manuell, Internet Verbindung vorhanden\\
\hline
\end{tabular}

\begin{tabular}{|l|p{13.75cm}|}
\hline
	Bezeichner: & FT-13\\
\hline
	Anwendungsfall & Admin anmelden\\
\hline
	Eingabe & Nutzername, Passwort\\
\hline
	Ausgabe & Rückmeldung, anmelden erfolgreich\\ &
	Bei Fehlerfall: Rückmeldung anmelden war nicht erfolgreich\\
\hline
	Umsetzung & Manuell\\
\hline
\end{tabular}

\begin{tabular}{|l|p{13.75cm}|}
\hline
	Bezeichner: & FT-14\\
\hline
	Anwendungsfall & Admin abmelden\\
\hline
	Eingabe & Logout-Button drücken\\
\hline
	Ausgabe & Rückmeldung abmelden erfolgreich\\ &
	Bei Fehlerfall: Rückmeldung abmelden nicht erfolgreich\\
\hline
	Umsetzung & Manuell, Admin ist angemeldet\\
\hline
\end{tabular}

\begin{tabular}{|l|p{13.75cm}|}
\hline
	Bezeichner: & FT-15\\
\hline
	Anwendungsfall & Frage hinzufügen\\
\hline
	Eingabe & Frageinformationen(Frage, Antworten)\\
\hline
	Ausgabe & Rückmeldung über erfolgreiches Hinzufügen\\ &
	Bei Fehlerfall: Rückmeldung Hinzufügen war nicht erfolgreich\\
\hline
	Umsetzung & Manuell, Benutzer ist Admin\\
\hline
\end{tabular}

\begin{tabular}{|l|p{13.75cm}|}
\hline
	Bezeichner: & FT-16\\
\hline
	Anwendungsfall & Frage bearbeiten\\
\hline
	Eingabe & Frageinformationen(Frage, Antworten)\\
\hline
	Ausgabe & Rückmeldung über erfolgreiches Ändern\\ &
	Bei Fehlerfall: Rückmeldung Ändern nicht erfolgreich\\
\hline
	Umsetzung & Manuell, Benutzer ist Bibliothekar\\
\hline
\end{tabular}

\begin{tabular}{|l|p{13.75cm}|}
\hline
	Bezeichner: & FT-17\\
\hline
	Anwendungsfall & Frage löschen\\
\hline
	Eingabe & zu löschende Frage\\
\hline
	Ausgabe & Rückmeldung über erfolgreiches Löschen\\ &
	Bei Fehlerfall: Rückmeldung Löschen war nicht erfolgreich\\
\hline
	Umsetzung & Manuell, Benutzer ist Admin\\
\hline
\end{tabular}

\begin{tabular}{|l|p{13.75cm}|}
\hline
	Bezeichner: & FT-18\\
\hline
	Anwendungsfall & Frageliste importieren\\
\hline
	Eingabe & Frageliste\\
\hline
	Ausgabe & Rückmeldung über erfolgreiches importieren\\ &
	Bei Fehlerfall: Rückmeldung Importieren war nicht erfolgreich\\
\hline
	Umsetzung & Manuell, Benutzer ist Admin\\
\hline
\end{tabular}

\begin{tabular}{|l|p{13.75cm}|}
\hline
	Bezeichner: & FT-19\\
\hline
	Anwendungsfall & Frageliste exportieren\\
\hline
	Eingabe & Frageliste\\
\hline
	Ausgabe & Rückmeldung über erfolgreiches exportieren\\ &
	Bei Fehlerfall: Rückmeldung Exportieren nicht erfolgreich\\
\hline
	Umsetzung & Manuell, Benutzer ist Admin\\
\hline
\end{tabular}

\begin{tabular}{|l|p{13.75cm}|}
\hline
	Bezeichner: & FT-20\\
\hline
	Anwendungsfall & User löschen\\
\hline
	Eingabe & User\\
\hline
	Ausgabe & Rückmeldung über erfolgreiches Löschen\\ &
	Bei Fehlerfall: Rückmeldung Löschen war nicht erfolgreich\\
\hline
	Umsetzung & Manuell, User existiert\\
\hline
\end{tabular}

\begin{tabular}{|l|p{13.75cm}|}
\hline
	Bezeichner: & FT-21\\
\hline
	Anwendungsfall & Passwort ändern\\
\hline
	Eingabe & Passwort\\
\hline
	Ausgabe & Rückmeldung über erfolgreiches Ändern\\ &
	Bei Fehlerfall: Rückmeldung Ändern war nicht erfolgreich\\
\hline
	Umsetzung & Manuell, Admin ist angemeldet\\
\hline
\end{tabular}





\subsection{Leistungstest}\label{c10c}

\subsubsection{Härtetest}

\begin{tabular}{|l|p{13.75cm}|}
\hline
	Bezeichner: & LT-1\\
\hline
	Beschreibung & 100 Nutzer starten gleichzeitig ein Spiel\\
\hline
	Ziel: & Robustheit der Datenbank mit vielen Anfragen umzugehen wird getestet\\
\hline
	Bei Erfolg: & Datenbank kann Anfragen bearbeiten; es gibt keine langen Wartezeiten\\
\hline
	Fehler: & TimeOut, Absturz\\
\hline
\end{tabular}


\subsubsection{Volumentest}

\begin{tabular}{|l|p{13.75cm}|}
\hline
	Bezeichner: & LT-2\\
\hline
	Beschreibung & Eine sehr große CSV-Datei wird importiert\\
\hline
	Ziel: & Robustheit der Datenbank mit großen Datenmengen umzugehen wird getestet\\
\hline
	Bei Erfolg: & Datenbank kann mit der Verarbeitung umgehen; es gibt keine Fehler\\
\hline
	Fehler: & TimeOut, Absturz\\
\hline
\end{tabular}


\subsubsection{Sicherheitstest}

\begin{tabular}{|l|p{13.75cm}|}
\hline
	Bezeichner: & LT-3\\
\hline
	Beschreibung & Ein Nutzer gibt ein falsches Passwort ein(App und Website)\\
\hline
	Ziel: & Korrekte Authentifizierung wird getestet\\
\hline
	Bei Erfolg: & Nutzer kann sich nicht einloggen; System gibt Fehlermeldung\\
\hline
	Fehler: & Benutzer kann sich anmelden\\
\hline
\end{tabular}

\subsubsection{Erholungstest}

\begin{tabular}{|l|p{13.75cm}|}
\hline
	Bezeichner: & LT-4\\
\hline
	Beschreibung & Ein Nutzer gibt mehrmals ein falsches Passwort ein\\
\hline
	Ziel: & Erholt sich das System; Kann man sich danach problemlos mit dem richtigen Passwort einloggen\\
\hline
	Bei Erfolg: & Bei richtiger Eingabe der Logindaten, ist man eingeloggt\\
\hline
	Fehler: & Man kann sich nicht mehr einloggen, Absturz\\
\hline
\end{tabular}

\begin{tabular}{|l|p{13.75cm}|}
\hline
	Bezeichner: & LT-5\\
\hline
	Beschreibung & Der Server geht offline und startet die Verbindung neu\\
\hline
	Ziel: & Ist das System nach neuer Verbindung wieder funktionstüchtig\\
\hline
	Bei Erfolg: & System produziert keine Fehler\\
\hline
	Fehler: & Funktionen werden nicht mehr unterstützt, Datenverlust, TimeOut, Absturz\\
\hline
\end{tabular}


\section{Testzeitplan}
\label{sec:testzeitplan}
Komponententests: Woche ab 28.7.2014\\
Integrationstests: Woche ab 28.7.2014\\
Funktionstests: Woche ab 28.7.2014\\
Leistungstests: Woche ab 04.8.2014\\

\end{document}
