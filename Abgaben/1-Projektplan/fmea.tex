	\begin{center}
		{
			\tiny
				\begin{tabular}{|p{1.5cm}|p{1.5cm}|p{1.7cm}|p{1.5cm}|p{0.25cm}|p{0.25cm}|p{0.25cm}|p{0.5cm}|p{1.5cm}|p{1.5cm}|} \hline

					\begin{center}\multirow{2}*{\parbox{5cm}{Systeme, \newline Merkmale}}\end{center}	&
					\begin{center}\multirow{2}*{\parbox{5cm}{potentielle \newline Fehler}}\end{center}	&
					\begin{center}\multirow{2}*{\parbox{5cm}{potentielle Fol-\newline gen des Fehlers}}\end{center}	&
					\begin{center}\multirow{2}*{\parbox{5cm}{potentielle \newline Fehlerursache}}\end{center}	&
					\multicolumn{4}{c|}{\textbf{Derzeitiger Zustand}}	&
					\begin{center}\multirow{2}*{\parbox{5cm}{Empfohlene \newline Abstellma\ss{}-\newline nahmen}}\end{center}	&
					\begin{center}\multirow{2}*{\parbox{5cm}{Verantwort- \newline lichkeit}}\end{center}	\\ \cline{5-8}
					
						&
						&
						&
						&
					%{Vorgesehene Pr\"{u}fma\ss{}nahmen}	&
					{\rotatebox{90}{Auftreten}}	&
					{\rotatebox{90}{Bedeutung}}	&
					{\rotatebox{90}{Entdeckung \quad}}	&
					RPZ	&
						&	\\ \hline
					
% Ab hier Inhalt einfuegen! Die Reihenfolge ist einmalig vorgegeben.
%
% Ein Zeilenumbruch in einer Tabellen-zelle wird mit einem simplen Bindestrich an der gewuenschten Stelle
% zusammen mit dem \newline -Befehl erzeugt.
%
%
% Copy&Paste VOR(!!!!) dem Eintragen der Inhalte empfohlen :)
% Zu entnehmen der Datei CundP.txt
%
% Hoechstwahrscheinlich werden mehrere Seiten mit dieser Tabelle noetig sein. Dazu muss der Quellcode von Zeile 5 an bis 33
% jedes Mal fuer eine Neue Seite mit kopiert werden. Im Klartext heisst das, dass es , es denn LaTeX ist cool und kann das, dass
% es nicht eine Tabelle ueber mehrere Seiten sondern viele Seiten mit mehreren Tabellen geben wrid. Das Muesste man dann ausprobieren.
%
% Bislang funktioniert es aber. Lediglich der Inhalt, den ich in die Tabelle geschrieben habe, ist vermutlich Bullshit.
%
%
%
% Systeme und Merkmale
{
	Rechner-Arbeitspl\"{a}tze und Laptops
}&

% potentielle Fehler
{
	ungesch\"{u}tzt
}&

% potentielle Folgen des Fehlers
{
	nicht autentifizierter Zugriff auf sensible Daten
}&

% potentielle Fehlerursache
{
	vers\"{a}mt den Bildschirm zu sperren
}&

% Auftreten
{
	2
}&

% Bedeutung
{
	5
}&

% Enddeckung
{
	5
}&

% RPZ (das Produkt aus den vorigen 3 Zahlen)
{
	50
}&

% Empfohlene Abstellmassnahmen
{
	NFC am K\"{o}rper
}&

% Verantwortlichkeit
{
	Jeder User selbst
}\\ \hline
%
%
% Neue Zeile
% Systeme und Merkmale
{
	Rechner-Arbeitspl\"{a}tze und Laptops
}&

% potentielle Fehler
{
	Diebstahl
}&

% potentielle Folgen des Fehlers
{
	Entwendung von Daten
}&

% potentielle Fehlerursache
{
	Laptop nicht physisch fest verbunden mit z.B. Tisch
}&

% Auftreten
{
	2
}&

% Bedeutung
{
	5
}&

% Enddeckung
{
	10
}&

% RPZ
{
	100
}&

% Empfohlene Abstellmassnahmen
{
	Kensington-Schlo\ss{}
}&

% Verantwortlichkeit
{
	der User selbst
}\\ \hline
%
%
% Neue Zeile
% Systeme und Merkmale
{
	Lasten- und Pflichtenheft
}&

% potentielle Fehler
{
	Verlust
}&

% potentielle Folgen des Fehlers
{
	keine Chance auf Fertigstellung des Auftrages
}&

% potentielle Fehlerursache
{
	Datenverlust
}&

% Auftreten
{
	2
}&

% Bedeutung
{
	6
}&

% Enddeckung
{
	10
}&

% RPZ
{
	200
}&

% Empfohlene Abstellmassnahmen
{
	Sicherungs- \newline kopien
}&

% Verantwortlichkeit
{
	Projektleiter
}\\ \hline
%
%
% Neue Zeile
% Systeme und Merkmale
{
	Kundenges-\newline pr\"{a}ch
}&

% potentielle Fehler
{
	findet nicht statt
}&

% potentielle Folgen des Fehlers
{
	Keine genauen Informationen zur Vollendung des Projektes bzw. keine Hintergrundinformationen zu gegebenen Sachverhalten
}&

% potentielle Fehlerursache
{
	Termin vers\"{a}umt
}&

% Auftreten
{
	1
}&

% Bedeutung
{
	10
}&

% Enddeckung
{
	10
}&

% RPZ
{
	100
}&

% Empfohlene Abstellmassnahmen
{
	Terminplan
}&

% Verantwortlichkeit
{
	Projektleiter
}\\ \hline
%
%
% Neue Zeile
% Systeme und Merkmale
{
	Lasten- und Pflichtenheft
}&

% potentielle Fehler
{
	W\"{u}nsche des Kunden nicht verstanden
}&

% potentielle Folgen des Fehlers
{
	unzufriedener Kunde
}&

% potentielle Fehlerursache
{
	Anforderungen falsch verstanden
}&

% Auftreten
{
	5
}&

% Bedeutung
{
	9
}&

% Enddeckung
{
	8
}&

% RPZ
{
	360
}&

% Empfohlene Abstellmassnahmen
{
	regelm\"{a}\ss{}ige Kommunikation mit dem Kunden
}&

% Verantwortlichkeit
{
	Projekt- und/oder Teamleiter
}\\ \hline
%
%
% Neue Zeile
% Systeme und Merkmale
{
	IST-Analyse
}&

% potentielle Fehler
{
	Fehler beim Analysieren des aktuellen Vorgehens des Kunden
}&

% potentielle Folgen des Fehlers
{
	
}&	

% potentielle Fehlerursache
{
	inkompetenter Mitarbeiter \par\medskip
	fehlerhafte bzw. ungenaue Beschreibung des Kunden
}&

% Auftreten
{
	5
}&

% Bedeutung
{
	4
}&

% Enddeckung
{
	6
}&

% RPZ (das Produkt aus den vorigen 3 Zahlen)
{
	120
}&

% Empfohlene Abstellmassnahmen
{
	Vier-Augen Prinzip
}&

% Verantwortlichkeit
{
	Projektleiter
}\\ \hline
%
%
% Neue Zeile
% Systeme und Merkmale
{
	Prototyp
}&

% potentielle Fehler
{
	nicht akzeptiert
}&

% potentielle Folgen des Fehlers
{
	
}&

% potentielle Fehlerursache
{
	nicht gem\"{a}\ss{} den W\"{u}nschen des Kunden
}&

% Auftreten
{
	7
}&

% Bedeutung
{
	3
}&

% Enddeckung
{
	3
}&

% RPZ (das Produkt aus den vorigen 3 Zahlen)
{
	63
}&

% Empfohlene Abstellmassnahmen
{
	Prototyp m\"{o}glichst nahe am gew\"{u}nschten Produkt entwickeln
}&

% Verantwortlichkeit
{
	Projektleiter
}\\ \hline
%
%
% Neue Zeile
% Systeme und Merkmale
{
	SOLL-Analyse
}&

% potentielle Fehler
{
	Problem- \newline stellungen falsch erfasst und/oder interpretiert
}&

% potentielle Folgen des Fehlers
{
	falsche Modelle f\"{u}r weiteres Vorgehen
}&

% potentielle Fehlerursache
{
	fehlerhafte Problembeschreibung \par\medskip
	fehlerhafte Dokumentation
}&

% Auftreten
{
	4
}&

% Bedeutung
{
	8
}&

% Enddeckung
{
	8
}&

% RPZ (das Produkt aus den vorigen 3 Zahlen)
{
	256
}&

% Empfohlene Abstellmassnahmen
{
	Vier-Augen Prinzip
}&

% Verantwortlichkeit
{
	Projektleiter
}\\ \hline
%
%
% Neue Zeile
% Systeme und Merkmale
{
	Lastenheft
}&

% potentielle Fehler
{
	nicht alle Kundenw\"{u}nsche ber\"{u}cksichtigt
}&

% potentielle Folgen des Fehlers
{
	kein Vertrag \newline
	verz\"{o}gerter Vertrag
}&

% potentielle Fehlerursache
{
	fehlerhafte Analysen
}&

% Auftreten
{
	3
}&

% Bedeutung
{
	9
}&

% Enddeckung
{
	6
}&

% RPZ (das Produkt aus den vorigen 3 Zahlen)
{
	162
}&

% Empfohlene Abstellmassnahmen
{
	regelm\"{a}\ss{}ige Kommunikation mit dem Kunden
}&

% Verantwortlichkeit
{
	Projektleiter
}\\ \hline
% Neue Zeile
% Systeme und Merkmale
{
	Lastenheft
}&

% potentielle Fehler
{
	ist inkonsistent
}&

% potentielle Folgen des Fehlers
{
	falsche Modellbildung
}&

% potentielle Fehlerursache
{
	fehlerhafte Analysen
}&

% Auftreten
{
	5
}&

% Bedeutung
{
	7
}&

% Enddeckung
{
	9
}&

% RPZ (das Produkt aus den vorigen 3 Zahlen)
{
	315
}&

% Empfohlene Abstellmassnahmen
{
	Pr\"{u}fung durch unabh\"{a}ngigen Fachmann
}&

% Verantwortlichkeit
{
	Projektleiter
}	\\ \hline

% Systeme und Merkmale
{
	Pflichtenheft
}&

% potentielle Fehler
{
	unvollst\"{a}ndig
}&

% potentielle Folgen des Fehlers
{
	fehlerhafter Prototyp
}&

% potentielle Fehlerursache
{
	fehlerhafte Anforderungsanalyse
}&

% Auftreten
{
	8
}&

% Bedeutung
{
	4
}&

% Enddeckung
{
	6
}&

% RPZ (das Produkt aus den vorigen 3 Zahlen)
{
	192
}&

% Empfohlene Abstellmassnahmen
{
	detailreiche Kommunikation mit dem Kunden
}&

% Verantwortlichkeit
{
	Projektleiter und Entwickler
}	\\ \hline
%
%
% Neue Zeile
% Systeme und Merkmale
{
	Pflichtenheft
}&

% potentielle Fehler
{
	inkonsistent
}&

% potentielle Folgen des Fehlers
{
	fehlerhafter Prototyp
}&

% potentielle Fehlerursache
{
	
}&

% Auftreten
{
	4
}&

% Bedeutung
{
	4
}&

% Enddeckung
{
	7
}&

% RPZ (das Produkt aus den vorigen 3 Zahlen)
{
	112
}&

% Empfohlene Abstellmassnahmen
{
	Pr\"{u}fung durch unabh\"{a}ngigen Fachmann
}&

% Verantwortlichkeit
{
	Projektleiter und Entwickler
}	\\ \hline
% Systeme und Merkmale
{
	UML-Diagramm
}&

% potentielle Fehler
{
	inkonsistent
}&

% potentielle Folgen des Fehlers
{
	falsche Implementierung
}&

% potentielle Fehlerursache
{
	fehlerhafte SOLL-Analyse
}&

% Auftreten
{
	4
}&

% Bedeutung
{
	8
}&

% Enddeckung
{
	9
}&

% RPZ (das Produkt aus den vorigen 3 Zahlen)
{
	288
}&

% Empfohlene Abstellmassnahmen
{
	Vier-Augen Prinzip
}&

% Verantwortlichkeit
{
	Entwickler
}	\\ \hline
%
%
% Neue Zeile
\end{tabular}} \newpage
		{
			\tiny
				\begin{tabular}{|p{1.5cm}|p{1.5cm}|p{1.7cm}|p{1.5cm}|p{0.25cm}|p{0.25cm}|p{0.25cm}|p{0.5cm}|p{1.5cm}|p{1.5cm}|} \hline

					\begin{center}\multirow{2}*{\parbox{5cm}{Systeme, \newline Merkmale}}\end{center}	&
					\begin{center}\multirow{2}*{\parbox{5cm}{potentielle \newline Fehler}}\end{center}	&
					\begin{center}\multirow{2}*{\parbox{5cm}{potentielle Fol-\newline gen des Fehlers}}\end{center}	&
					\begin{center}\multirow{2}*{\parbox{5cm}{potentielle \newline Fehlerursache}}\end{center}	&
					\multicolumn{4}{c|}{\textbf{Derzeitiger Zustand}}	&
					\begin{center}\multirow{2}*{\parbox{5cm}{Empfohlene \newline Abstellma\ss{}-\newline nahmen}}\end{center}	&
					\begin{center}\multirow{2}*{\parbox{5cm}{Verantwort- \newline lichkeit}}\end{center}	\\ \cline{5-8}
					
						&
						&
						&
						&
					%{Vorgesehene Pr\"{u}fma\ss{}nahmen}	&
					{\rotatebox{90}{Auftreten}}	&
					{\rotatebox{90}{Bedeutung}}	&
					{\rotatebox{90}{Entdeckung \quad}}	&
					RPZ	&
						&	\\ \hline
% Neue Zeile

% Systeme und Merkmale
{
	Implementier- \newline ung
}&

% potentielle Fehler
{
	nicht gem\"{a}\ss{} der Anforderungen des Lastenheftes
}&

% potentielle Folgen des Fehlers
{
	Kunde lehnt das Produkt ab
}&

% potentielle Fehlerursache
{
	Fehler im UML Diagramm
}&

% Auftreten
{
	4
}&

% Bedeutung
{
	9
}&

% Enddeckung
{
	7
}&

% RPZ (das Produkt aus den vorigen 3 Zahlen)
{
	252
}&

% Empfohlene Abstellmassnahmen
{
	F\"{a}higkeiten der einzelnen Mitarbeiter kennen
}&

% Verantwortlichkeit
{
	Entwickler
}	\\ \hline
%
%
% Neue Zeile
% Systeme und Merkmale
{
	Implementier- \newline ung
}&

% potentielle Fehler
{
	Implementier- \newline ung nicht geforderter bzw. nicht gew\"{u}nschter Inhalte
}&

% potentielle Folgen des Fehlers
{
	
}&

% potentielle Fehlerursache
{
	ungenaues Abhandeln des Lasten- und Pflichtenheftes
}&

% Auftreten
{
	2
}&

% Bedeutung
{
	6
}&

% Enddeckung
{
	6
}&

% RPZ (das Produkt aus den vorigen 3 Zahlen)
{
	72
}&

% Empfohlene Abstellmassnahmen
{
	erneute Absprache mit Kunden anhand Prototyp
}&

% Verantwortlichkeit
{
	Entwickler
}	\\ \hline
%
%
% Neue Zeile
% Systeme und Merkmale
{
	Testf\"{a}lle
}&

% potentielle Fehler
{
	falsche Testmodelle
}&

% potentielle Folgen des Fehlers
{
	Testen des Produktes wird verz\"{o}gert
}&

% potentielle Fehlerursache
{
	Fehlerhaftes Modell \par\medskip
	inkompetente Mitarbeiter
}&

% Auftreten
{
	3
}&

% Bedeutung
{
	4
}&

% Enddeckung
{
	9
}&

% RPZ (das Produkt aus den vorigen 3 Zahlen)
{
	108
}&

% Empfohlene Abstellmassnahmen
{
	sich \"{u}ber die F\"{a}higkeiten der einzelnen Mitarbeiter bewusst sein
}&

% Verantwortlichkeit
{
	Entwickler
}	\\ \hline
%
%
% Neue Zeile
% Systeme und Merkmale
{
	Abnahme
}&

% potentielle Fehler
{
	nicht akzeptiert
}&

% potentielle Folgen des Fehlers
{
	Kunde lehnt weitere Zusammenarbeit ab
}&

% potentielle Fehlerursache
{
	dem Kunden gef\"{a}llt das Ergebnis nicht
}&

% Auftreten
{
	2
}&

% Bedeutung
{
	10
}&

% Enddeckung
{
	10
}&

% RPZ (das Produkt aus den vorigen 3 Zahlen)
{
	200
}&

% Empfohlene Abstellmassnahmen
{
	
}&

% Verantwortlichkeit
{
	Projektleiter
}	\\ \hline
%
%
% Neue Zeile
% Systeme und Merkmale
{
	Zeitmanage- \newline ment
}&

% potentielle Fehler
{
	stimmt nicht
}&

% potentielle Folgen des Fehlers
{
	ein oder mehrere Arbeitspakete z\"{o}gern sich in ihrer Fertigkstellung hinaus
}&

% potentielle Fehlerursache
{
	falsch eingesch\"{a}tzt
}&

% Auftreten
{
	6
}&

% Bedeutung
{
	9
}&

% Enddeckung
{
	7
}&

% RPZ (das Produkt aus den vorigen 3 Zahlen)
{
	378
}&

% Empfohlene Abstellmassnahmen
{
	h\"{o}here Pufferzeiten einbauen und parallelisierte APs
}&

% Verantwortlichkeit
{
	Projektleiter
}	\\ \hline
%
%
% Neue Zeile
% Systeme und Merkmale
{
	Krankheits- \newline f\"{a}lle
}&

% potentielle Fehler
{
	Mitarbeiter erkranken
}&

% potentielle Folgen des Fehlers
{
	zu erledigende Arbeit kann nicht fertiggestellt werden
}&

% potentielle Fehlerursache
{
	Grippe o.\"{a}.
}&

% Auftreten
{
	5
}&

% Bedeutung
{
	2
}&

% Enddeckung
{
	1
}&

% RPZ (das Produkt aus den vorigen 3 Zahlen)
{
	10
}&

% Empfohlene Abstellmassnahmen
{
	vorausschau- \newline ende Arbeitsverteilung bzw. Umverteilung
}&

% Verantwortlichkeit
{
	Projekt- und/oder Teamleiter
}	\\ \hline
%
%
% Neue Zeile

\end{tabular}}
	\end{center}
	
\subsection{Projektüberwachung}\label{3.4-controlling}
\textit{Bearbeitet von: Sylvia Kamche Tague }\\
Für die Fortschrittsüberwachung des Projektes werden wir uns mindestens einmal pro Woche nach Absprache treffen. 
Der Phasenleiter wird das Treffen moderiert. Er prüft auch, ob der Zeitplan angehaltet ist oder nicht. 
Jeder wird sagen, was er vom letzten Treffen bis diesem gemacht hat, und was er für Probleme getroffen hat.
Also können die anderen auch eine Lösung dafür geben, falls sie eine Ahnung haben. Dann soll man auch mitteilen,
was er als weiteres machen wird. Es wird jedes Treffen ein Protokollant gewählt. Der schreibt alles was noch zu machen ist 
und schickt später eine Mail an alle.Wenn eine Thematik nicht abschließend behandelt werden kann,
wird diese auf einen späteren Zeitpunkt vertagt (nächstes Meeting oder zu einem anderen geeigneten Zeitpunkt). Der Phasenleiter ist durch
diese Meetings besser dazu in der Lage, die Aufgaben zu delegieren und mögliche Behinderungen zu beseitigen. 
Sollten während der Arbeit Probleme auftreten, werden diese direkt dem Phasenleiter gemeldet, 
sodass er in der Lage ist frühzeitig entsprechende Maßnahmen zu ergreifen.

\subsection{Mitarbeiter}
\textit{Bearbeitet von: Sylvia Kamche Tague }\\
Die technische Kompetenzen von Projektmitgliedern sind folgende:\\
\\
\begin{tabular}{l|l}
Name & Kompetenzen\\
\hline

Jannes Uken & Java, Latex, SQL, HTML, C, Android, PHP\\
Sylvia Kamche Tague & Latex, Java, SQL,  Android, C++\\
Dario Treffenfeld-Mäder & Latex, Java, MySQL, Webprogrammierung, Android, C \\
Sandor Herms & Java, Latex, SQL, HTML, C\\
Olga Miloevich & Java, Latex, SQL, Android, C++\\
Nils Sören Oja & Java, Latex, Haskell, HTML, PHP, C\\

\end{tabular}