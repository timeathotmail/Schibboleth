\section{Technische Prozesse}
\subsection{Methoden, Werkzeuge und Techniken}
\subsubsection{Entwicklungsplattform}
Zur Erstellung und Verwaltung der einzelnen Dateien, nutzen wir die IDE "`Eclipse Kepler"'. \\
Mithilfe des Plug-ins "`Android Developer Tools"' in der Version 1.0.0.\_8-7ScKtmH6E8McIScJXJXcITh9wdNni4 können wir direkt in der IDE für Android Software entwickeln. \\
UML-Diagrammen erstellen wir mit "`Dia"' Version 0.97.2 \\
Die Dokumenten-Erstellung erfolgt mithilfe des LaTex-Editors "`TeXstudio"' Version 2.7, und den dazu benötigten TeX-Paketen, welche von der Software "`MiKTex"' bereitgestellt werden. \\
Um von überall auf unsere Dateien zugreifen zu können, nutzen wir ein Git-Repository bei "`GitHub"'. \\
Für die komplexe Zeitplanung erstellen wir mithilfe von "`Microsoft Projekt"' ein Gantt-Diagramm. 

\subsubsection{Entwicklungsmethode}

Wir werden bei unserer Entwicklung dem Wasserfallmodell folgen und objektorientiert programmieren. 

\subsubsection{Programmiersprache und Bibliotheken}

Die Software wird mit Java auf Basis des "`Java SE Development Kit (JDK) 7"' entwickelt.\\
Zudem benutzen wir eine leichtgewichtige SQLite-Datenbank, welche wir mithilfe des von Oracle bereitgestellten "`java.sql"' Paketes verwenden können.\\

\subsection{Dokumentationsplan}

Klassen, Methoden und Variablen werden zeitnah mit der Implementierung dokumentiert.

\subsubsection{Codingstyle}

Wir folgen den "`Development Conventions and Guidelines"' von Eclipse.

\subsubsection{Kommentarsprache}

Die Kommentare werden in Englisch geschrieben, um Probleme mit Sonderzeichen zu vermeiden und das Verständnis des Source Codes, bei der Wartung oder Weiterentwicklung, auch nicht deutschsprachigen zu erleichtern bzw. zu ermöglichen.

\subsubsection{JavaDoc}

JavaDoc Kommentare werden für alle öffentlichen Klassen, Methoden und Variablen auf Englisch verfasst, und dienen der Übersicht und schnellerem Verstehen der Zusammenhänge des Projekts.
Private Klassen, Methoden und Variablen werden nur mit JavaDoc Kommentaren versehen, wenn diese Komplex oder für das Verständnis wesentlich sind.

\subsubsection{Begleitende Dokumentation}

Begleitend werden eine Installationsanleitung für die Anwendung und den Server, ein Benutzerhandbuch zur Bedienung und eine Testprotokoll zum schnellen Überblick über durchgeführte Test und deren Ergebnis geschrieben.

\subsection{Unterstützende Projektfunktionen}
%{\em Wie wird Euer Konfigurationsmanagement funktionieren? Wer ist verantwortlich? Benötigt Ihr dazu Ressourcen oder Zeit? Hier könnte z.B.\ der Pfad zum Repository angegeben werden. Plant Ihr Datensicherung?}

Alle erstellten Java-, SQL- und Latex-Dateien werden im Git Repository gespeichert. Dadurch haben wird nicht nur eine Datensicherung, sondern auch eine Versionskontrolle. Das Repository wird beim ersten Gruppentreffen erstellt und anschließend wird der Pfad zum Repository an dieser Stelle hinzugefügt. Das Repository liegt unter "`https://github.com/timeathotmail/ReSWP.git"' .\\
Zur Qualitätssicherung wird jedes Arbeitspaket nach Fertigstellung auf Erfüllung der Anforderungen überprüft, ob es unseren internen Richtlinien entspricht und wird gegebenenfalls noch einmal vom Qualitäts-Manager für eine Überarbeitung empfohlen. Dafür sind nur wenige Arbeitsstunden vorgesehen, wenn der jeweils zuständige nicht selbst an Arbeitsprodukten arbeitet.

%{\em Gibt es Maßnahmen zur Qualitätssicherung? Wer ist zuständig?
%  Wieviel Zeit ist dafür vorgesehen?}