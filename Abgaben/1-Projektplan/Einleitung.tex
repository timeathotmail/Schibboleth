\textit{Bearbeitet von Daniel Pupat}
		
		
\section{Einleitung}

Dieses Dokument ist ein Projektplan für das Modul Softwareprojekt2 im Sommersemester 2014 an der Universität Bremen. Der Projektplan entspricht der Struktur ANSI/IEEE Std. 1058.1-1987\footnote{\url{http://ieeexplore.ieee.org/stamp/stamp.jsp?tp=&arnumber=25325&userType=inst}}.

\subsection{Projektübersicht}


\subsubsection{Ziele}

Das Ziel unserer Gruppe ist es, das Softwareprojekt 2 der Universität Bremen zu bestehen. Dies setzt die Einhaltung der Fristen und Termine, eine ausreichende Fertigstellung des Projekts und die Abgabe aller in SWP2 geforderten Dokumente wie Projektplan, Anforderungsspezifikation und Angebot, Architekturbeschreibung, Schnittstellenbeschreibung, Testplan inklusive Blackbox-Tests und ein elektronisch geführtes Berichtsheft voraus. Darüber hinaus wollen wir einen GUI-Prototypen erstellen und den Akzeptanztest bestehen. Ein Quiz-App zu erstellen steht aber im Vordergrund.

Die Quiz-App benötigt eine Website für die Kundin, sowie einen Zugang für mobile Geräte in Form einer App.
 Ziel ist es geforderte Mindestanforderungen, wie in der Kick-off Veranstaltung vorgestellt und eventuell weitergehende Funktionen zu implementieren.

Zu den Mindestanforderungen gehören die Erstellung und Abgabe einer Quiz-App für mobile Geräte und ein Serverprogramm mit Datenbankanbindung. Die App soll später allen Studenten der Universität Bremen zur Verfügung stehen. Zusätzlich soll die Auftraggeberin über eine Website neue Fragen erstellen oder alte bearbeiten oder löschen können.

\subsubsection{Hauptarbeitsaktivitäten und --produkte}

In einem Softwareprojekt wird der Entwicklungsprozess einer Software in verschiedene Phasen unterteilt.\\
Da das Projekt aus verschiedenen Aktivitäten besteht, lassen sich diese Aktivitäten zu Arbeitsprodukten zusammenfügen. Die Folgende Tabelle bietet eine Übersicht der einzelnen Aktivitäten und den daraus resultierenden Arbeitsprodukten. Die Arbeitsprodukte werden im Laufe des Projektes nach und nach abgegeben.\\

\begin{table}[htbp]
\caption{Hauptaktivitäten und --produkte}
\centering
\begin{tabular}{p{7cm}|p{7cm}}
\hline Aktivität / Phase & Arbeitsprodukt \\ \hline
\hline Projektplanung & Projektplan\\
\hline Anforderungsanalyse, Angebotserstellung & Anforderungsspezifikation, Angebot\\
\hline Entwurf (Globale Analyse, Konzeptionelles Modell, Modulblickwinkel, Ausführungsblickwinkel, Codeblickwinkel) & Architekturbeschreibung\\
\hline erstellen des Testplans, Tests & Testplan, Schnittstellentests\\
\hline Implementierung & lauffähiges Programm\\
\hline Dokumentation & Installationsanweisung/-Skript\\
\hline Auslieferung & Kunde erhält Produkt\\
\hline 
\end{tabular}
\end{table}

\newpage

\subsubsection{Haupt-Meilensteine und grober Zeitplan}\label{meilensteine}

Die Haupt-Meilensteine resultieren aus den jeweiligen Abgabeterminen der einzelnen Dokumente.

\begin{description}
\item[M0 - 23.04.2014] Kick-Off Veranstaltung.

\item[M1 - 24.04.2014] Beginn des Projektes.

\item[M2 - 04.05.2014] Abgabe initialer Projektplan.\\
Jedes Mitglied muss seinen Teil fertig gestellt haben. Anschließend werden alle Einzelteile zusammengeführt und von allen auf Korrektheit geprüft.

\item[M3 - 08.05.2014] Kundengespräch.

\item[M4 - 27.05.2014] Anforderungsspezifikation (Intern). \\
Jedes Mitglied hat seinen Teil der Anforderungsspezifikation fertiggestellt. Anschließend werden die Teile zusammengeführt und von allen auf Korrektheit geprüft.

\item[M5 - 01.06.2014] Abgabe der Anforderungsspezifikation, GUI-Protoyp und Angebot. \\
Meilenstein 4 muss bereits fertig sein. Der GUI-Prototyp muss vollständig entwickelt sein. Abgabe via MEMS.

\item[M6 - 01.07.2014] Architektur- und Schnittstellenbeschreibung, Testplan, Tests (Intern).\\
Jedes Mitglied muss seine Aufgaben erfüllt haben. Teile werden zusammengeführt und kontrolliert. Tests müssen implementiert sein.

\item[M7 - 06.07.2014] Architekturbeschreibung, Testplan und Schnittstellentests fertig.\\
Meilenstein 7 muss bereits erreicht worden sein. Tests wurden lauffähig implementiert. Abgabe via MEMS.

\item[M8 - 28.07-01.08.2014] Akzeptanztest.

\item[M11 - 10.08.2014] Vollständige Abgabe der Dokumente und der Software. \\
Die Software muss lauffähig und vollständig implementiert sein,\\
Abgabe des Build-/Installationsskriptes
\end{description}

\subsubsection{Benötigte Ressourcen}

\begin{itemize}
\item \textbf{Menschliche Ressourcen}

An menschlichen Ressourcen stehen sechs Informatikstudenten der Universität Bremen zur Verfügung. Wir haben als durchschnittliche Arbeitszeit pro Woche und Person einen Aufwand von ca. 17 Stunden für das Projekt errechnet. Dieser Wert ergibt sich folgendermaßen:\\
Für das Modul Software Projekt 2 gibt es 9CP. 1CP entspricht 30 Semesterwochenstunden. 9 x 30 = 270 Stunden. Da wir 16 Wochen lang an dem Projekt arbeiten werden, ergibt sich ein aufgerundeter Wert von 17 Stunden pro Woche (270 / 16 = 16,875). Unsere Kontaktdaten sind dem Punkt Mitarbeiter im Abschnitt \ref{sec:Mitarbeiter} zu entnehmen.

\item \textbf{Hard-/ und Software}

Jedes unserer Mitglieder ist im Besitz, oder hat Zugriff, auf Computer, die folgenden Anforderungen und Verfügbarkeiten gerecht werden müssen:

\begin{itemize}
\item zum Anfertigen der Dokumente wird ein Textsatzprogramm benötigt (\LaTeX wird bevorzugt).
\item für die Entwicklung der Software müssen Java-Runtime, ein Texteditor und eine Entwicklungsumgebung mit Android-SDK installiert sein.
\item Git wird zum gleichzeitigen Bearbeiten der Dokumente und zum Datenaustausch der Entwickler benötigt.
\end{itemize}

\item \textbf{Räume}

Das Team wird sich während der gesamten Projektlaufzeit regelmäßig in der e0 im MZH treffen. Weitere spezielle Räumlichkeiten werden nicht benötigt, da wir den Kontakt regelmäßig via Skype oder E-Mail gewährleisten.

\end{itemize}

\subsubsection{Budget}

Ein Budget für dieses Projekt in Form von Geld entfällt, da die Software im Rahmen des Moduls Software Projekt 2 entwickelt wird. Wenn wir über 16 Wochen (vom 24.04.2014 bis zum 10.08.2014) an dem Projekt mit 6 Studenten 17 Stunden pro Woche arbeiten, ergibt sich eine Gesamtsumme von 1632 Entwicklerstunden (16 x 6 x 17 = 1632).

Wir entnehmen einer Studie von Gulp \footnote{\url{http://www.gulp.de/presse/pressemitteilungen/marktstudie-freiberufliche-software-entwickler-sind-gefragt.html}} das zwei Drittel der Software-Entwickler zwischen 60 und 80 Euro fordern. Da wir alle Studenten sind und somit noch in der Ausbildung, setzen wir den Stundenlohn für jeden Entwickler bei 40 Euro an. Somit würden sich für den Arbeitsaufwand der Entwicklerstunden Kosten von insgesamt 65.280 Euro ergeben.

\subsubsection{Kontaktdaten des Kunden}

{\em Auftraggeber:\\
	Jacqueline Sprindt\\
	Rektorat der Universität Bremen\\
	(vertreten durch die Pressestelle)\\
}

{\em Übergeordnete Organisation:\\ 
	Dr. Karsten Hölscher\\
	Büro: ECO5 (TAB) 2.56\\
    Telefon: +49 (421) 218 64475 \\
    Fax: +49 (421) 218 4322 \\
	E-Mail: hoelsch@uni-bremen.de 
}

\newpage

\subsubsection{Mitarbeiter}\label{sec:Mitarbeiter}

In der Folgenden Tabelle \ref{tableMitarbeiter} stehen die Kontaktdaten aller am Projekt Beteiligten. Ihr sind in Folge Nachname, Name, E-Mail und ein Foto des jeweiligen Teammitglieds zu entnehmen.

\begin{table}[htbp]
\caption{Mitarbeiter}
\label{tableMitarbeiter}
\begin{tabular}{|c|c|c|}
\hline 
\textbf{Name} & \textbf{Email} & \textbf{Foto}\\ \hline \hline
Wiechers, Tim & tim3@tzi.de & \includegraphicstotab[scale=0.09]{} \\ \hline
Hollatz, Patrick & phollatz@tzi.de & \includegraphicstotab[scale=0.045]{}\\ \hline
Dellert, Tobias & tode@tzi.de & \includegraphicstotab[scale=0.6, angle=90]{Tobias.jpg}\\\hline
Ellhoff, Tim & tellhoff@tzi.de & \includegraphicstotab[scale=0.1]{Tim.JPG}\\ \hline
Pupat, Daniel & dpupat@tzi.de & \includegraphicstotab[scale=0.45]{daniel.jpg} \\ \hline
Miloevich, Olga & halfelv@uni-bremen.de & \includegraphicstotab[scale=0.45]{olga.jpg} \\ \hline
\end{tabular}
\end{table}

\newpage

\subsection{Auszuliefernde Produkte\\}

Die Tabelle \ref{reqProd} listet alle auszuliefernde Produkte auf die während des gesamten Projektes anfallen werden.

\begin{table}[htbp]
\caption{Auszuliefernde Produkte}
\label{reqProd}
\centering
\begin{tabular}{|p{4cm}|p{8cm}|p{2cm}|}
\hline Datum & Beschreibung & Anzahl\\ \hline
\hline 04.05.2014 & Initialer Projektplan (dieses Dokument) & 1\\
\hline 01.06.2014 & Anforderungsspezifikation & 1\\
\hline 01.06.2014 & GUI-Prototyp & 1\\
\hline 01.06.2014 & Angebot & 1\\
\hline 06.07.2014 & Architekturbeschreibung & 1\\
\hline 06.07.2014 & Schnittstellenbeschreibung & 1\\
\hline 06.07.2014 & Testplan & 1\\
\hline 10.08.2014 & Vollständige Abgabe & 1\\
\hline 
\end{tabular}
\end{table}

\subsection{Evolution des Plans}

Der Projektplan wird über die gesamte Dauer der Entwicklung durch das Team aktualisiert. Die erste absehbare Aktualisierung wird nach dem Kundengespräch am 08.05.2014 durchgeführt. Weitere absehbare Aktualisierungen des Projektplans sind nach den jeweiligen Hauptabgaben vom Entwicklerteam durchzuführen. \\
Aufgrund der stetigen Entwicklung des Systems sind weiter Aktualisierungen abzusehen. Vor allem im Abschnitt Arbeitspakete \ref{aps} des Projektplans. Alle Änderungen werden von dem jeweiligen Phasenleiter, welche dem Abschnitt \ref{sec:verantwortlichkeiten} zu entnehmen sind, überwacht und Korrektheit geprüft. Als unvorhergesehen Aktualisierungen wären z.B. das Austreten eines Mitglieds aus der Gruppe zu nennen, da dies die meiste Umstrukturierung mit sich zieht. Die Arbeitspakete müssten in dem Fall neu auf die restlichen Teammitglieder aufgeteilt werden, was wiederum der jeweilige Phasenleiter übernimmt.
