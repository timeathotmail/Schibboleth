\section{Technische Prozesse}
\textit{Bearbeitet von: Sylvia Kamche Tague}\\
\subsection{Methoden, Werkzeuge und Techniken}
\subsubsection{Entwicklungsplattform}
Unser Produkt wird mit Hilfe der IDE Eclipse (letzte oder vorletzte Version) entwickelt. In Eclipse wird auch Maven integriert. Wir werden für dieses Projekt GlassFish 3.1 als Application-Server benutzen.  

\subsubsection{Entwicklungsmethode}

Zuerst entwickeln wir die Schnittstellen und implementieren die dazugehörigen Blackboxtests, erst danach realisieren wir die eigentliche Funktionalität. Anschließend testen wir den fertigen Code mit Whiteboxtests.

\subsubsection{Programmiersprache und Bibliotheken}
Die genutzten Programmiersprachen werden Java und Javascript sein, zudem benutzen wir HTML, CSS und XML und sowie SQL-Datenbank.

\subsection{Dokumentationsplan}
Klassen, Methoden und Variablen werden in der Phase der Implementierung dokumentiert.

\subsubsection{Codingstyle}
% Wir wollen als Codingstyle das schreiben Syntaxe von Eclipse benutzen. 
% Variablennamen sind darüber hinaus in Englisch festzulegen und wenn möglich mit einem aussagekräftigen Kürzel mit Unterstrich als Trennzeichen zu versehen
%  (z.B. Client\_Network, Server\_Network, Client\_GUI)
Wir verwenden die Formatierungsfunktion von Eclipse um einheitlich strukturierten Code zu schaffen.
Klassen-, Methoden- und Variablen sollten einen aussagekräftigen, englischen Namen erhalten, wobei gegebenenfalls ein Unterstrich als Trennzeichen genommen wird.


\subsubsection{Kommentarsprache}
Die Kommentare werden in Englisch geschrieben, um Probleme mit Sonderzeichen zu vermeiden und das Verständnis des Sourcecodes bei der Wartung auch Nicht-Deutschsprachigen zu erleichtern bzw. zu ermöglichen.

\subsubsection{JavaDoc}
 Alle JavaDoc Kommentaren werden auf Englisch verfasst, und dienen der Übersicht und dem schnelleren Erfassen der Zusammenhänge im Projekt.

\subsubsection{Begleitende Dokumentation}
% Begleitend werden eine Installationsanleitung für die App oder den mobilen Zugang zur Webseite, sowie ein Benutzerhandbuch zur Bedienung des Programms geschrieben.
Begleitend wird ein Benutzerhandbuch angefertigt, das Installation und Bedienung aller Komponenten des Produkts erklärt.

\subsection{Unterstützende Projektfunktionen}
  
Die Dateien werden im Git-Repository gespeichert. Für die Qualitätssicherung ist der jeweilige Phasenleiter zuständig. Dieser wird jede Abgabe auf Vollständigkeit und Inhalt prüfen und gegebenenfalls überarbeiten.