\textit{Bearbeitet von Sylvia Kamche Togue, Olga Miloevich}
		
\begin{itemize}
\item Android: Freies Betriebssystem für mobile Geräte.
\item Application: ggf Programm für die Nutzung auf mobilen Geräten
\item Client: bezeichnet ein Computerprogramm, das auf dem Rechner des Nutzers ausgeführt wird. Sie sind in aller Regel in eine Client-/Serverkommunikation eingebunden. (\textit{Wikipedia, die freie Enzyklop"adie, Aufruf von Client-Seite 18.10.2013 14:08}).
\item Eclipse: Eine IDE, ggf hauptsächlich für Java-Programmierung benutzt.
\item Gantt-Diagramm: Nach Henry L. Gantt benanntes Diagramm für Zeit- und Ressourcenmanagement. Stellt die zeitliche Abfolge von Arbeitsschritten dar.
\item Git: Versionenverwaltug. Speichert die Dateien und sichert den Zugriff auf vergangene Versionen.
\item Java: Objektorientierte Programmiersprache.
\item JavaDoc: Dokumentationswerkzeug, das aus Quellcodekommentaren HTML-Dokumentation erstellt.
\item HTML: Die Hypertext Markup Language. Eine textbasierte Auszeichnungssprache zur Strukturierung von Inhalten wie Texten, Bildern und Hyperlinks in Dokumenten. HTML-Dokumente sind die Grundlage des World Wide Web und werden von einem Webbrowser dargestellt. (\textit{Wikipedia, die freie Enzyklop"adie, Aufruf von HTML-Seite 18.10.2013 13:46})
\item IDE: Eine integrierte Entwicklungsumgebung (Abkürzung IDE, von engl. integrated development environment) ist eine Sammlung von Anwendungsprogrammen, mit denen die Aufgaben der Softwareentwicklung (SWE) möglichst ohne Medienbrüche bearbeitet werden können.
(\textit{Wikipedia, die freie Enzyklop"adie, Aufruf von IDE-Seite 18.10.2013 13:49})
\item LaTeX: Softwarepaket, das die Benutzung des Textsatzsystems TeX mit Hilfe von Makros vereinfacht. (\textit{Wikipedia, die freie Enzyklop"adie, Aufruf von LaTeX-Seite 18.10.2013 13:56}) 
\item Makro: Ein Makro ist in der Softwareentwicklung eine unter einer bestimmten Bezeichnung (Makroname) zusammengefasste Folge von Anweisungen oder Deklarationen, um diese (anstelle der Einzelanweisungen, i. d. R. an mehreren Stellen im Programm) mit nur einem einfachen Aufruf ausführen zu können. Alle Anweisungen des Makros werden automatisch an der Programmstelle ausgeführt, an denen das Makro codiert wurde. (\textit{Wikipedia, die freie Enzyklop"adie, Aufruf von Makro-Seite 18.10.2013 13:58})
\item PDF: ein Dateiformat f"ur Dokumente.
\item Plug-in: Software, welche in andere Programme zur Erweiterung der Funktionalit"at eingebunden werden kann.
\item Server:  Rechner, der Anfragen von Client-Rechnern entgegen nimmt und verarbeitet.
\item SQL: Structured Query Language, Datenbanksprache für die Bearbeitung und Abrufen von Daten innerhalb einer Datenbank

\end{itemize}