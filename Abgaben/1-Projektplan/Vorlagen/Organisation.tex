\section{Projektorganisation}
\textit{Bearbeitet von: Sandor Herms}
\subsection{Prozessmodell}
\textit{Folgender Abschnitt w"urde vom Projektplan der Gruppe IrgendwieCool (Vorlesung Software Projekt 2, SoSe2013, S. 11) "ubernommen}
Wir arbeiten nach dem Wasserfallmodell mit den Phasen Projektplan, Anforderungen, (Architektur-)Entwurf, Schreiben erster Tests und Festlegung der Schnittstellen, Implementierung und Dokumentation sowie abschließendem Testen. Die Punkte werden strukturiert nacheinander abgearbeitet, wobei das Testen auch während der Implementierung noch eine Rolle spielt. Wenn es während einer Phase Probleme gibt, haben wir jederzeit die Möglichkeit, zum vorherigen Punkt zurückzuspringen.\\
Es muss ein Plan des weiteren Vorgehens erstellt werden, in welchem unter anderem Aufgaben und Zeiteinteilungen vorzunehmen sind.\\
Darauf folgt die Analyse des Projekts. Dabei ist es wichtig zu überprüfen, was mit den uns zur Verfügung stehenden Ressourcen umzusetzen ist.\\
Vor der Implementierung des Codes schreiben wir bereits sämtliche Black Box-Tests. So können wir beim Schreiben des Codes jederzeit überprüfen, ob er valide ist.\\
Wir müssen stets auf die begrenzten Ressourcen der mobilen Geräte achten. Neben weniger Speicher ist die Rechenleistung auch um einiges geringer, als bei einem heutigen handelsüblichen PC. Des weiteren muss auf die Stabilität der Software geachtet werden.\\
Nach der Implementierung des Codes wird die Software vor der Endabgabe nochmals ausführlich getestet, u.a. durch White Box-Tests. Nur so kann gewährleistet werden, dass ein funktionierendes und hochwertiges Produkt erstellt wurde. Für das Handbuch und allgemein alle Dokumente gelten ähnliche Maßstäbe; die Grammatik muss überprüft und eventuelle Rechtschreibfehler beseitigt werden.

\subsection{Organisationsstruktur}

Unsere Dateien werden im uns zur Verfügung gestellten Git-Repository gespeichert.\\
Die Kommunikation findet über E-Mail statt. Da eines unserer Mitglieder am Wochenende kein Internetzugang hat, muss notfalls Kontakt per Handy aufgenommen werden. \\ 
Zudem werden wir uns wöchentlich in der Cafete im MZH treffen. Der genaue Tag dafür steht noch nicht fest.\\ 
Für das Projekt haben wir einen Projektleiter, welcher für die allgemeine Leitung und Organisation des Projekts zuständig ist und einen Kontroller, welcher die Arbeit des Projektleiters überprüft bestimmt. Für die einzelnen Arbeitspakete haben wir jeweils einen Phasenleiter bestimmt. Dieser ist für die vollständige und rechtzeitige Bearbeitung sowie für die Qualitätssicherung der Abgaben zuständig. Näheres dazu ist im Abschnitt Managementprozess zu finden.

\subsection{Organisationsgrenzen und --schnittstellen}

Unsere Organisationsgrenze wird zu unserer Auftragsgeberin gezogen, die unsere übergeordnete Organisation darstellt.\\
Über folgende Daten kann mit ihr Kontakt aufgenommen werden.: \\
\\
Hui Shi  \\
\begin{tabular}{l l}
Kontakt & Technologiezentrum Informatik, Universität Bremen \\
Briefe & Postfach 330 440, D-28344 Bremen, Germany \\
Büro & Cartesium 1.053 \\
Telefon & +49 (421) 218-64260 \\
Telefax & +49 (421) 218-9864260 \\
Email & shi@tzi.de \\
\end{tabular}

\subsection{Verantwortlichkeiten}
\begin{tabular}{|l|l|}
Mitarbeiter & Rolle \\
\hline
Sandor Herms & Projektleiter \\
 & Phasenleiter Projektplan \\
\hline
Sylvia Kamche Tague & Controller \\
& Phasenleiterin Architekturplan \\ 
\hline
Olga Miloevich & Phasenleiterin Anforderungsspezifikation \\
\hline
Dario Treffenfeld-Mäder & Phasenleiter Implementierung \\
\hline
Jannes Uken & Phasenleiter Testplan \\
\hline
Nils Sören Oja & Phasenleiter Präsentation
\end{tabular}