\section{Managementprozess}

\textit{Bearbeitet von: Olga Miloevich, Sandor Herms}\\

\subsection{Managementprozess und --prioritäten}

\textit{Folgender Abschnitt wurde vom Projektplan der Gruppe Five and a half men (Vorlesung Software Projekt 1, SoSe2013, S. 7 "ubernommen}

F"ur die Organisation unseres Projektes haben wir einen Projektleiter gew"ahlt. Au"serdem, haben wir auch die Phasenleiter ausgesucht. Diese sind f"ur die rechtzeitigen und vollst"andigen Abgaben der Phasen verantwortlich. Die Kernidee ist, da"s damit die gesamte Aufgabe von dem Projektleiter erleichtert wird. \\
Wir haben yudem die Gruppe in einyelne Arbeitsgruppen eingeteilt. Daf"ur haben wir nach einem schnellen dynamischen System aus der Wirtschaft gesucht und haben das sogenannte Mehrliniensystem ausgew"ahlt. Dieses Mehrliniensystem ist im Gegensatz zur Matrixorganisation oder dem Einzelliniensystem am besten f"ur unsere Umst"ande und Zwecke geeignet. Das System passt am besten f"ur kleine Gruppen, wie unsere. Es lassen sich mehrere Personen zu einer Gruppe zuordnen und die Kommunikationen zwischen Gruppen und Projektteilnehmern spiegeln. \newline
Der Projektmanager "uberwacht den Ablauf des gesamten Projektes und kontrolliert die Arbeitsgruppen. Die Arbeitsgruppen kommunizieren mit den jeweiligen Mitgliedern ihrer Arbeitsgruppe sowie mit allen Gruppenleitern. Bei Bedarf kommunizieren die Arbeiter einer Arbeitsgruppe auch mit Arbeitern anderer Arbeitsgruppen. \\
Die Kommunikationen innerhalb der gesamten Projektgruppen wird durch regelm"a"sige Meetings erforderlich. Bei Konflikten oder Unstimmigkeiten ist der Projektleiter ein Ansprechpartner und auch Entscheidungstr"ager.\\
Unsere Priorit"ateten liegen bei diesem Projekt haupts"achlich bei der Erf"ullung der Mindestanforderunden in dem gegebenen Zeitraum. Dabei haben wir auch einen jeweiligen Puffer eingeplant. Werden die Mindestanforderunden fr"uhzeitig erf"ullt, dann wird dieser Puffer dazu benutzt, einige Zusatzfunktionen in unsere Software zu integrieren. \\
Budget spielt in unserem Projekt keine Rolle, da wir ggf ohne welchen arbeiten. 
Es folgt unser Mehrliniendiagramm. Die farbigen Pfeile zeigen auf die einzelnen Gruppenmitglieder.\\
\begin{figure}[H]
\center{\includegraphics[scale=0.575]{MPLD.png}}
\caption{Mehrliniendiagramm}
\label{Bild:image}
\end{figure}

\subsection{Annahmen, Abhängigkeiten und Einschränkungen}

\textbf{Annahmen}
\begin{itemize}
 \item Jeder Mitarbeiter besitzt Kenntnisse in Java, Latex und SQL.
 \item Die Grundfunktionalität kann bis zum 23.02.2014 verwirklicht werden.
\end{itemize}

\textbf{Abhängigkeiten}
\begin{itemize}
 \item Es gibt festgelegte Deadlines.
 \item Der jetzige Zeitplan wird sich z.B. bei Personalausfall ändern.
\end{itemize}

\textbf{Einschränkungen}
\begin{itemize}
 \item Wir haben vorgegebene Technologien und Sprachen zu verwenden (Java, SQL etc.) 
 \item Der Zeitplan ist eingeschränkt, für jede Abgabe ist ein festes Datum gesetzt.

\end{itemize}

\subsection{Risikomanagement}\label{riskmanagement}
 