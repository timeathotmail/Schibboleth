\section{Sonstige Elemente}

\textit{Bearbeitet von: Olga Miloevich}\\

\subsection{Pl"ane f"ur die Konvertierung von Daten}
Es wird geplannt, keine Daten zu benutzen, die konvertiert werden m"ussen.

\subsection{Managementpl"ane f"ur Unterauftragsnehmer}
Es gibt keine Unterauftragsunternehmer, alle Arbeiten werden von der Gruppe selbst ausgef"uhtr, insofern ist es auch unn"otig, eine dahingehende Schnittstelle zu definieren.

\subsection{Ausbildungspl"ane}
In 3.5 ist bereits darauf hingewiesen, was f"ur F"ahigkeiten unsere Mitarbeiter haben. Wir gehen davon aus, dass folgende F"ahigkeiten noch verbessert werden m"ussen:
Name  und Zuverbessernde Kompetenzen
name    was lernen

\subsection{Raumpl"ane}
Wir haben geplannt, uns regelm"a"sig ein Mal pro Woche zu einem Gruppenmeeting zu treffen. Als Standarttreffpunkt wird uns die Stuga-Cafete in MZH E1 dienen. Treffzeiten werden erst in n"achste Woche festgelegt, wenn alle Tutoriumtermine bekannt gegeben werden. Auf diesen Meetings werden Treffen zur inhaltlichen Arbeiten festgelegt. 

\subsection{Installationspl"ane}

Jedes Gruppenmitglied muss f"ur erfolgreiches Arbeiten folgende Software installiert haben:
\begin{enumerate}
 \item Java 7 (Programmiersprache)
 \item Eclipse (IDE)
 \item Android SDK (SDK zur Android Entwicklung)
 \item Maven (Projekt-build-tool)
 \item Glassfish 
 \item MS Visio (Software zur Erstellung der UML-Diagramme)
 \item Gantt (Projektmanagementsoftware)
\end{enumerate}
Au"serdem, mu"s jedes Gruppenmitglied die Software zur Versionenverwaltung (SVN-Client) und Bearbeitung der Dokumentation (LaTeX-Editor) installiert haben. Die Client-Versionenwahl und Latex-Editor stehen f"ur jedes Gruppenmitglied frei und h"angen von seinem Betriebsystem ab.

\subsection{Pl"ane f"ur die "Ubergabe des Systems}
Das Endprodukt wird im Form einer finalen Pr"asentation dem Kunden und dem Tutor vorgestellt.
Die Software wird in entsprechenden Dateien (*.jar/ *.apk) an den Kunden "ubergeben. Die Software wird auf CD-Disk geschrieben. Au"serdem, "ubergeben wir zu der Software geh"orende Dokumentationen, wie Projektplan, Anforderungsspezifikation, GUI-Prototyp, Test-, Architektur- und Implementierungsplan, Lasten und Pflichtenhefte und User Manual.


