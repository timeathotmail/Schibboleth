\textit{Bearbeitet von: Olga Miloevich}\\
\textit{Folgender Abschnitt wurde aus dem Projektplan der Gruppe Five and a half men (Vorlesung Software Projekt 1, Sose 2013, S. 12-14) "ubernomnmen}
Um m"ogliche Risiken zu identifizieren und zu beschreiben, haben wir hier eine sogenannte FMEA verwendet. Dies ist als ``Fehlerm"oglichkeits- und Einflussanalyse'' bekannt und wird in "ahnlicher Form auch bei Siemens oder Toyota durchgef"uhrt. Hier geht es um die konkreten Risiken, welche die Entwicklergruppe zu ber"ucksichtigen hat. Die FMEA-Tabelle l"asst die eintretende Risiken in Zahlen analysieren, so wie auch deren Eintrittswahrscheinlichkeit, Ausma"sen von Folgen und Ursachen.\\
Die folgende FMEA-Tabelle zeigt die m"oglichen Risiken auf, bewertet ihre Eintrittswahrscheinlichkeit, Gewichtung und Entdeckung. Die Gewichtung sagt, wie schwer das Risiko ist und die Entdeckung wie offensichtlich das Risiko zu entdecken ist. Die Tabelle l"asst auch ferner sagen, welche Person f"ur welchen Bereich die Verantwortung "ubernommen hat. Zuletzt sind auch m"ogliche Ma"snahmen f"ur Risikosmilderung genannt. \\
Folgende Informationen sind wichtig, zu beachten:\\
\begin{itemize}
 \item Die Zahlen in den Kategorie-Spalten ``Auftreten'', ``Bedeutung'' und ``Entdeckung'' repr\"{a}sentieren eine Eingliederung des Risikos entsprechend der genannten Kategorie auf einer Skala von 1 bis 10, wobei eine niedrigere Zahl positiv, eine h\"{o}here Zahl negativ zu verstehen ist.
 \item Die RPZ (die \textit{Risiko-Priorit\"{a}tszahl}) ist das Produkt aus den drei vorhergehenden Kategorien Auftreten, Bedeutung und Entdeckung.\newline
 Eine RPZ unter 100 ist als unkritisch einzustufen, eine RPZ \"{u}ber 700 jedoch als sehr kritisch. Hier ist es sehr wahrscheinlich, dass bei Eintritt dieses Fehlers oder Risikos das Projekt scheitern wird.
\end{itemize}

