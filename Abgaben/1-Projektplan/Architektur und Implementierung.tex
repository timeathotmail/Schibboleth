\subsubsection{Architekturbeschreibung, Schnittstellenbeschreibung, Testplan und Blackbox-Tests}

\begin{tabular}{|p{7.43cm}|p{7.43cm}|}
\hline
\textbf{AP Titel: }Architekturbeschreibung, Schnittstellenbeschreibung, Testplan und Blackbox-Tests & \textbf{AP Nummer: }3\\ 
\hline
\textbf{Dauer: }02.06.14 - 30.06.14& \textbf{Aufwand: }150 Std.\\
\hline
\end{tabular}
\begin{tabular}{|p{15.3cm}|}
\hline
\textbf{AP-Teilnehmer: }alle\\
\hline
\textbf{Beschreibung: }Hier werden zur besseren Implementierbarkeit die zusammengestellten Konzepte der Anforderungsspezifikation in technischer Form umgestaltet und denkbare Probleme und Schwierigkeiten technischer und personeller Art entsprechende Lösungsstrategien zugeordnet.  \\
\hline
\textbf{Voraussetzung: }Anforderungsspezifikation\\
\hline 
\textbf{Ziele: }Es soll insgesamt eine solide und anwendbare Architektur, was Gruppenmanagement und Implementierungsmodelle angeht,  für unser Projekt entstehen.\\
\hline 
\end{tabular}
\begin{verbatim}

\end{verbatim}

\begin{tabular}{|p{7.43cm}|p{7.43cm}|}
\hline
\textbf{AP Titel: }Globale Analyse & \textbf{AP Nummer: }3.1\\ 
\hline
\textbf{Dauer: }04.06.14 - 08.06.14& \textbf{Aufwand: }24 Std.\\
\hline
\end{tabular}
\begin{tabular}{|p{15.3cm}|}
\hline
\textbf{AP-Teilnehmer: }Olga Miloevich, Daniel Pupat, Tobias Dellert\\
\hline
\textbf{Beschreibung: }siehe Unterpunkte\\
\hline
\textbf{Voraussetzung: }Systemattribute der AS\\
\hline 
\textbf{Ziele: }Eine umfassende Analyse denkbarer Einflussfaktoren und ein ausreichender Vorrat an entsprechenden Lösungsstrategien\\
\hline 
\end{tabular}
\begin{verbatim}

\end{verbatim}

\begin{tabular}{|p{7.43cm}|p{7.43cm}|}
\hline
\textbf{AP Titel: }Einflussfaktoren & \textbf{AP Nummer: }3.1.1\\ 
\hline
\textbf{Dauer: }04.06.14 - 05.06.14& \textbf{Aufwand: }6 Std.\\
\hline
\end{tabular}
\begin{tabular}{|p{15.3cm}|}
\hline
\textbf{AP-Teilnehmer: }Olga Miloevich, Tobias Dellert, Daniel Pupat\\
\hline
\textbf{Beschreibung: }Hier werden die gegebenen Umstände in technischer und personeller Hinsicht auf Abhängigkeiten untersucht, um so mögliche Einflussfaktoren herauszuarbeiten.\\
\hline
\textbf{Voraussetzung: }Siehe Oberpaket\\
\hline 
\textbf{Ziele: }Alle Einflussfaktoren, die das Bearbeiten eines Arbeitspaketes stark einschränkt oder komplett behindern, sind herausgearbeitet.  \\
\hline 
\end{tabular}
\begin{verbatim}

\end{verbatim}

\begin{tabular}{|p{7.43cm}|p{7.43cm}|}
\hline
\textbf{AP Titel: }Problem- und Strategienkarten & \textbf{AP Nummer: }3.1.2\\ 
\hline
\textbf{Dauer: }05.06.14 - 08.06.14& \textbf{Aufwand: }18 Std.\\
\hline
\end{tabular}
\begin{tabular}{|p{15.3cm}|}
\hline
\textbf{AP-Teilnehmer: }Olga Miloevich, Tobias Dellert, Daniel Pupat\\
\hline
\textbf{Beschreibung: }Zu den Einflussfaktoren werden Lösungsstrategien entwickelt. \\
\hline
\textbf{Voraussetzung: }Siehe Oberpaket\\
\hline 
\textbf{Ziele: Zu jedem Einflussfaktor sollen mindestens zwei Lösungsstrategien bereitgestellt werden}\\
\hline 
\end{tabular}
\begin{verbatim}

\end{verbatim}

\begin{tabular}{|p{7.43cm}|p{7.43cm}|}
\hline
\textbf{AP Titel: }Konzeptionelle Sicht & \textbf{AP Nummer: }3.2\\ 
\hline
\textbf{Dauer: }04.06.14 - 04.06.14& \textbf{Aufwand: }15 Std.\\
\hline
\end{tabular}
\begin{tabular}{|p{15.3cm}|}
\hline
\textbf{AP-Teilnehmer: }alle\\
\hline
\textbf{Beschreibung: Ein UML-Diagramm, welches auf hoher Abstraktionsebene
die Struktur unseres zu implementierendes System darstellt.}\\
\hline
\textbf{Voraussetzung: }Anforderungsspezifikation\\
\hline 
\textbf{Ziele: Das Diagramm ist in nicht mehr zusammenfassbare Module bezüglich ihrer
Aufgaben aufgeteilt}\\
\hline 
\end{tabular}
\begin{verbatim}

\end{verbatim}

\begin{tabular}{|p{7.43cm}|p{7.43cm}|}
\hline
\textbf{AP Titel: }Modulsicht & \textbf{AP Nummer: }3.3\\ 
\hline
\textbf{Dauer: }08.06.14 - 20.06.14& \textbf{Aufwand: }60 Std.\\
\hline
\end{tabular}
\begin{tabular}{|p{15.3cm}|}
\hline
\textbf{AP-Teilnehmer: }Tim Wiechers, Patrick Hollatz, Tim Ellhoff\\
\hline
\textbf{Beschreibung: }Eine detailliertere statische Form der konzeptionellen Sicht, indem dessen grobe Module in Teilmodule unterteilt werden, um ein Arbeitspaket im Ausmaß von höchstens einer Arbeitswoche eines Gruppenmitglieds darzustellen. Zudem werden Schnittstellen
zwischen den Teilmodulen erstellt. \\
\hline
\textbf{Voraussetzung: }Konzeptionelle Sicht, ( Globale Analyse )\\
\hline 
\textbf{Ziele: Die Module sind detailliert genug, um als Anleitung für die Programmierung zu dienen.}\\
\hline 
\end{tabular}
\begin{verbatim}

\end{verbatim}

\begin{tabular}{|p{7.43cm}|p{7.43cm}|}
\hline
\textbf{AP Titel: }Datensicht & \textbf{AP Nummer: }3.4\\ 
\hline
\textbf{Dauer: }20.06.14 - 21.06.14& \textbf{Aufwand: }12 Std.\\
\hline
\end{tabular}
\begin{tabular}{|p{15.3cm}|}
\hline
\textbf{AP-Teilnehmer: }Tobias Dellert\\
\hline
\textbf{Beschreibung: Das anwendungs- und vorgangsaufzeigende Datenmodell wird bezüglich implementierungsspezifischer Details erweitert. }\\
\hline
\textbf{Voraussetzung: }Datenmodell\\
\hline 
\textbf{Ziele: Die Datensicht stellt auf praktische Weise den Übergang vom Anwendungskonzept zur tatsächlichen Implementierungsstruktur dar.}\\
\hline 
\end{tabular}
\begin{verbatim}

\end{verbatim}

\begin{tabular}{|p{7.43cm}|p{7.43cm}|}
\hline
\textbf{AP Titel: }Ausführungssicht & \textbf{AP Nummer: }3.5\\ 
\hline
\textbf{Dauer: }20.06.14 - 22.06.14& \textbf{Aufwand: }12 Std.\\
\hline
\end{tabular}
\begin{tabular}{|p{15.3cm}|}
\hline
\textbf{AP-Teilnehmer: }Daniel Pupat \\
\hline
\textbf{Beschreibung: Stellt den zeitlichen Verlauf unseres Systems während der Anwendung dar.}\\
\hline
\textbf{Voraussetzung: }Modulsicht\\
\hline 
\textbf{Ziele: Die Ausführungssicht spezifiziert den Implementierungsvorgang, indem es deutlich einzuhaltende Reihenfolgen miteinander kommunizierender Module aufzeigt.}\\
\hline 
\end{tabular}
\begin{verbatim}

\end{verbatim}

\begin{tabular}{|p{7.43cm}|p{7.43cm}|}
\hline
\textbf{AP Titel: }Zusammenhänge zw. Anwendungsfälle und Architektur & \textbf{AP Nummer: }3.6\\ 
\hline
\textbf{Dauer: }22.06.14 - 24.06.14& \textbf{Aufwand: }30 Std.\\
\hline
\end{tabular}
\begin{tabular}{|p{15.3cm}|}
\hline
\textbf{AP-Teilnehmer: }Tobias Dellert, Olga Miloevich, Tim Wiechers\\
\hline
\textbf{Beschreibung: } Die Anwendungsfälle werden im bisherigen System in Form von Sequenzdiagrammen dargestellt, um die Anwendbarkeit unseres Systems an unserem Projekt
zusätzlich zu sichern.\\
\hline
\textbf{Voraussetzung: }Detaillierte Anwendungsfälle, Modulsicht, ( Ausführungssicht )\\
\hline 
\textbf{Ziele: Das Funktionieren des Zusammenspiels aller Module in mindestens einem Anwendungsfall muss aufgezeigt sein. Es gibt keine Fehler in der Modulstruktur.}\\
\hline 
\end{tabular}
\begin{verbatim}

\end{verbatim}

\begin{tabular}{|p{7.43cm}|p{7.43cm}|}
\hline
\textbf{AP Titel: }Evolution & \textbf{AP Nummer: }3.7\\ 
\hline
\textbf{Dauer: }26.06.14 - 27.06.14& \textbf{Aufwand: }16 Std.\\
\hline
\end{tabular}
\begin{tabular}{|p{15.3cm}|}
\hline
\textbf{AP-Teilnehmer: }Patrick Hollatz, Tim Ellhoff\\
\hline
\textbf{Beschreibung: Hier werden Änderungen in unserem System, bei sich ändernden Rahmenbedingungen und Anforderungen aufgezeigt.}\\
\hline
\textbf{Voraussetzung: }Modulsicht\\
\hline 
\textbf{Ziele: Die genannten Punkte aus dem Abschnitt Ausblick der Anforderungsspezifikation 
werden mit unserem bisherigen System konfrontiert. Am Ende soll sich das System als sehr gut Wart- und Änderbar herausstellen.}\\
\hline 
\end{tabular}
\begin{verbatim}

\end{verbatim}

\begin{tabular}{|p{7.43cm}|p{7.43cm}|}
\hline
\textbf{AP Titel: }Einführung: Zweck/Status/Definitionen/Referenzen/Übersicht & \textbf{AP Nummer: }3.8\\ 
\hline
\textbf{Dauer: }28.06.14 - 30.06.14& \textbf{Aufwand: }6 Std.\\
\hline
\end{tabular}
\begin{tabular}{|p{15.3cm}|}
\hline
\textbf{AP-Teilnehmer: }Tobias Dellert, Tim Ellhoff\\
\hline
\textbf{Beschreibung: Eine Zusammenfassung des Zwecks der Architekturbeschreibung, den Status unseres Projekts und kleinerer Unterstützungen beim Zurechtfinden in diesem Dokument.}\\
\hline
\textbf{Voraussetzung: }Alle dem Architekturbeschreibungsdokument zugehörigen Pakete\\
\hline 
\textbf{Ziele: Für eine nicht in das Projekt eingeweihte Person soll es möglich sein, unsere Vorhaben mit den einzelnen Arbeitsschritten nachzuvollziehen.}\\
\hline 
\end{tabular}
\begin{verbatim}

\end{verbatim}

\begin{tabular}{|p{7.43cm}|p{7.43cm}|}
\hline
\textbf{AP Titel: }Testplan & \textbf{AP Nummer: }3.10\\ 
\hline
\textbf{Dauer: }24.06.14 - 28.06.14& \textbf{Aufwand: }24 Std.\\
\hline
\end{tabular}
\begin{tabular}{|p{15.3cm}|}
\hline
\textbf{AP-Teilnehmer: }Olga Miloevich, Tobias Dellert, Patrick Hollatz\\
\hline
\textbf{Beschreibung: } Erstellung eines Testplans zur besseren Verifizierung der Funktionalitäten unseres Systems. \\
\hline
\textbf{Voraussetzung: }Modulsicht, ( Ausführungssicht )\\
\hline 
\textbf{Ziele: } Die wichtigsten Funktionen des Systems und typische Fehlerquellen sind im Testplan enthalten. \\
\hline 
\end{tabular}
\begin{verbatim}

\end{verbatim}

\begin{tabular}{|p{7.43cm}|p{7.43cm}|}
\hline
\textbf{AP Titel: }Blackbox-Tests & \textbf{AP Nummer: }3.11\\ 
\hline
\textbf{Dauer: }27.06.14 - 02.07.14& \textbf{Aufwand: }30 Std.\\
\hline
\end{tabular}
\begin{tabular}{|p{15.3cm}|}
\hline
\textbf{AP-Teilnehmer: }Olga Miloevich, Daniel Pupat, Tobias Dellert\\
\hline
\textbf{Beschreibung: Die Erfüllung der spezifizierten Anforderungen unseres Systems, ohne
Beachtung der Implementierungsspezifischen Details einzelner Module.}\\
\hline
\textbf{Voraussetzung: }( Testplan ), Modulsicht\\
\hline 
\textbf{Ziele: Bei allen für einen Anwendungsvorgang verwendeten Schnittstellen stehen BlackBox-Tests bereit.}\\
\hline 
\end{tabular}
\begin{verbatim}

\end{verbatim}

\ldots

\subsubsection{Implementierung}

Die während der Implementierungsphase zu bewältigenden Arbeitspakete sind zu diesem Zeitpunkt noch nicht ersichtlich, bzw. dessen Zuordnungen von Bearbeitern, weswegen hier auf Paket- und Arbeitseinteilung weitestgehend verzichtet wird. Was jedoch bereits feststehen soll, ist das regelmäßige Treffen als Gruppe. 

\begin{tabular}{|p{7.43cm}|p{7.43cm}|}
\hline
\textbf{AP Titel: }Implementierungsvorgang & \textbf{AP Nummer: }4\\ 
\hline
\textbf{Dauer: }09.07.14 - 08.08.14& \textbf{Aufwand: } 516 Std.\\
\hline
\end{tabular}
\begin{tabular}{|p{15.3cm}|}
\hline
\textbf{AP-Teilnehmer: }alle\\
\hline
\textbf{Beschreibung: }Implementierung unserer Architektur\\
\hline
\textbf{Voraussetzung: }Architekturbeschreibung, Schnittstellenbeschreibung, Testplan und Blackbox-Tests \\
\hline 
\textbf{Ziele: }Fertig Implementiert\\
\hline 
\end{tabular}


\begin{tabular}{|p{7.43cm}|p{7.43cm}|}
\hline
\textbf{AP Titel: }Gruppentreffen & \textbf{AP Nummer: }4.1\\ 
\hline
\textbf{Dauer: }09.07.14 - 08.08.14& \textbf{Aufwand: } 96 Std.\\
\hline
\end{tabular}
\begin{tabular}{|p{15.3cm}|}
\hline
\textbf{AP-Teilnehmer: }alle\\
\hline
\textbf{Beschreibung: }Qualitätssichernde Maßnahme zur Gruppenkommunikation\\
\hline
\textbf{Voraussetzung: } \\
\hline 
\textbf{Ziele: Jedes Gruppenmitglied hat die Gelegenheit Fragen und Unklarheiten umfassend in der Gruppe zu klären.}\\
\hline 
\end{tabular}

\begin{tabular}{|p{7.43cm}|p{7.43cm}|}
\hline
\textbf{AP Titel: }Implementierung & \textbf{AP Nummer: }4.2\\ 
\hline
\textbf{Dauer: }09.07.14 - 09.08.14& \textbf{Aufwand: } 420 Std.\\
\hline
\end{tabular}
\begin{tabular}{|p{15.3cm}|}
\hline
\textbf{AP-Teilnehmer: }alle\\
\hline
\textbf{Beschreibung: }Implementieren des Projekts\\
\hline
\textbf{Voraussetzung: }Architekturbeschreibung, Schnittstellenbeschreibung, Testplan und Blackbox-Tests \\
\hline 
\textbf{Ziele: }Die Anforderungen des Kunden wurden erfüllt.\\
\hline 
\end{tabular}

\begin{tabular}{|p{7.43cm}|p{7.43cm}|}
\hline
\textbf{AP Titel: }Whitboxtests & \textbf{AP Nummer: }4.3\\ 
\hline
\textbf{Dauer: }29.07.14 - 08.08.14& \textbf{Aufwand: } 120 Std.\\
\hline
\end{tabular}
\begin{tabular}{|p{15.3cm}|}
\hline
\textbf{AP-Teilnehmer: }Alle\\
\hline
\textbf{Beschreibung: }Hier wird nicht nur das Ergebnis einer Anfrage auf Übereinstimmung der spezifizierten Anforderungen betrachtet, sondern auch die Vorgänge 
in den einzelnen Modulen.\\
\hline
\textbf{Voraussetzung: }Architekturbeschreibung, Schnittstellenbeschreibung, Testplan und Blackbox-Tests \\
\hline 
\textbf{Ziele: }Die Funktionalität der Modulstruktur ist auch bei wiederholten Gebrauch gegeben.\\
\hline 
\end{tabular}

\begin{tabular}{|p{7.43cm}|p{7.43cm}|}
\hline
\textbf{AP Titel: }Installationshandbuch anlegen & \textbf{AP Nummer: }4.4\\ 
\hline
\textbf{Dauer: }07.08.14 - 08.08.14& \textbf{Aufwand: } 20 Std.\\
\hline
\end{tabular}
\begin{tabular}{|p{15.3cm}|}
\hline
\textbf{AP-Teilnehmer: }Tim Ellhoff, Daniel Pupat\\
\hline
\textbf{Beschreibung: }Erstellung eines umfassenden Handbuches zur Installation unseres Systems.\\
\hline
\textbf{Voraussetzung: }Architekturbeschreibung, Schnittstellenbeschreibung, Testplan und Blackbox-Tests \\
\hline 
\textbf{Ziele: }Auch nicht-technisch versierte Leute können den Anweisungen folgen und so das System erfolgreich installieren.\\
\hline 
\end{tabular}


