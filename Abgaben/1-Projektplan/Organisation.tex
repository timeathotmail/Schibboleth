\section{Projektorganisation}
\textit{Bearbeitet von: Daniel Pupat}
\subsection{Prozessmodell}
\label{sec:prozessmodell}

Das von uns verwendete Wasserfallmodell gliedert die einzelnen Phasen sequentiell aufeinanderfolgend:\\
\begin{itemize}
\item Anforderungsspezifikation
\item Architekturbeschreibung (Entwurf)
\item Implementierung
\item Test
\item Dokumentation
\item Auslieferung
\end{itemize}
In unserem Fall werden sich die Phasen Implementierung, Test und Dokumentation überschneiden um eine effiziente Arbeitsweise zu gewährleisten.

\subsection{Organisationsstruktur}

Unsere Dateien werden im uns zur Verfügung gestellten Git-Repository gespeichert.\\
Die Kommunikation findet über E-Mail und über Skype statt. Da eines unserer Mitglieder am Wochenende kein Internetzugang hat, muss notfalls Kontakt per Handy aufgenommen werden. \\ 
Zudem werden wir uns wöchentlich in der e0 im MZH treffen. Der genaue Tag dafür steht noch nicht fest.\\ 
Für das Projekt haben wir einen Projektleiter, welcher für die allgemeine Leitung und Organisation des Projekts zuständig ist und einen Kontroller, welcher die Arbeit des Projektleiters überprüft. Für die einzelnen Arbeitspakete haben wir jeweils einen Phasenleiter bestimmt. Dieser ist für die vollständige und rechtzeitige Bearbeitung sowie für die Qualitätssicherung der Abgaben zuständig. Näheres dazu ist im Abschnitt Managementprozess zu finden.

\subsection{Organisationsgrenzen und --schnittstellen}

Bei dem Arbeitgeber und der übergeordneten Organisation handelt es sich um zwei verschiedene Parteien, da es sich bei dem Auftraggeber um einen echten Kunden handelt.

{\em Auftraggeber:\\
	Jacqueline Sprindt\\
	Rektorat der Universität Bremen\\
	(vertreten durch die Pressestelle)\\
}

{\em Übergeordnete Organisation:\\ 
	Dr. Karsten Hölscher\\
	Büro: ECO5 (TAB) 2.56\\
    Telefon: +49 (421) 218 64475 \\
    Fax: +49 (421) 218 4322 \\
	E-Mail: hoelsch@uni-bremen.de 
}

\end{tabular}

\subsection{Verantwortlichkeiten}
\begin{tabular}{|l|l|}
Mitarbeiter & Rolle \\
\hline
Tobias Dellert & Projektleiter \\
 & Phasenleiter Projektplan \\
\hline
Daniel Pupat & Controller \\
& Phasenleiter  Anforderungsspezifikation \\ 
\hline
Olga Miloevich & Phasenleiterin Architekturbeschreibung \\
\hline
Tim Ellhoff & Phasenleiter Implementierung \\
\hline
Patrick Hollatz & Phasenleiter Testplan \\
\hline
Tim Wiechers & Phasenleiter Präsentation
\end{tabular}

Weitere Verantwortlichkeiten, die von jeweils einem Teammitglied während des gesamten Projektzeitraumes besetzt werden müssen, sind Qualitätsmanagement und Risikomanagement.
Die Aufgaben des Qualitätsmanagers sind es, die Qualität aller Bearbeitungen der Teammitglieder sicherzustellen.
Zu den Aufgaben des Risikomanagers gehören das frühzeitige Erkennen von möglichen Problemen und diese präventiv zu vermeiden oder einzudämmen, indem er z.B. Arbeitspakete umverteilt oder andere Lösungen findet. Siehe Abschnitt \ref{riskmanagement}.