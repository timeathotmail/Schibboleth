\section{Sonstige Elemente}

\textit{Bearbeitet von: Olga Miloevich}\\

\textit{Folgender Abschnitt w"urde von Projektplan von der Gruppe YNotZoidberg (Vorlesung Software Projekt 2, WiSe2013/14, S. 6-11 "ubernommen}\\

\subsection{Pl"ane f"ur die Konvertierung von Daten}
Es wird geplannt, keine Daten zu benutzen, die konvertiert werden m"ussen.

\subsection{Managementpl"ane f"ur Unterauftragsnehmer}
Es gibt keine Unterauftragsunternehmer, alle Arbeiten werden von der Gruppe selbst ausgef"uhtr, insofern ist es auch unn"otig, eine dahingehende Schnittstelle zu definieren.

\subsection{Ausbildungspl"ane}
Wir behaupten, unsere Kentnisse sei genug, um laufendes Projekt realisieren zu k"onnen. Es wird auf jedem Fall im Laufe des Semesters aus allgemeinen Grunden jene Kentniss verbessert. Falls es n"otig wird, setzen die das brauchende Gruppenmitglieder sich intensiv mit B"ucher und Ausbildungsvideokursen zusammen ein.

\subsection{Raumpl"ane}
Wir haben geplannt, uns regelm"a"sig ein Mal pro Woche zu einem Gruppenmeeting zu treffen. Als Standarttreffpunkt wird uns der Mac-Raum in MZH E0 dienen. Treffzeiten werden erst in n"achste Woche festgelegt, wenn alle Tutoriumtermine bekannt gegeben werden. Auf diesen Meetings werden Treffen zur inhaltlichen Arbeiten festgelegt. 

\subsection{Installationspl"ane}

Jeder Gruppenmitglied mu"s f"ur erfolgreiches Arbeiten folgende Software installiert haben:
\begin{enumerate}
 \item Java 7 (Programmiersprache)
 \item Eclipse (IDE)
 \item Android SDK (SDK zur Android Entwicklung)
 \item Maven (Projekt-build-tool)
 \item Glassfish 
 \item MS Visio (Software zur Erstellung der UML-Diagramme)
 \item Gantt (Projektmanagementsoftware)
\end{enumerate}
Au"serdem, mu"s jedes Gruppenmitglied die Software zur Versionenverwaltung (Git-Client) und Bearbeitung der Dokumentation (LaTeX-Editor) installiert haben. Die Client-Versionenwahl und Latex-Editor stehen f"ur jedes Gruppenmitglied frei und h"angen von seinem Betriebsystem ab.

\subsection{Pl"ane f"ur die "Ubergabe des Systems}
Das Endprodukt wird im Form einer finalen Pr"asentation dem Kunden und dem Tutor vorgestellt.
Die Software wird in entsprechenden Dateien (*.jar/ *.apk) an den Kunden "ubergeben. Die Software wird auf CD-Disk geschrieben. Au"serdem, "ubergeben wir zu der Software geh"orende Dokumentationen, wie Projektplan, Anforderungsspezifikation, GUI-Prototyp, Test-, Architektur- und Implementierungsplan, Lasten und Pflichtenhefte und User Manual.


