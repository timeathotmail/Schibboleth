{\it
Diese Sicht beschreibt den statischen Aufbau des Systems mit Hilfe von
Modulen, Subsystemen, Schichten und Schnittstellen. 
Diese Sicht ist hierarchisch, d.h. Module werden in Teilmodule
zerlegt. Die Zerlegung endet bei Modulen, die ein klar umrissenes
Arbeitspaket für eine Person darstellen und in einer Kalenderwoche
implementiert werden können. Die Modulbeschreibung der Blätter dieser
Hierarchie muss genau genug und ausreichend sein, um das Modul 
implementieren zu können.

Die Modulsicht wird durch {UML}-Paket- und Klassendiagramme visualisiert.

Die Module werden durch ihre Schnittstellen beschrieben. 
Die Schnittstelle eines Moduls $M$ ist die Menge aller Annahmen, die
andere Module über $M$ machen dürfen, bzw.\ jene Annahmen, die $M$
über seine verwendeten Module macht (bzw. seine Umgebung, wozu auch
Speicher, Laufzeit etc.\ gehören).
Konkrete Implementierungen dieser Schnittstellen sind das Geheimnis des Moduls
und können vom Programmierer festgelegt werden. Sie sollen hier
dementsprechend nicht beschrieben werden. 

Die Diagramme der Modulsicht sollten die zur Schnittstelle gehörenden Methoden
enthalten. Die Beschreibung der einzelnen Methoden (im Sinne der Schnittstellenbeschreibung)
geschieht allerdings per Javadoc im zugehörigen Quelltext. Das bedeutet, dass Ihr
für alle Eure Module Klassen, Interfaces und Pakete erstellt und sie mit den Methoden der
Schnittstellen verseht. Natürlich noch ohne Methodenrümpfe bzw.\ mit minimalen Rümpfen.
Dieses Vorgehen vereinfacht den Schnittstellenentwurf und stellt Konsistenz sicher.

Jeder Schnittstelle liegt ein
Protokoll zugrunde. Das Protokoll beschreibt die Vor- und
Nachbedingungen der Schnittstellenelemente. Dazu gehören die erlaubten
Reihenfolgen, in denen Methoden der Schnittstelle aufgerufen werden
dürfen, sowie Annahmen über Eingabeparameter und Zusicherungen über
Ausgabeparameter. Das Protokoll von Modulen wird in der Modulsicht beschrieben.
Dort, wo es sinnvoll ist, sollte es mit Hilfe von Zustands- oder
Sequenzdiagrammen spezifiziert werden. Diese sind dann einzusetzen, wenn der
Text allein kein ausreichendes Verständnis vermittelt (insbesondere
bei komplexen oder nicht offensichtlichen Zusammenhängen).

Der Bezug zur konzeptionellen Sicht muss klar ersichtlich sein. Im
Zweifel sollte er explizit erklärt werden. Auch für diese Sicht muss
die Entstehung anhand der Strategien erläutert werden.
}