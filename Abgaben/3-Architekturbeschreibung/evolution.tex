
In diesem Abschnitt geht es um mögliche Änderungen, Anpassungen bzw. Erweiterungen, die vorgenommen werden müssten, wenn sich Anforderungen des Systems ändern. \\
Dabei ist wichtig, dass sich solche Änderungen möglichst modular realisieren lassen, ohne die bestehende Architektur komplett zu verändern, was sehr aufwändig und somit nicht wünschenswert wäre. \\
Im Folgenden werden einige wichtige mögliche neue Anforderungen bzw. Erweiterungen aufgelistet und deren jeweiligen zu implementierenden Änderungen an der Architektur beschrieben. 


\subsection*{Erweiterungsmöglichkeiten}

Da sich aus der Anforderungsspezifikation im Abschnitt ''Ausblick'' noch keine genauen absehbaren Änderungen ergeben haben, werden im Folgenden potenzielle Änderungen aufgezeigt, die näher beschrieben werden.

\subsubsection*{1. Erweiterung des GUI-Layouts}

Es wäre möglicherweise wünschenswert, wenn man als Benutzer nicht nur ein GUI-Design in der Quiz-App verwenden könnte, sondern mehrere. Dafür müsste eine Funktionalität hinzugefügt werden, um zwischen verschiedenen Benutzeroberflächen wählen zu können (z.B. verschiedene Themes oder Farbwahlen). \\
Dazu müssten entsprechende Referenzierungen von neuen GUI-Style-Änderungen mit Bilddateien stattfinden sowie für Textanpassungen die XML-Dateien im entsprechendem Paket verändert bzw. erweitert werden. 

\subsubsection*{2. Mehrsprachigkeit}

Auch wenn die Mindestanforderungen keine Mehrsprachigkeit für die Quizapp vorschreibt, wäre es ggf. wünschenswert, dass die Möglichkeit besteht, mehrere Sprachen für die Quiz-App zu unterstützen, um einen größeren bzw. internationaleren Spielerkreis anzusprechen. Insofern wäre es denkbar, dass neben Deutsch eine weitere Sprache eingebaut werden könnte. Englisch würde sich natürlich anbieten.