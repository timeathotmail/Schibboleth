\subsection{Zweck}
Dieses Dokument ist die Architekturbeschreibung der von uns zu entwickelnden Software.
Sie dient der Kommunikation zwischen allen Interessenten. Dies ist unerlässlich
für die Entwicklung des Systems, da die Entwickler der Architekturbeschreibung die
Funktionalität einzelner Komponenten entnehmen. Sie dient der Aufteilung der Arbeit
in unabhängig bearbeitbare Teile, besitzt anfangs einen hohen Abstraktionsgrad, der
von vielen verstanden werden kann und wird in den Schichten weiter unten in diesem
Dokument präziser ausgearbeitet. Die präzise Ausarbeitung der Architektur ist wichtig,
um Möglichkeiten und Probleme der Entwicklung auszuloten und präventive Strategien
und Maßnahmen zu entwickeln.\\
Die Architektur des Systems ist daher das Fundament unserer Implementierung, die
direkt aus der Architektur resultiert.


\subsection{Status}
Dies ist der erste Architekturentwurf vom 06.07.2014.

\subsection{Definitionen, Akronyme und Abkürzungen}

\subsection{Referenzen}

\begin{itemize}
\item{\url{https://elearning.uni-bremen.de/scm.php?cid=2b323f34b16a84e8dce31dcdfc0be6ad\&show\_scm=4c88951a202b2543c96de2c8a476d471}}
Die Mindestanforderungen für die Quiz-App 
\item \url{http://www.elearning.uni-bremen.de} Plattform der Universität Bremen. Zugriff auf Folien der Veranstaltung Software Projekt 1 des Sommersemesters 2013
und Übungen des Software Projekts 2 des Wintersemesters 13/14 nur eingeschränkt
möglich.
\item Vorlage dieses Dokuments - Stud.IP - 3-Architekturbeschreibung-Vorlage.tex
\item Hinweise zu diesem Dokument - Stud.IP 3-Hinweise-Abgabe-Architektur.pdf
\end{itemize}
\subsection{Übersicht über das Dokument}

Dieses Dokument basiert auf der Vorlage des IEEE P1471 2002 Standards. Der Inhalt
dieses Dokuments ist wie folgt aufgegliedert:\\
\paragraph{1. Einführung}
Die Einführung beschreibt den Nutzen dieses Dokuments. Sie erläutert Definitionen,
Akronyme und Abkürzungen und listet die benutzten Referenzen auf, sowie
eine Übersicht über dieses Dokument.
\paragraph{2. Globale Analyse}
In diesem Abschnitt werden die relevanten Einflussfaktoren aufgezeigt und bewertet,
sowie Strategien entwickelt, um Probleme bzw. interferierende Einflussfaktoren
zu behandeln und auf diese entsprechend zu reagieren.
\paragraph{3. Konzeptionelle Sicht}
Die konzeptionelle Sicht zeigt grob die einzelnen Komponenten und deren Zusammenspiel
des zu entwickelnden Systems auf. Dies geschieht auf einer hohen Abstraktionsebene
und wird im weiteren Verlauf des Dokuments und den folgenden
Sichten konkretisiert und verfeinert.
\paragraph{4. Modulsicht}
Im Abschnitt Modulsicht dieser Architekturbeschreibung geht es um eine tiefere
Ebene der Abstraktion. Hier werden die Komponenten in einzelne Pakete zerlegt
und diese wiederum in Module, welche eine Einheit bilden, die ein Entwickler in
einer Arbeitswoche implementieren kann.
\paragraph{5. Datensicht}
Die Datensicht beschreibt das zugrundeliegende Datenmodell und das Zusammenspiel
der einzelnen Daten der Datenbank. Dies wird in Form eines erklärenden
Textes und UML-Diagrammen realisiert.
\paragraph{6. Ausführungssicht}
Die Ausführungssicht zeigt im Prinzip das System in ''Aktion'', d.h. es zeigt auf,
welche Prozesse laufen, welche Module hierfür gebraucht werden und wie diese
zusammenspielen.
\paragraph{7. Zusammenhänge zwischen Anwendungsfällen und Architektur}
Hier werden die Zusammenhänge zwischen Architektur und den Anwendungsfällen
der Anforderungsspezifikation beschrieben.
\paragraph{8. Evolution}
In diesem Teil der Architekturbeschreibung wird beschrieben, welche Änderungen
vorgenommen werden müssen, wenn sich Anforderungen und oder Rahmenbedingungen
ändern. Ein besonderes Augenmerk liegt hierbei auf die in der Anforderungsspezifikation unter ''Ausblick'' genannten Punkte.