\documentclass[fontsize=12pt,paper=a4,twoside]{scrartcl}

\newcommand{\grad}{\ensuremath{^{\circ}} }
\renewcommand{\strut}{\vrule width 0pt height5mm depth2mm}

\usepackage[utf8]{inputenc}
\usepackage[final]{pdfpages}
% obere Seitenränder gestalten können
\usepackage{fancyhdr}
\usepackage{moreverb}
% Graphiken als jpg, png etc. einbinden können
\usepackage{graphicx}
\usepackage{stmaryrd}
% Floats Objekte mit [H] festsetzen
\usepackage{float}
% setzt URL's schön mit \url{http://bla.laber.com/~mypage}
\usepackage{url}
% Externe PDF's einbinden können
\usepackage{pdflscape}
% Verweise innerhalb des Dokuments schick mit " ... auf Seite ... "
% automatisch versehen. Dazu \vref{labelname} benutzen
\usepackage[ngerman]{varioref}
\usepackage[ngerman]{babel}
\usepackage{ngerman}
% Bibliographie
\usepackage{bibgerm}
% Tabellen
\usepackage{tabularx}
\usepackage{supertabular}
\usepackage[colorlinks=true, pdfstartview=FitV, linkcolor=blue,
            citecolor=blue, urlcolor=blue, hyperfigures=true,
            pdftex=true]{hyperref}
\usepackage{bookmark}

\hyphenation{Arbeits-paket}

\newboolean{langversion} %Deklaration
\setboolean{langversion}{true} %Zuweisung ist 'false' für Blockkurs
\newcommand{\highlight}[1]{\textcolor{blue}{\textbf{#1}}}
\newcommand{\nurlangversion}[0]{%
\ifthenelse{\boolean{langversion}}{\highlight{}}{\highlight{Entfällt in SWP-1}}}

% erstes Argument: SWP-2, zweites SWP-1
\newcommand{\variante}[2]{\ifthenelse{\boolean{langversion}}{#1}{#2}}

% Damit Latex nicht zu lange Zeilen produziert:
\sloppy
%Uneinheitlicher unterer Seitenrand:
%\raggedbottom

% Kein Erstzeileneinzug beim Absatzanfang
% Sieht aber nur gut aus, wenn man zwischen Absätzen viel Platz einbaut
\setlength{\parindent}{0ex}

% Abstand zwischen zwei Absätzen
\setlength{\parskip}{1ex}

% Seitenränder für Korrekturen verändern
\addtolength{\evensidemargin}{-1cm}
\addtolength{\oddsidemargin}{1cm}

\bibliographystyle{gerapali}

% Lustige Header auf den Seiten
  \pagestyle{fancy}
  \setlength{\headheight}{70.55003pt}
  \fancyhead{}
  \fancyhead[LO,RE]{Software--Projekt 1\\ SoSe 2014
  \\Anforderungsspezifikation}
  \fancyhead[LE,RO]{Seite \thepage\\\slshape \leftmark\\\slshape \rightmark}

%
% Und jetzt geht das Dokument los....
%

\begin{document}

% Lustige Header nur auf dieser Seite
  \thispagestyle{fancy}
  \fancyhead[LO,RE]{ }
  \fancyhead[LE,RO]{Universität Bremen\\FB 3 -- Informatik\\
  Dr.\ Karsten Hölscher \\TutorIn: Karsten Hölscher}
  \fancyfoot[C]{}

% Start Titelseite
  \vspace{3cm}

  \begin{minipage}[H]{\textwidth}
  \begin{center}
  \bf
  \Large
  Software--Projekt 2 2014\\
  \smallskip
  \small
  VAK 03-BA-901.02\\
  \vspace{3cm}
  \end{center}
  \end{minipage}
  \begin{minipage}[H]{\textwidth}
  \begin{center}
  \vspace{1cm}
  \bf
  \Large Architekturbeschreibung\\
  \vfill
  \includegraphics[width=0.6\textwidth]{Bilder/Logo.png}
  \end{center}
  \end{minipage}
  \vfill
  \begin{minipage}[H]{\textwidth}
  \begin{center}
  \sf
  \begin{tabular}{lrr}
 Patrick Hollatz & phollatz@tzi.de & 2596537 \\
  Tobias Dellert & tode@tzi.de & 2936941 \\
  Tim Ellhoff & tellhoff@tzi.de & 2520913\\
  Daniel Pupat & dpupat@tzi.de & 2703053 \\
  Olga Miloevich & halfelv@tzi.de  & 2586817\\  
  Tim Wiechers & tim3@tzi.de & 2925222 \\
  \end{tabular}
  \\ ~
  \vspace{2cm}
  \\
  \it Abgabe: 06.Juli 2014 --- Version 1.7\\ ~
  \end{center}
  \end{minipage}

% Ende Titelseite

% Start Leerseite

\newpage

  \thispagestyle{fancy}
  \fancyhead{}
  \fancyhead[LO,RE]{Software--Projekt \\  2014
  \\Architekturbeschreibung}
  \fancyhead[LE,RO]{Seite \thepage\\\slshape \leftmark\\~}
  \fancyfoot{}
  \renewcommand{\headrulewidth}{0.4pt}
  \tableofcontents

\newpage

  \fancyhead[LE,RO]{Seite \thepage\\\slshape \leftmark\\\slshape \rightmark}


%%%%%%%%%%%%%%%%%%%%%%%%%%%%%%%%%%%%%%%%%%%%%%%%%%%%%%%%%%%%%%%%%%%%%%%%

\paragraph{Wichtiger Hinweis:} Diese Architekturbeschreibung wurde zu einem Teil aus verschiedenen Dokumententeilen der Architekturbeschreibung unserer Gruppenmitglieder aus dem Wintersemester 2013/14 erstellt (Gruppe \textit{IT\_R3V0LUTION}). Einige Teile wurden komplett übernommen, andere überarbeitet bzw. angepasst. \\
Diese Vereinbarung haben wir in der Kick-Off-Veranstaltung für RE SWP 2014 mit dem Veranstalter Dr. Karsten Hölscher getroffen. \\
\section*{Version und Änderungsgeschichte}

\begin{tabular}{ccl}
Version & Datum & Änderungen \\
\hline
1.0 & 28.06.2014 & Einleitung \\
1.1 & 02.07.2014 & Globale Analyse \\
1.2 & 02.07.2014 & Konzeptionelle Sicht \\
1.3 & 03.07.2014 & Modulsicht \\
1.4 & 03.07.2014 & Datensicht \\
1.5 & 04.07.2014 & Ausführungssicht \\
1.6 & 04.07.2014 & Zusammenhänge zwischen Anwendungsfällen und Architektur \\
1.7 & 05.07.2014 & Evolution \\
\end{tabular}


%%%%%%%%%%%%%%%%%%%%%%%%%%%%%%%%%%%%%%%%%%%%%%%%%%%%%%%%%%%%%%%%%%%%%%%%
\section{Einführung}

\subsection{Zweck}
Dieses Dokument ist die Architekturbeschreibung der von uns zu entwickelnden Software.
Sie dient der Kommunikation zwischen allen Interessenten. Dies ist unerlässlich
für die Entwicklung des Systems, da die Entwickler der Architekturbeschreibung die
Funktionalität einzelner Komponenten entnehmen. Sie dient der Aufteilung der Arbeit
in unabhängig bearbeitbare Teile, besitzt anfangs einen hohen Abstraktionsgrad, der
von vielen verstanden werden kann und wird in den Schichten weiter unten in diesem
Dokument präziser ausgearbeitet. Die präzise Ausarbeitung der Architektur ist wichtig,
um Möglichkeiten und Probleme der Entwicklung auszuloten und präventive Strategien
und Maßnahmen zu entwickeln.\\
Die Architektur des Systems ist daher das Fundament unserer Implementierung, die
direkt aus der Architektur resultiert.


\subsection{Status}
Dies ist der erste Architekturentwurf vom 06.07.2014.

\subsection{Definitionen, Akronyme und Abkürzungen}

\subsection{Referenzen}

\begin{itemize}
\item{\url{https://elearning.uni-bremen.de/scm.php?cid=2b323f34b16a84e8dce31dcdfc0be6ad\&show\_scm=4c88951a202b2543c96de2c8a476d471}}
Die Mindestanforderungen für die Quiz-App 
\item \url{http://www.elearning.uni-bremen.de} Plattform der Universität Bremen. Zugriff auf Folien der Veranstaltung Software Projekt 1 des Sommersemesters 2013
und Übungen des Software Projekts 2 des Wintersemesters 13/14 nur eingeschränkt
möglich.
\item Vorlage dieses Dokuments - Stud.IP - 3-Architekturbeschreibung-Vorlage.tex
\item Hinweise zu diesem Dokument - Stud.IP 3-Hinweise-Abgabe-Architektur.pdf
\end{itemize}
\subsection{Übersicht über das Dokument}

Dieses Dokument basiert auf der Vorlage des IEEE P1471 2002 Standards. Der Inhalt
dieses Dokuments ist wie folgt aufgegliedert:\\
\paragraph{1. Einführung}
Die Einführung beschreibt den Nutzen dieses Dokuments. Sie erläutert Definitionen,
Akronyme und Abkürzungen und listet die benutzten Referenzen auf, sowie
eine Übersicht über dieses Dokument.
\paragraph{2. Globale Analyse}
In diesem Abschnitt werden die relevanten Einflussfaktoren aufgezeigt und bewertet,
sowie Strategien entwickelt, um Probleme bzw. interferierende Einflussfaktoren
zu behandeln und auf diese entsprechend zu reagieren.
\paragraph{3. Konzeptionelle Sicht}
Die konzeptionelle Sicht zeigt grob die einzelnen Komponenten und deren Zusammenspiel
des zu entwickelnden Systems auf. Dies geschieht auf einer hohen Abstraktionsebene
und wird im weiteren Verlauf des Dokuments und den folgenden
Sichten konkretisiert und verfeinert.
\paragraph{4. Modulsicht}
Im Abschnitt Modulsicht dieser Architekturbeschreibung geht es um eine tiefere
Ebene der Abstraktion. Hier werden die Komponenten in einzelne Pakete zerlegt
und diese wiederum in Module, welche eine Einheit bilden, die ein Entwickler in
einer Arbeitswoche implementieren kann.
\paragraph{5. Datensicht}
Die Datensicht beschreibt das zugrundeliegende Datenmodell und das Zusammenspiel
der einzelnen Daten der Datenbank. Dies wird in Form eines erklärenden
Textes und UML-Diagrammen realisiert.
\paragraph{6. Ausführungssicht}
Die Ausführungssicht zeigt im Prinzip das System in ''Aktion'', d.h. es zeigt auf,
welche Prozesse laufen, welche Module hierfür gebraucht werden und wie diese
zusammenspielen.
\paragraph{7. Zusammenhänge zwischen Anwendungsfällen und Architektur}
Hier werden die Zusammenhänge zwischen Architektur und den Anwendungsfällen
der Anforderungsspezifikation beschrieben.
\paragraph{8. Evolution}
In diesem Teil der Architekturbeschreibung wird beschrieben, welche Änderungen
vorgenommen werden müssen, wenn sich Anforderungen und oder Rahmenbedingungen
ändern. Ein besonderes Augenmerk liegt hierbei auf die in der Anforderungsspezifikation unter ''Ausblick'' genannten Punkte.



\section{Globale Analyse}
\label{sec:globale_analyse}

{\it Hier werden Einflussfaktoren aufgezählt und bewertet sowie Strategien
zum Umgang mit interferierenden Einflussfaktoren entwickelt.}

\subsection{Einflussfaktoren}
\label{sec:einflussfaktoren}
{\it Hier sind Einflussfaktoren gefragt, die sich auf die Architektur
  beziehen. Es sind ausschließlich architekturrelevante
  Einflussfaktoren, und nicht z.B.\ solche, die lediglich einen
  Einfluss auf das Projektmanagement haben. Fragt Euch also bei jedem
  Faktor: Beeinflusst er wirklich die Architektur? Macht einen
  einfachen Test: Wie würde die Architektur aussehen, wenn ihr den
  Einflussfaktor E berücksichtigt? Wie würde sie aussehen, wenn Ihr E nicht
  berücksichtigt? Kommt in beiden Fällen dieselbe Architektur heraus,
  dann kann der Einflussfaktor nicht architekturrelevant sein.

  Es geht hier um Einflussfaktoren, die
  \begin{enumerate}
  \item sich über die Zeit ändern,
  \item viele Komponenten betreffen,
  \item schwer zu erfüllen sind oder
  \item mit denen man wenig Erfahrung hat.
  \end{enumerate}
  Die Flexibilität und Veränderlichkeit müssen ebenfalls charakterisiert werden. 
  \begin{enumerate}
  \item Flexibilität: Könnt Ihr den Faktor zum jetzigen Zeitpunkt beeinflussen?
  \item Veränderlichkeit: ändert der Faktor sich später durch äußere Einflüsse?
\end{enumerate}

  Unter Auswirkungen sollte dann beschrieben werden, {\em wie} der
  Faktor {\em was} beeinflusst. Das können sein:
  \begin{itemize}
  \item andere Faktoren
  \item Komponenten
  \item Operationsmodi
  \item Designentscheidungen (Strategien)
  \end{itemize}

  Verwendet eine eindeutige Nummerierung der Faktoren, um sie auf den
  Problemkarten einfach referenzieren zu können.  }


\subsection{Probleme und Strategien}
\label{sec:strategien}

{\it Aus einer Menge von Faktoren ergeben sich Probleme, die nun in
  Form von Problemkarten beschrieben werden. Diese resultieren
  z.B. aus
  \begin{itemize}
  \item Grenzen oder Einschränkungen durch Faktoren
  \item der Notwendigkeit, die Auswirkung eines Faktors zu begrenzen
  \item der Schwierigkeit, einen Produktfaktor zu erfüllen, oder
  \item der Notwendigkeit einer allgemeinen Lösung zu globalen
    Anforderungen.
  \end{itemize}
  Dazu entwickelt Ihr Strategien, um mit den identifizierten Problemen
  umzugehen.

  Achtet auch hier darauf, dass die Probleme und Strategien wirklich
  die Architektur betreffen und nicht etwa das Projektmanagement. Die
  Strategien stellen im Prinzip die Designentscheidungen dar. Sie
  sollten also die Erklärung für den konkreten Aufbau der
  verschiedenen Sichten liefern.}


\textit{Beschreibt möglichst mehrere Alternativen und gebt
  an, für welche Ihr Euch letztlich aus welchem Grunde entschieden
  habt. Natürlich müssen die genannten Strategien in den folgenden
  Sichten auch tatsächlich umgesetzt werden!}

\textit{Ein sehr häufiger Fehler ist es, dass SWP-Gruppen
  arbeitsteilig vorgehen: die eine Gruppe schreibt das Kapitel zur
  Analyse von Faktoren und zu den Strategien, die andere Gruppe
  beschreibt die diversen Sichten, ohne dass diese beiden Gruppen sich
  abstimmen. Natürlich besteht aber ein Zusammenhang zwischen den
  Faktoren, Strategien und Sichten. Dieser muss erkennbar sein, indem
  sich die verschiedenen Kapitel eindeutig aufeinander beziehen.}

\section{Konzeptionelle Sicht}
\label{sec:konzeptionell}

{\it Diese Sicht beschreibt das System auf einer hohen Abstraktionsebene,
d.h. mit sehr starkem Bezug zur Anwendungsdomäne und den geforderten
Produktfunktionen und -attributen. Sie legt die Grobstruktur fest,
ohne gleich in die Details von spezifischen Technologien abzugleiten. 
Sie wird in den nachfolgenden Sichten konkretisiert und verfeinert. Die
konzeptionelle Sicht wird mit {UML}-Komponentendiagrammen visualisiert.}

\section{Modulsicht}
\label{sec:modulsicht}

{\it
Diese Sicht beschreibt den statischen Aufbau des Systems mit Hilfe von
Modulen, Subsystemen, Schichten und Schnittstellen. 
Diese Sicht ist hierarchisch, d.h. Module werden in Teilmodule
zerlegt. Die Zerlegung endet bei Modulen, die ein klar umrissenes
Arbeitspaket für eine Person darstellen und in einer Kalenderwoche
implementiert werden können. Die Modulbeschreibung der Blätter dieser
Hierarchie muss genau genug und ausreichend sein, um das Modul 
implementieren zu können.

Die Modulsicht wird durch {UML}-Paket- und Klassendiagramme visualisiert.

Die Module werden durch ihre Schnittstellen beschrieben. 
Die Schnittstelle eines Moduls $M$ ist die Menge aller Annahmen, die
andere Module über $M$ machen dürfen, bzw.\ jene Annahmen, die $M$
über seine verwendeten Module macht (bzw. seine Umgebung, wozu auch
Speicher, Laufzeit etc.\ gehören).
Konkrete Implementierungen dieser Schnittstellen sind das Geheimnis des Moduls
und können vom Programmierer festgelegt werden. Sie sollen hier
dementsprechend nicht beschrieben werden. 

Die Diagramme der Modulsicht sollten die zur Schnittstelle gehörenden Methoden
enthalten. Die Beschreibung der einzelnen Methoden (im Sinne der Schnittstellenbeschreibung)
geschieht allerdings per Javadoc im zugehörigen Quelltext. Das bedeutet, dass Ihr
für alle Eure Module Klassen, Interfaces und Pakete erstellt und sie mit den Methoden der
Schnittstellen verseht. Natürlich noch ohne Methodenrümpfe bzw.\ mit minimalen Rümpfen.
Dieses Vorgehen vereinfacht den Schnittstellenentwurf und stellt Konsistenz sicher.

Jeder Schnittstelle liegt ein
Protokoll zugrunde. Das Protokoll beschreibt die Vor- und
Nachbedingungen der Schnittstellenelemente. Dazu gehören die erlaubten
Reihenfolgen, in denen Methoden der Schnittstelle aufgerufen werden
dürfen, sowie Annahmen über Eingabeparameter und Zusicherungen über
Ausgabeparameter. Das Protokoll von Modulen wird in der Modulsicht beschrieben.
Dort, wo es sinnvoll ist, sollte es mit Hilfe von Zustands- oder
Sequenzdiagrammen spezifiziert werden. Diese sind dann einzusetzen, wenn der
Text allein kein ausreichendes Verständnis vermittelt (insbesondere
bei komplexen oder nicht offensichtlichen Zusammenhängen).

Der Bezug zur konzeptionellen Sicht muss klar ersichtlich sein. Im
Zweifel sollte er explizit erklärt werden. Auch für diese Sicht muss
die Entstehung anhand der Strategien erläutert werden.
}

\section{Datensicht}
\label{sec:datensicht}

{\it Hier wird das der Anwendung zugrundeliegende Datenmodell
  beschrieben. Hierzu werden neben einem erläuternden Text auch ein
  oder mehrere {UML}-Klassendiagramme verwendet. Das hier beschriebene
  Datenmodell wird u.a. jenes der Anforderungsspezifikation enthalten,
  allerdings mit implementierungsspezifischen Änderungen und
  Erweiterungen. Siehe die gesonderten Hinweise.}

\section{Ausführungssicht}
\nurlangversion

\label{sec:ausfuehrung}

{\it
Die Ausführungssicht beschreibt das Laufzeitverhalten. Hier
werden die Laufzeitelemente aufgeführt und beschrieben, welche Module
sie zur Ausführung bringen. Ein Modul kann von mehreren
Laufzeitelementen zur Laufzeit verwendet werden. Die Ausführungssicht
beschreibt darüber hinaus, welche Laufzeitelemente spezifisch
miteinander kommunizieren. Zudem wird bei verteilten Systemen
(z.B. Client-Server-Systeme) dargestellt, welche Module von welchen
Prozessen auf welchen Rechnern ausgeführt werden.}


\section[Zusammenhänge zwischen Anwendungsfällen und Architektur]{Zusammenhänge zwischen Anwendungsfällen und Architektur\sectionmark{Zusammenhänge AF u. Architektur}}
\sectionmark{Zusammenhänge AF u. Architektur}
\label{sec:anwendungsfaelle}

{\it In diesem Abschnitt sollen Sequenzdiagramme mit Beschreibung(!)
  für \variante{zwei bis drei von Euch ausgewählte
    Anwendungsfälle}{einen von Euch ausgewählten Anwendungsfall}
  erstellt werden. Ein Sequenzdiagramm beschreibt den
  Nachrichtenverkehr zwischen allen Modulen, die an der Realisierung
  des Anwendungsfalles beteiligt sind.  \variante{Wählt die
    Anwendungsfälle so, dass nach Möglichkeit alle Module Eures
    entworfenen Systems in mindestens einem Sequenzdiagramm
    vorkommen. Falls Euch das nicht gelingt, versucht möglichst viele
    und die wichtigsten Module abzudecken.}{Dazu könnt ihr Euch einen
    Anwendungsfall heraussuchen, der möglichst viele Module der
    Architektur abdeckt. In SWP-2 werden wir mehrere Anwendungsfälle
    betrachten und eine umfangreichere Abdeckung der Architektur
    anstreben.} }

\section{Evolution}
\label{sec:evolution}

In diesem Abschnitt geht es um mögliche Änderungen, Anpassungen bzw. Erweiterungen, die vorgenommen werden müssten, wenn sich Anforderungen des Systems ändern. \\
Dabei ist wichtig, dass sich solche Änderungen möglichst modular realisieren lassen, ohne die bestehende Architektur komplett zu verändern, was sehr aufwändig und somit nicht wünschenswert wäre. \\
Im Folgenden werden einige wichtige mögliche neue Anforderungen bzw. Erweiterungen aufgelistet und deren jeweiligen zu implementierenden Änderungen an der Architektur beschrieben. 


\subsection*{Erweiterungsmöglichkeiten}

Da sich aus der Anforderungsspezifikation im Abschnitt ''Ausblick'' noch keine genauen absehbaren Änderungen ergeben haben, werden im Folgenden potenzielle Änderungen aufgezeigt, die näher beschrieben werden.

\subsubsection*{1. Erweiterung des GUI-Layouts}

Es wäre möglicherweise wünschenswert, wenn man als Benutzer nicht nur ein GUI-Design in der Quiz-App verwenden könnte, sondern mehrere. Dafür müsste eine Funktionalität hinzugefügt werden, um zwischen verschiedenen Benutzeroberflächen wählen zu können (z.B. verschiedene Themes oder Farbwahlen). \\
Dazu müssten entsprechende Referenzierungen von neuen GUI-Style-Änderungen mit Bilddateien stattfinden sowie für Textanpassungen die XML-Dateien im entsprechendem Paket verändert bzw. erweitert werden. 

\subsubsection*{2. Mehrsprachigkeit}

Auch wenn die Mindestanforderungen keine Mehrsprachigkeit für die Quizapp vorschreibt, wäre es ggf. wünschenswert, dass die Möglichkeit besteht, mehrere Sprachen für die Quiz-App zu unterstützen, um einen größeren bzw. internationaleren Spielerkreis anzusprechen. Insofern wäre es denkbar, dass neben Deutsch eine weitere Sprache eingebaut werden könnte. Englisch würde sich natürlich anbieten.


\end{document}


%%% Local Variables: 
%%% mode: latex
%%% mode: reftex
%%% mode: flyspell
%%% ispell-local-dictionary: "de_DE"
%%% TeX-master: t
%%% End: 
