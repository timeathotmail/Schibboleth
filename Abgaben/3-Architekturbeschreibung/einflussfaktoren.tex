Die Einflussfaktoren werden im Folgenden unterteilt in:

\begin{itemize}
\item{Organisatorische Faktoren}
\item{Technische Faktoren}
\item{Produktfaktoren}
\end{itemize}

\subsubsection{Organisatorische Faktoren}
\label{sec:orgfaktoren}

\begin{table}[H]
\centering
\caption{Organisatorische Faktoren}
\begin{tabular}{|l|l|} \hline
\textbf{O1} & \textbf{Time-To-Market} \\ \hline
\textbf{O2} & \textbf{Auslieferung von Produktfunktionen} \\ \hline
\textbf{O3} & \textbf{Budget} \\ \hline
\textbf{O4} & \textbf{Kenntnisse in Java, SQL und Android} \\ \hline
\textbf{O5} & \textbf{Kenntnisse in J-Unit} \\ \hline
\textbf{O6} & \textbf{Anzahl der Entwickler}\\ \hline
\end{tabular}
\end{table}

\begin{table}[H]
\caption{O1}
\begin{tabular}{|p{3cm}|p{12cm}|}\hline
\textbf{O1} & \textbf{Time-To-Market}\\ \hline \hline
Faktor & Auslieferungsdatum 10.08.2014\\ \hline
Flexibilität und Veränderlichkeit & Die Deadline kann nicht verändert werden.\\ \hline
Auswirkungen & Die Software muss zum Abgabedatum lauffähig sein.\\ \hline
\end{tabular}
\end{table}

\begin{table}[H]
\caption{O2}
\begin{tabular}{|p{3cm}|p{12cm}|}\hline
\textbf{O2} & \textbf{Auslieferung von Produktfunktionen}\\ \hline \hline
Faktor & Alle Mindestanforderungen\\ \hline
Flexibilität und Veränderlichkeit & Es müssen alle Mindestanforderungen erfüllt sein; sie sind jedoch vom Kunden oder beim Verlassen eines Gruppenmitglieds veränderbar.\\ \hline
Auswirkungen & Architektur muss alle Mindestanforderungen abdecken; es muss darauf geachtet werden, dass diese sich im Verlauf noch ändern.\\ \hline
\end{tabular}
\end{table}

\begin{table}[H]
\caption{O3}
\begin{tabular}{|p{3cm}|p{12cm}|}\hline
\textbf{O3} & \textbf{Budget} \\ \hline \hline
Faktor & Kein finanzielles Budget\\ \hline
Flexibilität und Veränderlichkeit & Es werden keine finanziellen Unterstützungen für das Produkt geben. \\ \hline
Auswirkungen & Es können keine kostenpflichtigen Dienste in Anspruch genommen werden.\\ \hline
\end{tabular}
\end{table}

\begin{table}[H]
\caption{O4}
\begin{tabular}{|p{3cm}|p{12cm}|}\hline
\textbf{O4} & \textbf{Kenntnisse in Java, SQL und Android} \\ \hline \hline
Faktor & Kenntnisse der Entwickler in Java, SQL und Android\\ \hline
Flexibilität und Veränderlichkeit & Kenntnisse sind nicht flexibel, es muss in Java programmiert werden und über Smartphone laufen. Die Server-Software muss mit der SQL-Datenbank kommunizieren. Die Kenntnisse können sich im Laufe ändern, z.B. durch neue Erfahrungen und neu erworbene Kenntnisse.\\ \hline
Auswirkungen & Bei wenig Kenntnissen muss mehr Zeit eingeplant werden, um sich diese anzueignen.\\ \hline
\end{tabular}
\end{table}

\begin{table}[H]
\caption{O5}
\begin{tabular}{|p{3cm}|p{12cm}|}\hline
\textbf{O5} & \textbf{Kenntnisse in J-Unit} \\ \hline \hline
Faktor & Kenntnisse in J-Unit Tests\\ \hline
Flexibilität und Veränderlichkeit & Da Tests mit J-Unit gefordert werden, sind diese nicht verhandelbar oder flexibel.\\ \hline
Auswirkungen & Bei unzureichenden Tests kann es später beim Programm zu Problemen kommen, da Fehler spät oder gar nicht erkannt werden.\\ \hline
\end{tabular}
\end{table}

\begin{table}[H]
\caption{O6}
\begin{tabular}{|p{3cm}|p{12cm}|}\hline
\textbf{O6} & \textbf{Anzahl der Entwickler}\\ \hline \hline
Faktor & Die Anzahl der Entwickler\\ \hline
Flexibilität und Veränderlichkeit & Es können keine neuen Gruppenmitglieder dazukommen, es können aber jederzeit Gruppenmitglieder wegfallen. \\ \hline
Auswirkungen & Wenn Gruppenmitglieder wegfallen, müssen die restlichen Mitglieder mehr Arbeit und mehr Zeit einplanen. Auch müssen Projektplan und Architektur neu angepasst werden.\\ \hline
\end{tabular}
\end{table}

%====================================================

\subsubsection{Technische Faktoren}
\label{sec:techfaktoren}

\begin{table}[H]
\centering
\caption{Technische Faktoren}
\begin{tabular}{|l|l|} \hline
\textbf{T0} &  \textbf{Hardware des Kunden}\\ \hline
\textbf{T1} & \textbf{Software funktioniert unter Windows und Linux} \\ \hline
\textbf{T2} & \textbf{Software funktioniert als App(Andriod 2.3 oder höher)}\\ \hline
\textbf{T3} & \textbf{SQL-Datenbank} \\ \hline
\textbf{T4} & \textbf{Mehrere parallele Nutzer} \\ \hline
\textbf{T5} & \textbf{Client-Server System} \\ \hline
\textbf{T6} & \textbf{Benutzerschnittstelle} \\ \hline
\textbf{T7} & \textbf{Implementierungssprache Java} \\ \hline
\textbf{T8} &  \textbf{Beschränkungsfreiheit für Fremdbibliotheken}\\ \hline
\textbf{T9} &  \textbf{Testbarkeit}\\ \hline
\end{tabular}
\end{table}

\begin{table}[H]
\caption{T0}
\begin{tabular}{|p{3cm}|p{12cm}|}\hline
\textbf{T0} & \textbf{Hardware des Kunden} \\ \hline \hline
Faktor & Die Hardware des Kunden stellt eine Beschränkungen dar.\\ \hline
Flexibilität und Veränderlichkeit & nicht flexibel. In dem entscheidenden Zeitraum werden keine Ver"anderungen stattfinden. \\ \hline
Auswirkungen & Es muss darauf geachtet werden, da"s unsere Software die Hardware nicht zu sehr belastet.\\ \hline
\end{tabular}
\end{table}

\begin{table}[H]
\caption{T1}
\begin{tabular}{|p{3cm}|p{12cm}|}\hline
\textbf{T1} & \textbf{Software funktioniert unter Windows und Linux} \\ \hline \hline
Faktor & Die Software muss auf den Betriebssystemen Windows und Linux laufen.\\ \hline
Flexibilität und Veränderlichkeit & nicht flexibel, da dies zu den Mindestanforderungen gehört. Veränderungen können jederzeit vom Kunden vorgenommen werden.  \\ \hline
Auswirkungen & Die Entwickler müssen sich mit beiden Betriebsprogrammen befassen und sichergehen, dass es auf beiden funktioniert.\\ \hline
\end{tabular}
\end{table}


\begin{table}[H]
\caption{T2}
\begin{tabular}{|p{3cm}|p{12cm}|}\hline
\textbf{T2} & \textbf{Software funktioniert als App (Andriod 2.3 oder höher)}. \\ \hline \hline
Faktor & Die Software muss als Android App auf einem Smartphone laufen.\\ \hline
Flexibilität und Veränderlichkeit & nicht flexibel, da dies zu den Mindestanforderungen gehört. Veränderungen können jederzeit vom Kunden vorgenommen werden.  \\ \hline
Auswirkungen & Die Software muss wie gefordert als App auf einem Android-Smartphone laufen.\\ \hline
\end{tabular}
\end{table}


\begin{table}[H]
\caption{T3}
\begin{tabular}{|p{3cm}|p{12cm}|}\hline
\textbf{T3} & \textbf{SQL-Datenbank} \\ \hline \hline
Faktor & Software läuft über eine relationale Datenbank.\\ \hline
Flexibilität und Veränderlichkeit & Flexibel, jedoch muss eine Datenbank mit SQL oder SQL-ähnlichen Abfragen verwendet werden.  \\ \hline
Auswirkungen & Es muss eine relationale Datenbank für die serverseitige Persistenz benutzt werden. Es muss eine Datenbank mit SQL oder SQL-ähnlichen abfragen verwendet werden.\\ \hline
\end{tabular}
\end{table}

\begin{table}[H]
\caption{T4}
\begin{tabular}{|p{3cm}|p{12cm}|}\hline
\textbf{T4} & \textbf{Mehrere parallele Nutzer} \\ \hline \hline
Faktor & Es greifen mehrere Nutzer zur gleichen Zeit auf die Software zu.\\ \hline
Flexibilität und Veränderlichkeit & Es ist uns überlassen, wie viele Nutzer zur gleichen Zeit auf das System zugreifen dürfen.  \\ \hline
Auswirkungen & Die Software muss darauf ausgelegt sein, mehrere Nutzer zur gleichen Zeit zu verwalten.\\ \hline
\end{tabular}
\end{table}

\begin{table}[H]
\caption{T5}
\begin{tabular}{|p{3cm}|p{12cm}|}\hline
\textbf{T5} & \textbf{Client-Server System} \\ \hline \hline
Faktor & Die Software arbeitet über ein Client-Server System.\\ \hline
Flexibilität und Veränderlichkeit & Da wir übers Internet auf den Server zugreifen müssen, ist es notwendig, ein Server-Client System zu verwenden. \\ \hline
Auswirkungen & Die Implementierung wird in Server und Client aufgeteilt (siehe \ref{sec:konzeptionell}). Übers Internet werden die Daten zwischen Server und Client ausgetauscht.\\ \hline
\end{tabular}
\end{table}

\begin{table}[H]
\caption{T6}
\begin{tabular}{|p{3cm}|p{12cm}|}\hline
\textbf{T6} & \textbf{Benutzerschnittstelle} \\ \hline \hline
Faktor & Es sollte eine übersichtliche und ansprechende GUI geben.\\ \hline
Flexibilität und Veränderlichkeit & Die Gestaltung der GUI ist uns überlassen. \\ \hline
Auswirkungen & Für eine benutzerfreundliche Gestaltung sind Kenntnisse in LibGDX und in HTML5 notwendig.\\ \hline
\end{tabular}
\end{table}

\begin{table}[H]
\caption{T7}
\begin{tabular}{|p{3cm}|p{12cm}|}\hline
\textbf{T7} & \textbf{Implementierungssprache Java} \\ \hline \hline
Faktor & Die Software muss in Java 6 oder höher geschrieben werden.\\ \hline
Flexibilität und Veränderlichkeit & Nicht flexibel, da dies zu den Mindestanforderungen gehört.\\ \hline
Auswirkungen & Die Software muss in Java geschrieben werden, daher müssen alle Entwickler diese Sprache beherrschen. \\ \hline
\end{tabular}
\end{table}

\begin{table}[H]
\caption{T8}
\begin{tabular}{|p{3cm}|p{12cm}|}\hline
\textbf{T8} & \textbf{Beschränkungsfreiheit für Fremdbibliotheken}\\ \hline \hline
Faktor & Fremdbibliotheken dürfen für den Einsatz in Forschung.\\
& und Lehre keine Beschränkungen aufweisen. \\ \hline
Flexibilität und Veränderlichkeit & Nicht flexibel, da dies zu den Mindestanforderungen gehört.\\ \hline
Auswirkungen & Es darf keine Software oder Bibliothek verwendet werden, die kostenpflichtig ist. \\ \hline
\end{tabular}
\end{table}


\begin{table}[H]
\caption{T9}
\begin{tabular}{|p{3cm}|p{12cm}|}\hline
\textbf{T9} & \textbf{Testbarkeit}\\ \hline \hline
Faktor & Die Software muss auf Richtigkeit getestet werden. \\ \hline
Flexibilität und Veränderlichkeit & Geringer Einfluss auf Testumfang. Es sind keine Ver"anderungen zu erwarten. \\ \hline
Auswirkungen & Es muss darauf geachtet werden, da"s die Blackbox-Tests implementierbar sind. \\ \hline
\end{tabular}
\end{table}

%====================================================

\subsubsection{Produktfaktoren}
\label{sec:produktfaktoren}

\begin{table}[H]
\centering
\caption{Produktfaktoren}
\begin{tabular}{|l|l|} \hline
\textbf{P1} & \textbf{Funktionalit"at} \\ \hline
\textbf{P2} &  \textbf{Performanz}\\ \hline
\textbf{P3} &  \textbf{Benutzerfreundlichkeit} \\ \hline
\textbf{P4} &  \textbf{Benutzerrechte} \\ \hline
\textbf{P5} &  \textbf{Fehlererkennung} \\ \hline
\textbf{P6} &  \textbf{Sichercheit} \\ \hline
\textbf{P7} &  \textbf{Erweiterbarkeit und Bedienbarkeit} \\ \hline
\end{tabular}
\end{table}

\begin{table}[H]
\caption{P1}
\begin{tabular}{|p{3cm}|p{12cm}|}\hline
\textbf{P1} & \textbf{Funktionalit"at} \\ \hline \hline
Faktor & Das Produkt muss alle Mindestanforderungen enthalten.\\ \hline
Flexibilität und Veränderlichkeit & Alle Anforderungen müssen zum Bestehen erfüllt werden. Die Anforderungen können vom Kunden oder Dozenten verändert werden oder die Anforderungen werden bei einem Austritt eines Mitglieds verringert. \\ \hline
Auswirkungen & Es müssen alle Mindestanforderungen implementiert werden. Hat allgemein gro"sen Einfluss auf die Architektur.\\ \hline
\end{tabular}
\end{table}

\begin{table}[H]
\caption{P2}
\begin{tabular}{|p{3cm}|p{12cm}|}\hline
\textbf{P2} &  \textbf{Performanz}\\ \hline \hline
Faktor & Möglichst schnelle Ausführungszeiten\\ \hline
Flexibilität und Veränderlichkeit & Flexibel, da nichts davon in den Mindestanforderungen steht.\\ \hline
Auswirkungen & Es sollte bei der Implementierung auf einen schnellen Datenaustausch zwischen Server und Client geachtet werden. \\ \hline
\end{tabular}
\end{table}

\begin{table}[H]
\caption{P3}
\begin{tabular}{|p{3cm}|p{12cm}|}\hline
\textbf{P3} &  \textbf{Benutzerfreundlichkeit} \\ \hline \hline
Faktor & Das Produkt soll so einfach wie m"oglich und so schwer wie n"otig zu bedienen sein.\\ \hline
Flexibilität und Veränderlichkeit & Es gibt keine Vorschriften, sodass wir eigenst"andig entscheiden k"onnen.\\ \hline
Auswirkungen & Hat Auswirkung auf den Bereich der Architektur, der direkt vom Benutzer verwendet wird.\\ \hline
\end{tabular}
\end{table}

\begin{table}[H]
\caption{P4}
\begin{tabular}{|p{3cm}|p{12cm}|}\hline
\textbf{P4} &  \textbf{Benutzerrechte} \\ \hline \hline
Faktor & Es gibt verschiedene Benutzer mit unterschiedlichen Rechten.\\ \hline
Flexibilität und Veränderlichkeit & Nicht flexibel, da dies vom Kunden gefordert wird.\\ \hline
Auswirkungen & Es müssen unterschiedliche Benutzer implementiert werden, die unterschiedliche Rechte haben und diese auch nicht überschreiten dürfen. \\ \hline
\end{tabular}
\end{table}

\begin{table}[H]
\caption{P5}
\begin{tabular}{|p{3cm}|p{12cm}|}\hline
\textbf{P5} &  \textbf{Fehlererkennung} \\ \hline \hline
Faktor & Fehler sollten von der Software erkannt werden und entsprechend behandelt werden.\\ \hline
Flexibilität und Veränderlichkeit & Flexibel, da dies nicht ausdrücklich vom Kunden gefordert wird.\\ \hline
Auswirkungen & Fehler müssen erkannt und durch entsprechende Exceptions korrigiert werden. Die Software sollte weiter laufen.  \\ \hline
\end{tabular}
\end{table}

\begin{table}[H]
\caption{P6}
\begin{tabular}{|p{3cm}|p{12cm}|}\hline
\textbf{P6} &  \textbf{Sicherheit} \\ \hline \hline
Faktor & Stabiler Datenaustausch, m"ogliche SQL-Injektions sollen m"oglichst abgefangen werden.\\ \hline
Flexibilität und Veränderlichkeit & Flexibel, da dies nicht ausdrücklich vom Kunden gefordert wird.\\ \hline
Auswirkungen & Die Software soll sorgf"altig auf Schwachstellen untersucht werden, dies ist zeitaufwendig. Erfordert erweiterte Kentnisse in SQL und Sicherheitsfragen.\\ \hline
\end{tabular}
\end{table}

\begin{table}[H]
\caption{P7}
\begin{tabular}{|p{3cm}|p{12cm}|}\hline
\textbf{P7} &  \textbf{Erweiterbarkeit und Bedienbarkeit} \\ \hline \hline
Faktor & Es ist w"unschenswert, dass unser Produkt sich leicht erweitern l"a"st.\\ \hline
Flexibilität und Veränderlichkeit & Flexibel, da dies nicht ausdrücklich vom Kunden gefordert wird. Man kann auf die Erweiterbarkeit verzichten, um an Abstraktion zu sparen.\\ \hline
Auswirkungen & Solange es kein zu gro"ses Hinderniss darsteht, kann beim Entwurf darauf geachtet werden, dass die Software erweiterbar bleibt.\\ \hline
\end{tabular}
\end{table}

%---------------------------------