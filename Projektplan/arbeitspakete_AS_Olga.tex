\subsubsection{Anforderungsspezifikation}

\textit{Bearbeitet von: Olga Miloevich}\\

\begin{tabular}{|p{7.43cm}|p{7.43cm}|}
\hline
\textbf{AP Titel: }2 Anforderungsspezifikation & \textbf{AP Nummer: 1} \\ 
\hline
\textbf{Dauer: 05.05.14- 21.05.14} & \textbf{Aufwand: }20 Std.\\
\hline
\end{tabular}
\begin{tabular}{|p{15.3cm}|}
\hline
\textbf{AP-Teilnehmer: } alle Mitglieder\\
\hline
\textbf{Beschreibung: }Anfertigung und Komplettierung der Anforderungsspezifikation. Details erfolgen in den jeweligen Unterpunkten\\
\hline
\textbf{Voraussetzung: }keine \\
\hline 
\textbf{Ziele: }Die Fertigstellung der Anforderungsspezifikation ist erfolgt.\\
\hline 
\end{tabular}
\begin{verbatim}

\end{verbatim}
\begin{tabular}{|p{7.43cm}|p{7.43cm}|}
\hline
\textbf{AP Titel: } 1.2.2.1.3 Kundengespräch & \textbf{AP Nummer: 2} \\ 
\hline
\textbf{Dauer: }xx.05.14& \textbf{Aufwand: }2 Std.\\
\hline
\end{tabular}
\begin{tabular}{|p{15.3cm}|}
\hline
\textbf{AP-Teilnehmer: Tim Ellhoff, Tobias Dellert\\
\hline
\textbf{Beschreibung: }Gespräch mit dem Kunden um dessen Wünsche und Erwartungen an das Produkt festzustellen.\\
\hline
\textbf{Voraussetzung: }keine\\
\hline 
\textbf{Ziele: }Klarheit über die Anforderungen herstellen.\\
\hline 
\end{tabular}
\begin{verbatim}

\end{verbatim}
\begin{tabular}{|p{7.43cm}|p{7.43cm}|}
\hline
\textbf{AP Titel: }2.1 GUI-Prototyp & \textbf{AP Nummer: }3 \\ 
\hline
\textbf{Dauer: }07.05.14 - 09.05.14 & \textbf{Aufwand: }15 Std.\\
\hline
\end{tabular}
\begin{tabular}{|p{15.3cm}|}
\hline
\textbf{AP-Teilnehmer: }Tobias Dellert, Olga Miloevich\\
\hline
\textbf{Beschreibung: }Dem Kundengespräch folgend, wird dem Kunden erstmal eine Benutzeroberfl"ache vorgestellt.\\
\hline
\textbf{Voraussetzung: }stattgefundenes Kundengespräch\\
\hline 
\textbf{Ziele: }Die Grundfunktionen m"ussen anhand des Prototyps pr"asentierbar sein.\\
\hline 
\end{tabular}
\begin{verbatim}

\end{verbatim}
\begin{tabular}{|p{7.43cm}|p{7.43cm}|}
\hline
\textbf{AP Titel: }2.1.1 Story-Board & \textbf{AP Nummer: }4 \\ 
\hline
\textbf{Dauer: }08.05.14 - 10.05.14 & \textbf{Aufwand: }12 Std.\\
\hline
\end{tabular}
\begin{tabular}{|p{15.3cm}|}
\hline
\textbf{AP-Teilnehmer: }Patrick Hollatz\\
\hline
\textbf{Beschreibung: }\\
\hline
\textbf{Voraussetzung: }keine\\
\hline 
\textbf{Ziele: }Mit dem Prototyp lassen sich die Leistungsfaktoren pr"asentieren.\\
\hline 
\end{tabular}
\begin{verbatim}

\end{verbatim}
\begin{tabular}{|p{7.43cm}|p{7.43cm}|}
\hline
\textbf{AP Titel: }2.1.2 Design & \textbf{AP Nummer: }5 \\ 
\hline
\textbf{Dauer: }08.05.14- 09.05.15 & \textbf{Aufwand: }10 Std.\\
\hline
\end{tabular}
\begin{tabular}{|p{15.3cm}|}
\hline
\textbf{AP-Teilnehmer: }Olga Miloevich\\
\hline
\textbf{Beschreibung: }\\
\hline
\textbf{Voraussetzung: }keine\\
\hline 
\textbf{Ziele: }Einzusetzendes Design.\\
\hline 
\end{tabular}
\begin{verbatim}

\end{verbatim}
\begin{tabular}{|p{7.43cm}|p{7.43cm}|}
\hline
\textbf{AP Titel: }2.1.3 Entwicklungsphase I & \textbf{AP Nummer: }6 \\ 
\hline
\textbf{Dauer: }09.05.14 - 15.05.14 & \textbf{Aufwand: }20 Std.\\
\hline
\end{tabular}
\begin{tabular}{|p{15.3cm}|}
\hline
\textbf{AP-Teilnehmer: }Olga Miloevich, Tim Ellhoff, Tobias Dellert\\
\hline
\textbf{Beschreibung: }Entwicklung des GUI-Prototyps.\\
\hline
\textbf{Voraussetzung: }Kundengespräch erfolgt.\\
\hline 
\textbf{Ziele: }Mit dem Prototyp lassen sich die Leistungsfaktoren pr"asentieren.\\
\hline 
\end{tabular}
\begin{verbatim}

\end{verbatim}
\begin{tabular}{|p{7.43cm}|p{7.43cm}|}
\hline
\textbf{AP Titel: }2.1.4 Prototyp-Vorstellung & \textbf{AP Nummer: }7\\ 
\hline
\textbf{Dauer: }16.05.14 & \textbf{Aufwand: }2 Std.\\
\hline
\end{tabular}
\begin{tabular}{|p{15.3cm}|}
\hline
\textbf{AP-Teilnehmer: }Daniel Pupat\\
\hline
\textbf{Beschreibung: }Vorstellung dem Kunde des GUI-Prototyps.. \\
\hline
\textbf{Voraussetzung: }Mit dem GUI-Prototyp lassen sich die Leistungsfaktoren pr"asentieren.\\
\hline 
\textbf{Ziele: } Vorstellung erfolgt.\\
\hline 
\end{tabular}
\begin{verbatim}

\end{verbatim}
\begin{tabular}{|p{7.43cm}|p{7.43cm}|}
\hline
\textbf{AP Titel: }2.1.5 Entwicklungsphase II & \textbf{AP Nummer: }8 \\ 
\hline
\textbf{Dauer: }15.05.14-17.05.14 & \textbf{Aufwand: }10 Std.\\
\hline
\end{tabular}
\begin{tabular}{|p{15.3cm}|}
\hline
\textbf{AP-Teilnehmer: }Tim Wiechers, Daniel Pupat\\
\hline
\textbf{Beschreibung: } \\
\hline
\textbf{Voraussetzung: }\\
\hline 
\textbf{Ziele: }\\
\hline 
\end{tabular}
\begin{verbatim}

\end{verbatim}
\begin{tabular}{|p{7.43cm}|p{7.43cm}|}
\hline
\textbf{AP Titel: }2.2 Anforderungsspezifikation Dokument & \textbf{AP Nummer: }9\\ 
\hline
\textbf{Dauer: }05.05.14 - 21.05.14 & \textbf{Aufwand: }26 Std.\\
\hline
\end{tabular}
\begin{tabular}{|p{15.3cm}|}
\hline
\textbf{AP-Teilnehmer: }alle Mitglieder\\
\hline
\textbf{Beschreibung: } Herstellung des PDF-Dokumentes Anforderungsspezifikation.\\
\hline
\textbf{Voraussetzung: }Die Unterpunkte des Kundengespr"achs wurden abgearbeitet. \\
\hline 
\textbf{Ziele: }Der Ist-Zustand ist analysiert und dokumentiert.\\
\hline 
\end{tabular}
\begin{verbatim}

\end{verbatim}
\begin{tabular}{|p{7.43cm}|p{7.43cm}|}
\hline
\textbf{AP Titel: }2.2.1 Einleitung & \textbf{AP Nummer: }10\\ 
\hline
\textbf{Dauer: }05.05.14 & \textbf{Aufwand: }4 Std.\\
\hline
\end{tabular}
\begin{tabular}{|p{15.3cm}|}
\hline
\textbf{AP-Teilnehmer: }Patrick Hollatz\\
\hline
\textbf{Beschreibung: }Erledigung des Projektplanteils. Die Projekt"ubersicht liefert die Ziele, Hauptarbeitsaktivit"aten und -produkte, die Hauptmeilsteine und einen groben Zeitplan, die Erfassung der Ressourcen, die zur Verf"ugung stehenden Budgets sowie die Kontaktdaten des Kunden und Informationen "uber die Mitarbeiter am Projekt. Des Weiteren beinhaltet die Einleitung eine "Ubersicht der auszuliefernden Produke mit Terminangaben, sowie Informationen "uber die Evoluation des Plans und die Festlegung von Definitionen bzw. Akronymen.\\
\hline
\textbf{Voraussetzung: }Projektplanteil ist fertig.\\
\hline 
\textbf{Ziele: }Projektplanteil wird zum Gruppenreview weitergegeben.\\
\hline 
\end{tabular}
\begin{verbatim}

\end{verbatim}
\begin{tabular}{|p{7.43cm}|p{7.43cm}|}
\hline
\textbf{AP Titel: }2.2.2 Allgemeine Beschreibungen & \textbf{AP Nummer: }11\\ 
\hline
\textbf{Dauer: }05.05.14-15.05.14& \textbf{Aufwand: }20 Std.\\
\hline
\end{tabular}
\begin{tabular}{|p{15.3cm}|}
\hline
\textbf{AP-Teilnehmer: }Tim Ellhoff, Tobias Dellert, Daniel Pupat\\
\hline
\textbf{Beschreibung: }siehe Unterpunkte.\\
\hline
\textbf{Voraussetzung: }\\
\hline 
\textbf{Ziele: }\\
\hline 
\end{tabular}
\begin{verbatim}

\end{verbatim}
\begin{tabular}{|p{7.43cm}|p{7.43cm}|}
\hline
\textbf{AP Titel: }2.2.2.1 Vorgang der Ist-Analyse & \textbf{AP Nummer: }12\\ 
\hline
\textbf{Dauer: }05.05.14-10.05.14 & \textbf{Aufwand: }10 Std.\\
\hline
\end{tabular}
\begin{tabular}{|p{15.3cm}|}
\hline
\textbf{AP-Teilnehmer: }Tobias Dellert, Tim Ellhoff\\
\hline
\textbf{Beschreibung: }siehe Unterpunkte.\\
\hline
\textbf{Voraussetzung: }Kundengesrp"ach erfolgt.\\
\hline 
\textbf{Ziele: }Die Unterpunkte des Kundengespr"achs wurden abgearbeitet, somit ist der Ist-Zustand analysiert und dokumentiert.\\
\hline 
\end{tabular}

\begin{tabular}{|p{7.43cm}|p{7.43cm}|}
\hline
\textbf{AP Titel: }2.2.2.1.1 Analyse "ahnlicher Systeme& \textbf{AP Nummer: }13\\
\textbf{Dauer: }10.05.14-13.05.14& \textbf{Aufwand: }6 Std.\\
\hline
\end{tabular}
\begin{tabular}{|p{15.3cm}|}
\hline
\textbf{AP-Teilnehmer: }Patrick Hollatz, Daniel Pupat\\
\hline
\textbf{Beschreibung: }Das ermittelte Gesamtbild wird mit "ahnlichen bew"ahrten Softwaresystemen verglichen. Deren Eingenheiten und daraus resultierende Vor- und Nachteile werden erschlossen und die Systeme als Ganzes bewertet.\\
\hline
\textbf{Voraussetzung: }\\
\hline 
\textbf{Ziele: }Analyse "ahnlicher Systeme.\\
\hline 
\end{tabular}

\begin{tabular}{|p{7.43cm}|p{7.43cm}|}
\hline
\textbf{AP Titel: }1.2.2.1.2 Vorbereitung auf Kundengespr"ach& \textbf{AP Nummer: }14\\ 
\hline
\textbf{Dauer: }xx.05.14& \textbf{Aufwand: } 5 Std.\\
\hline
\end{tabular}
\begin{tabular}{|p{15.3cm}|}
\hline
\textbf{AP-Teilnehmer: }Tim Ellhoff, Tobias Dellert\\
\hline
\textbf{Beschreibung: }Um ein effektives, aufschlussreiches Interview zu gew"ahrleisten, ist eine konkrete Vorstellung der zu stellenden Fragen von N"oten. Ziel ist es, nach M"oglichkeit alle Anforderungen und W"unsche des Kunden, sowohl die bewussten, als auch die unbewussten, herauszustellen. Daf"ur wird hier ein Fragenkatalog erstellt, der auf Kunden mit beliebigem Wissenstand "uber technische Details anwendbar ist.\\
\hline
\textbf{Voraussetzung: }Fragen gesammelt.\\
\hline 
\textbf{Ziele: }Der entstehende Fragenkatalog muss weitestgehend alle Fragen enthalten, die f"ur genaues Verst"andnis der W"unsche und Anforderungen des Kunden n"otig sind.\\
\hline 
\end{tabular}

\begin{tabular}{|p{7.43cm}|p{7.43cm}|}
\hline
\textbf{AP Titel: }1.2.2.1.4 Auswertung des Kundengespr"achs& \textbf{AP Nummer: }15\\ 
\hline
\textbf{Dauer: }xx.05.14& \textbf{Aufwand: } 5 Std.\\
\hline
\end{tabular}
\begin{tabular}{|p{15.3cm}|}
\hline
\textbf{AP-Teilnehmer: }alle Mitglieder\\
\hline
\textbf{Beschreibung: }Die erhaltenen Ausk"unfte werden zusammengebracht und zu einem gro"sen Bild aus Anforderungen zusammengef"ugt.\\
\hline
\textbf{Voraussetzung: }Kundengespräch erfolgt.\\
\hline 
\textbf{Ziele: }Das erhaltene Gesamtbild enth"alt alle gegebenen Anforderungen, die nach dem Kundengespräch vom Kunden als Leistungsfaktoren angesehen werden.\\
\hline 
\end{tabular}

\begin{tabular}{|p{7.43cm}|p{7.43cm}|}
\hline
\textbf{AP Titel: }2.2.2.2 Vorgang der Soll-Analyse& \textbf{AP Nummer: }16\\ 
\hline
\textbf{Dauer: }09.05.14 - 15.05.14& \textbf{Aufwand: } 10 Std.\\
\hline
\end{tabular}
\begin{tabular}{|p{15.3cm}|}
\hline
\textbf{AP-Teilnehmer: }Daniel Pupat, Tim Ellhoff, Patrick Hollatz, Tim Wiechers\\
\hline
\textbf{Beschreibung: }\\
\hline
\textbf{Voraussetzung: }Kundengespräch erfolgt.\\
\hline 
\textbf{Ziele: }\\
\hline 
\end{tabular}

\begin{tabular}{|p{7.43cm}|p{7.43cm}|}
\hline
\textbf{AP Titel: 2.2.2.2.1 Herausstellen der Produktperspektiven& \textbf{AP Nummer: }17\\ 
\hline
\textbf{Dauer: }15.05.14 - 19.05.14& \textbf{Aufwand: } 8 Std.\\
\hline
\end{tabular}
\begin{tabular}{|p{15.3cm}|}
\hline
\textbf{AP-Teilnehmer: }Tim Ellhoff, Daniel Pupat, Olga Miloevich\\
\hline
\textbf{Beschreibung: }Hier werden die gegebenen Rahmenbedienungen und M"oglichkeiten analysiert, um die realistische Durchf"uhrung und dessen Aufwand zu erfassen. Beachtet und analysiert werden folgende Bereiche:
\begin{enumerate}
 \item Systemschnittstellen
 \item Benutzerschnittstellen
 \item Hardwareschnittstellen
 \item Softwareschnittstellen
 \item Kommunikationsschnittstellen
 \item Speicherbeschr"ankungen
 \item Betriebsmodi
 \item lokale Anpassungen
\end{enumerate}
\\
\hline
\textbf{Voraussetzung: }Ist- und Soll-Analyse erfolgt.\\
\hline 
\textbf{Ziele: } Die in der Beschreibung aufgez"ahlten Punkte k"onnen in der Anforderungsspezifikation eindeutig und exakt beschrieben werden.\\
\hline 
\end{tabular}

\begin{tabular}{|p{7.43cm}|p{7.43cm}|}
\hline
\textbf{AP Titel: }2.2.2.2.2 Anwendungsf"alle& \textbf{AP Nummer: }18\\ 
\hline
\textbf{Dauer: }17.05.14 - 19.05.14& \textbf{Aufwand: } 8 Std.\\
\hline
\end{tabular}
\begin{tabular}{|p{15.3cm}|}
\hline
\textbf{AP-Teilnehmer: }Tobias Dellert, Patrick Hollatz\\
\hline
\textbf{Beschreibung: }Siehe Unterpunkte.\\
\hline
\textbf{Voraussetzung: }Kundengespr"ach ausgewertet.\\
\hline 
\textbf{Ziele: }Anwendungsf"alle aller Grundfunktionen sind erfasst und kurz beschrieben.\\
\hline 
\end{tabular}

\begin{tabular}{|p{7.43cm}|p{7.43cm}|}
\hline
\textbf{AP Titel: }2.2.2.2.2.1 Auflistung der Anwendungsf"alle& \textbf{AP Nummer: }19\\ 
\hline
\textbf{Dauer: }17.05.14& \textbf{Aufwand: } 2 Std.\\
\hline
\end{tabular}
\begin{tabular}{|p{15.3cm}|}
\hline
\textbf{AP-Teilnehmer: }Tobias Dellert\\
\hline
\textbf{Beschreibung: }Die einzelnen Anwendungsf"alle werden hier aufgelistet und kurz beschrieben.\\
\hline
\textbf{Voraussetzung: }Kundengespr"ach ausgewertet.\\
\hline 
\textbf{Ziele: }Anwendungsf"alle aller Grundfunktionen sind erfasst und kurz beschrieben.\\
\hline 
\end{tabular}

\begin{tabular}{|p{7.43cm}|p{7.43cm}|}
\hline
\textbf{AP Titel: }2.2.2.2.2 Anwendungs- und Sequenzdiagramme& \textbf{AP Nummer: }20\\ 
\hline
\textbf{Dauer: }15.05.14& \textbf{Aufwand: } 5 Std.\\
\hline
\end{tabular}
\begin{tabular}{|p{15.3cm}|}
\hline
\textbf{AP-Teilnehmer: }Tim Wiechers, Daniel Pupat\\
\hline
\textbf{Beschreibung: }Die Anwendungsf"alle werden passenderweise in Anwendungsdiagramme "ubertragen, wobei bei komplexeren F"allen zum Wohle der "Ubersicht auch Sequenzdiagramme hinzugezogen werden k"onnen.\\
\hline
\textbf{Voraussetzung: }Anwendungsf"alle ausgewertet.\\
\hline 
\textbf{Ziele: }Die Anwendungsf"alle der Grundfunktionen sind in einem Anwendungsfalldiagramm dargestellt.\\
\hline 
\end{tabular}

\begin{tabular}{|p{7.43cm}|p{7.43cm}|}
\hline
\textbf{AP Titel: }2.2.2.2.3 Charakteristika & \textbf{AP Nummer: }21\\ 
\hline
\textbf{Dauer: }18.05.14& \textbf{Aufwand: } 4 Std.\\
\hline
\end{tabular}
\begin{tabular}{|p{15.3cm}|}
\hline
\textbf{AP-Teilnehmer: }Daniel Pupat\\
\hline
\textbf{Beschreibung: }Sinn dieses Arbeitspakets ist, einen vollst"andigen Einblick des f"ur uns relevanten Ausschnitts der Realit"at zu erhalten, um damit in sich geschlossene Strukturen von den einzelnen Akteuren untereinander zu schaffen und damit einen verbesserten "Uberblick der m"olichen Anwendungsf"älle zu bekommen. Hierzu werden Benutzerklassen erstellt, die so gut wie m"oglich die unterschiedlichsten Charaktere, Umst"ande und Motive abdecken.
\\
\hline
\textbf{Voraussetzung: }\\
\hline 
\textbf{Ziele: }Die entstandenen Persona sind umfassend und untereinander verschieden genug, um alle m"oglichen Anwendungsf"alle in Bezug auf die erforderten Grundfunktionen der Software zu erhalten.
\\
\hline 
\end{tabular}

\begin{tabular}{|p{7.43cm}|p{7.43cm}|}
\hline
\textbf{AP Titel: }2.2.2.2.4 Einschr"ankungen& \textbf{AP Nummer: }22\\ 
\hline
\textbf{Dauer: }16.05.14 - 17.05.14& \textbf{Aufwand: } 6 Std.\\
\hline
\end{tabular}
\begin{tabular}{|p{15.3cm}|}
\hline
\textbf{AP-Teilnehmer: }Patrick Hollatz\\
\hline
\textbf{Beschreibung: }Es werden sowohl technische, als auch gesetzliche Rahmenbedingungen festgelegt. Zus"atzlich sollen sicherheitskritische Aspekte durch genauere Spezifikationen der einzelnen Schnittstellen im vorigen Arbeitspaket 2.2.1 begutachtet werden.\\
\hline
\textbf{Voraussetzung: }Herausstellen der Produktperspektiven.\\
\hline 
\textbf{Ziele: }Die technischen und gesetzlichen Rahmenbedingungen sind eindeutig spezifiziert. Alle sicherheitskritischen Aspekte wurden erfasst.\\
\hline 
\end{tabular}

\begin{tabular}{|p{7.43cm}|p{7.43cm}|}
\hline
\textbf{AP Titel: }2.2.2.2.5 "uberblick, sowie Ausblick in die Zukunft& \textbf{AP Nummer: }23\\ 
\hline
\textbf{Dauer: }18.05.14 - 19.05.14& \textbf{Aufwand: } 4 Std.\\
\hline
\end{tabular}
\begin{tabular}{|p{15.3cm}|}
\hline
\textbf{AP-Teilnehmer: }Tim Ellhoff\\
\hline
\textbf{Beschreibung: }"1berblick der Abh"angigkeiten von projekteigenen Faktoren untereinander, sowie ein Blick in die nahe Zukunft, was Ver"anderungen und Erweiterungen sowohl im technischen und rechtlichen Bereich, sowie im Anwendungsumfeld der zu entwickelnden Software angeht. Die Anforderungen sollen entsprechend ver"ander-
 und anpassbar sein.\\
\hline
\textbf{Voraussetzung: }\\
\hline 
\textbf{Ziele: }Die Abh"angigkeit des Entwicklungs- und Erweiterungsprozess von einzelnen pojekteigenen Faktoren wurden umfassend herausgestellt. Zu erwartende Ver"anderungen in der nahen Zukunft( sofern vorhanden ) wurden dokumentiert.\\
\hline 
\end{tabular}

\begin{tabular}{|p{7.43cm}|p{7.43cm}|}
\hline
\textbf{AP Titel: }2.2.3 Detallierte Beschreibungen & \textbf{AP Nummer: }24\\ 
\hline
\textbf{Dauer: }16.05.14 - 19.05.14& \textbf{Aufwand: } 12 Std.\\
\hline
\end{tabular}
\begin{tabular}{|p{15.3cm}|}
\hline
\textbf{AP-Teilnehmer: }Tobias Dellert, Daniel Pupat, Olga Miloevich\\
\hline
\textbf{Beschreibung: }\\
\hline
\textbf{Voraussetzung: }\\
\hline 
\textbf{Ziele: }\\
\hline 
\end{tabular}

\begin{tabular}{|p{7.43cm}|p{7.43cm}|}
\hline
\textbf{AP Titel: }2.2.3.1 Datenmodell& \textbf{AP Nummer: }25\\ 
\hline
\textbf{Dauer: }15.05.14 - 19.05.14& \textbf{Aufwand: } 12 Std.\\
\hline
\end{tabular}
\begin{tabular}{|p{15.3cm}|}
\hline
\textbf{AP-Teilnehmer: }Tobias Dellert\\
\hline
\textbf{Beschreibung: }Das zu entwickelnde Datenmodell soll eine konzeptionelle Darstellung von dem Weltausschnitt in dem Quizz-App, als UML-Diagramm, werden.
\\
\hline
\textbf{Voraussetzung: }.\\
\hline 
\textbf{Ziele: }Die Informationen und deren Beziehungen sind dem Weltausschnitt entsprechend in das Datenmodell gesetzt und es l"asst sich somit als Konzept f"ur sp"atere Spezifizierungen benutzen.\\
\hline 
\end{tabular}

\begin{tabular}{|p{7.43cm}|p{7.43cm}|}
\hline
\textbf{AP Titel: }2.2.3.2 Detallierte Anwendungsf"alle& \textbf{AP Nummer: }26\\ 
\hline
\textbf{Dauer: }07.05.14 - 12.05.14& \textbf{Aufwand: } 20 Std.\\
\hline
\end{tabular}
\begin{tabular}{|p{15.3cm}|}
\hline
\textbf{AP-Teilnehmer: }Daniel Pupat, Olga Miloevich\\
\hline
\textbf{Beschreibung: }Die bereits aufgelisteten Anwendungsf"alle werden mit den Personas und genaueren Beschreibungen erg"anzt. Das Hineinversetzen in die unterschiedlichen Personas hilft die bisherige Funktionalit"at aus der Sicht der Benutzer zu sehen.\\
\hline
\textbf{Voraussetzung: }\\
\hline 
\textbf{Ziele: }Die Anwendungsf"älle m"ussen insgesamt die ganze gew"unschte Funktionalit"at der Software wiedergeben.\\
\hline 
\end{tabular}

\begin{tabular}{|p{7.43cm}|p{7.43cm}|}
\hline
\textbf{AP Titel: }2.2.3.2.1 Screenshots der Benutzeroberfl"ache& \textbf{AP Nummer: }27\\ 
\hline
\textbf{Dauer: }15.05.14& \textbf{Aufwand: } 1 Std.\\
\hline
\end{tabular}
\begin{tabular}{|p{15.3cm}|}
\hline
\textbf{AP-Teilnehmer: }Tim Wiechers\\
\hline
\textbf{Beschreibung: }Die Screenshots der Benutzeroberfl"ache werden benutzt, um die spezifischen Anwendungsf"älle zu illustrieren. Dazu werden die Screenshots an de passenden Stellen beschriftet.\\
\hline
\textbf{Voraussetzung: }GUI-Prototyp ist fertig.\\
\hline 
\textbf{Ziele: }\\
\hline 
\end{tabular}

\begin{tabular}{|p{7.43cm}|p{7.43cm}|}
\hline
\textbf{AP Titel: }2.2.3.3 Aktionen& \textbf{AP Nummer: }28\\ 
\hline
\textbf{Dauer: }09.05.14 - 12.05.14& \textbf{Aufwand: } 8 Std.\\
\hline
\end{tabular}
\begin{tabular}{|p{15.3cm}|}
\hline
\textbf{AP-Teilnehmer: }Daniel Pupat\\
\hline
\textbf{Beschreibung: }Die bisher beschriebenen Anwendungsf"älle werden noch genauer ausgef"uhrt, sodass jede aufgelistete Aktion eine einzelne Nutzung einer beliebigen Funktion( Knopfdruck, Klick ins Textfeld ) der Software beschreibt\\
\hline
\textbf{Voraussetzung: }\\
\hline 
\textbf{Ziele: }Alle Anwendungsf "alle sind in die kleinstm"oglichen Aktionen aufgeteilt und geordnet.\\
\hline 
\end{tabular}

\begin{tabular}{|p{7.43cm}|p{7.43cm}|}
\hline
\textbf{AP Titel: }2.2.3.4 Systemattribute& \textbf{AP Nummer: }29\\ 
\hline
\textbf{Dauer: }13.05.14 - 18.05.14& \textbf{Aufwand: } 16 Std.\\
\hline
\end{tabular}
\begin{tabular}{|p{15.3cm}|}
\hline
\textbf{AP-Teilnehmer: }Tobias Dellert\\
\hline
\textbf{Beschreibung: }\\
\hline
\textbf{Voraussetzung: }\\
\hline 
\textbf{Ziele: }Die Informationen und deren Beziehungen sind dem Weltausschnitt entsprechend in das Datenmodell gesetzt und es l"asst sich somit als Konzept f"ur sp"atere Spezifizierungen benutzen.\\
\hline 
\end{tabular}

\begin{tabular}{|p{7.43cm}|p{7.43cm}|}
\hline
\textbf{AP Titel: }2.5 Anforderungsspezifikation komplettieren& \textbf{AP Nummer: }30\\ 
\hline
\textbf{Dauer: }18.05.14& \textbf{Aufwand: } 4 Std.\\
\hline
\end{tabular}
\begin{tabular}{|p{15.3cm}|}
\hline
\textbf{AP-Teilnehmer: }Tobias Dellert, Daniel Pupat, Tim Ellhoff, Olga Miloevich\\
\hline
\textbf{Beschreibung: }Maßnahme, um Ergebnisse zusammenzufassen und zu korrigieren. Zus"atzlich wird noch besonders Wert auf langfristige Schwachstellen des momentanen Ist-Zustand gelegt.\\
\hline
\textbf{Voraussetzung: }\
\hline 
\textbf{Ziele: }Sobald der Projektplanteil fertig ist, kann er zum Gruppenreview weitergegeben werden.\\
\hline 
\end{tabular}

\begin{tabular}{|p{7.43cm}|p{7.43cm}|}
\hline
\textbf{AP Titel: }2.6 Angebot f"ur den Kunden& \textbf{AP Nummer: }31\\ 
\hline
\textbf{Dauer: }18.05.14 - 19.05.14& \textbf{Aufwand: } 6 Std.\\
\hline
\end{tabular}
\begin{tabular}{|p{15.3cm}|}
\hline
\textbf{AP-Teilnehmer: }alle Mitglieder\\
\hline
\textbf{Beschreibung: }Angebot f"ur den Kunden inkl. Kostenaufwand wird erstellt.\\
\hline
\textbf{Voraussetzung: }Fragen gesammelt.\\
\hline 
\textbf{Ziele: }Das Angebot f"ur den Kunden ist hinreichend erstellt und kann nun dem Gruppenreview "ubergeben werden.\\
\hline 
\end{tabular}

\begin{tabular}{|p{7.43cm}|p{7.43cm}|}
\hline
\textbf{AP Titel: }2.7 Gruppeninternes Review u. Finalisierung Anforderungsspezifikation& \textbf{AP Nummer: }32\\ 
\hline
\textbf{Dauer: }20.05.14& \textbf{Aufwand: } 10 Std.\\
\hline
\end{tabular}
\begin{tabular}{|p{15.3cm}|}
\hline
\textbf{AP-Teilnehmer: }alle Mitglieder\\
\hline
\textbf{Beschreibung: }Das Dokument wird in der Gruppe auf den Pr"ufstand gestellt und ggf. noch ver"andert bzw. verbessert.\\
\hline
\textbf{Voraussetzung: }2.5 Anforderungsspezifikation komplettieren, 2.6 Angebot.\\
\hline 
\textbf{Ziele: }Nach der Diskussion in der Gruppe wurde das Dokument optimiert und fertiggestellt.\\
\hline 
\end{tabular}
