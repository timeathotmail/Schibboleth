\section{Arbeitspakete, Zeitplan und Budget}

\textit{Bearbeitet von: Dario Treffenfeld - Mäder, Jannes Uken und Nils Sören Oja}\\

\subsection{Arbeitspakete}\label{aps}

Hier listen wir die Arbeitspakete auf. Wir planen die Anforderungsspezifikation detailliert und die anderen Phasen nur grob. 

\subsubsection{Projektplan}
\ldots
\subsubsection{Anforderungsspezifikation}


\begin{tabular}{|p{7.43cm}|p{7.43cm}|}
\hline
\textbf{AP Titel: }1. Einführung & \textbf{AP Nummer: }1 \\ 
\hline
\textbf{Dauer: }21.10.13 - 22.10.13 & \textbf{Aufwand: }8 Std.\\
\hline
\end{tabular}
\begin{tabular}{|p{15.3cm}|}
\hline
\textbf{AP-Teilnehmer: }Sandor Herms, Olga Miloevich\\
\hline
\textbf{Beschreibung: }Bearbeiten des Abschnittes Einleitung\\
\hline
\textbf{Voraussetzung: }keine\\
\hline 
\textbf{Ziele: }Die Punkte Zweck, Rahmen, Definitionen, Akronyme und Abkürzungen sowie Referenzen und die Übersicht über das Dokument sind erstellt.\\
\hline 
\end{tabular}
\begin{verbatim}

\end{verbatim}
\begin{tabular}{|p{7.43cm}|p{7.43cm}|}
\hline
\textbf{AP Titel: }Kundengespräch & \textbf{AP Nummer: 2} \\ 
\hline
\textbf{Dauer: }23.10.13 & \textbf{Aufwand: }4 Std.\\
\hline
\end{tabular}
\begin{tabular}{|p{15.3cm}|}
\hline
\textbf{AP-Teilnehmer: }Sandor Herms, Olga Miloevich\\
\hline
\textbf{Beschreibung: }Gespräch mit dem Kunden um dessen Wünsche und Erwartungen an das Produkt festzustellen.\\
\hline
\textbf{Voraussetzung: }keine\\
\hline 
\textbf{Ziele: }Klarheit über die Anforderungen herstellen.\\
\hline 
\end{tabular}
\begin{verbatim}

\end{verbatim}
\begin{tabular}{|p{7.43cm}|p{7.43cm}|}
\hline
\textbf{AP Titel: }2.1 Ist-Analyse und 2.2 Produktperspektive & \textbf{AP Nummer: }3 \\ 
\hline
\textbf{Dauer: }23.10.13 - 27.10.13 & \textbf{Aufwand: }20 Std.\\
\hline
\end{tabular}
\begin{tabular}{|p{15.3cm}|}
\hline
\textbf{AP-Teilnehmer: }Sandor Herms, Olga Miloevich\\
\hline
\textbf{Beschreibung: }Analyse des momentanen Zustands betreffend der Ressourcen und Kompetenzen der Entwickler und des Kunden.
 Beschreibung der Einbettung des Produkts in das Gesamtsystem mit den dazu n"otigen Schnittstellen.\\
\hline
\textbf{Voraussetzung: }stattgefundenes Kundengespräch\\
\hline 
\textbf{Ziele: }Die Punkte Ist-Analyse und Produktperspektive der allgemeinen Beschreibung abarbeiten.\\
\hline 
\end{tabular}
\begin{verbatim}

\end{verbatim}
\begin{tabular}{|p{7.43cm}|p{7.43cm}|}
\hline
\textbf{AP Titel: }2.3 Anwendungsfälle (kurz) bis 2.7 Ausblick & \textbf{AP Nummer: }4 \\ 
\hline
\textbf{Dauer: }21.10.13 - 25.10.13 & \textbf{Aufwand: }20 Std.\\
\hline
\end{tabular}
\begin{tabular}{|p{15.3cm}|}
\hline
\textbf{AP-Teilnehmer: }Sylvia Kamche Tague, Nils Sören Oja\\
\hline
\textbf{Beschreibung: }Grobe Liste der Anwendungsfälle aus den Mindestanforderungen erstellen.\\
\hline
\textbf{Voraussetzung: }keine\\
\hline 
\textbf{Ziele: }Übersicht schaffen, Liste ist auch für den Prototypen wichtig\\
\hline 
\end{tabular}
\begin{verbatim}

\end{verbatim}
\begin{tabular}{|p{7.43cm}|p{7.43cm}|}
\hline
\textbf{AP Titel: }3. Datenmodell & \textbf{AP Nummer: }5 \\ 
\hline
\textbf{Dauer: }25.10.13 - 30.10.13 & \textbf{Aufwand: }20 Std.\\
\hline
\end{tabular}
\begin{tabular}{|p{15.3cm}|}
\hline
\textbf{AP-Teilnehmer: }Sandor Herms, Olga Miloevich\\
\hline
\textbf{Beschreibung: }Das  Datenmodell beschreibt die in unserem Bibliotheksprojekt verwendeten Daten\\
\hline
\textbf{Voraussetzung: }2.1 Ist-Analyse und 2.2 Produktperspektive\\
\hline 
\textbf{Ziele: }Fertiges Datenmodel, dass die Struktur des Systems veranschaulicht\\
\hline 
\end{tabular}
\begin{verbatim}

\end{verbatim}
\begin{tabular}{|p{7.43cm}|p{7.43cm}|}
\hline
\textbf{AP Titel: }3.2 Anwendungsfälle & \textbf{AP Nummer: }6 \\ 
\hline
\textbf{Dauer: }28.10.13 - 01.11.13 & \textbf{Aufwand: }20 Std.\\
\hline
\end{tabular}
\begin{tabular}{|p{15.3cm}|}
\hline
\textbf{AP-Teilnehmer: }Sylvia Kamche Tague, Nils Sören Oja\\
\hline
\textbf{Beschreibung: }Aufzählung aller möglichen Szenarien die vorkommen können.\\
\hline
\textbf{Voraussetzung: }2.3 Anwendungsfälle (kurz) bis 2.7 Ausblick\\
\hline 
\textbf{Ziele: }Detaillierte Liste von Anwendungsfällen\\
\hline 
\end{tabular}
\begin{verbatim}

\end{verbatim}
\begin{tabular}{|p{7.43cm}|p{7.43cm}|}
\hline
\textbf{AP Titel: }3.3 Aktionen & \textbf{AP Nummer: }7\\ 
\hline
\textbf{Dauer: }4.11.13 - 06.11.13 & \textbf{Aufwand: }12 Std.\\
\hline
\end{tabular}
\begin{tabular}{|p{15.3cm}|}
\hline
\textbf{AP-Teilnehmer: }Sylvia Kamche Tague, Nils Sören Oja\\
\hline
\textbf{Beschreibung: }Aufzählung aller möglichen durchführbaren Handlungen die beim durchführen eines Anwendungsfall auftreten. \\
\hline
\textbf{Voraussetzung: }3.2 Anwendungsfälle\\
\hline 
\textbf{Ziele: }Detaillierte Liste aller Aktionen, die in den Anwendungsfällen vorkommen\\
\hline 
\end{tabular}
\begin{verbatim}

\end{verbatim}
\begin{tabular}{|p{7.43cm}|p{7.43cm}|}
\hline
\textbf{AP Titel: }3.4 Entwurfseinschr"ankungen und 3.5 Softwaresystemattribute & \textbf{AP Nummer: }8 \\ 
\hline
\textbf{Dauer: }6.11.13 - 12.11.13 & \textbf{Aufwand: }20 Std.\\
\hline
\end{tabular}
\begin{tabular}{|p{15.3cm}|}
\hline
\textbf{AP-Teilnehmer: }Sandor Herms, Olga Miloevich\\
\hline
\textbf{Beschreibung: }Merkmale des Entwurfes und des Systems, die uns Einschränken. \\
\hline
\textbf{Voraussetzung: }Datenmodell\\
\hline 
\textbf{Ziele: }Liste von Entwurfseinschr"ankungen und Softwaresystemattribute.\\
\hline 
\end{tabular}
\begin{verbatim}

\end{verbatim}
\begin{tabular}{|p{7.43cm}|p{7.43cm}|}
\hline
\textbf{AP Titel: }GUI-Prototyp: Vorbereitung & \textbf{AP Nummer: }9\\ 
\hline
\textbf{Dauer: }21.10.13 - 23.10.13 & \textbf{Aufwand: }12 Std.\\
\hline
\end{tabular}
\begin{tabular}{|p{15.3cm}|}
\hline
\textbf{AP-Teilnehmer: }Dario Treffenfeld-Mäder, Jannes Uken\\
\hline
\textbf{Beschreibung: }Erarbeitung eines Konzepts für die GUI unter Gesichtspunkten der Bedienbarkeit, Optik sowie Umsetztbarkeit und Effizienz.\\
\hline
\textbf{Voraussetzung: }keine\\
\hline 
\textbf{Ziele: }Ein Ziel für die Implementierung definieren.\\
\hline 
\end{tabular}
\begin{verbatim}

\end{verbatim}
\begin{tabular}{|p{7.43cm}|p{7.43cm}|}
\hline
\textbf{AP Titel: }GUI-Prototyp: Implementierung & \textbf{AP Nummer: }10\\ 
\hline
\textbf{Dauer: }24.10.13 - 30.10.13 & \textbf{Aufwand: }28 Std.\\
\hline
\end{tabular}
\begin{tabular}{|p{15.3cm}|}
\hline
\textbf{AP-Teilnehmer: }Dario Treffenfeld-Mäder, Jannes Uken\\
\hline
\textbf{Beschreibung: }Umsetzung des zuvor bestimmten Konzepts.\\
\hline
\textbf{Voraussetzung: }Abschluss von Arbeitspaket 9\\
\hline 
\textbf{Ziele: }Etwas zwecks Visualisierung präsentieren können.\\
\hline 
\end{tabular}
\begin{verbatim}

\end{verbatim}
\begin{tabular}{|p{7.43cm}|p{7.43cm}|}
\hline
\textbf{AP Titel: }GUI-Prototyp: Präsentation & \textbf{AP Nummer: }11\\ 
\hline
\textbf{Dauer: }06.11.13 - 07.11.13 & \textbf{Aufwand: }3 Std.\\
\hline
\end{tabular}
\begin{tabular}{|p{15.3cm}|}
\hline
\textbf{AP-Teilnehmer: }alle\\
\hline
\textbf{Beschreibung: }Vorstellung des Prototyps.\\
\hline
\textbf{Voraussetzung: }GUI-Prototyp: Implementierung\\
\hline 
\textbf{Ziele: }Rückmeldung zur Qualität und Vollständigkeit des Prototyps.\\
\hline 
\end{tabular}
\begin{verbatim}

\end{verbatim}
\begin{tabular}{|p{7.43cm}|p{7.43cm}|}
\hline
\textbf{AP Titel: }Angebot & \textbf{AP Nummer: }12\\ 
\hline
\textbf{Dauer: }6.11.13 & \textbf{Aufwand: }4 Std.\\
\hline
\end{tabular}
\begin{tabular}{|p{15.3cm}|}
\hline
\textbf{AP-Teilnehmer: }Dario Treffenfeld-Mäder, Jannes Uken\\
\hline
\textbf{Beschreibung: }Aufzeigen aller zu gewährleistenden und nicht zu gewährleistenden Merkmale und Szenarien des Projektes sowie eine Kosten Einschätzung.\\
\hline
\textbf{Voraussetzung: }keine\\
\hline 
\textbf{Ziele: }Fertiges Angebot in Briefform.\\
\hline 
\end{tabular}


\subsubsection{Architekturbeschreibung, Schnittstellenbeschreibung, Testplan und Blackbox-Tests}

\begin{tabular}{|p{7.43cm}|p{7.43cm}|}
\hline
\textbf{AP Titel: }Architekturbeschreibung, Schnittstellenbeschreibung, Testplan und Blackbox-Tests & \textbf{AP Nummer: }3\\ 
\hline
\textbf{Dauer: }18.11.13 - 22.12.13& \textbf{Aufwand: }150 Std.\\
\hline
\end{tabular}
\begin{tabular}{|p{15.3cm}|}
\hline
\textbf{AP-Teilnehmer: }alle\\
\hline
\textbf{Beschreibung: }Hier werden zur besseren Implementierbarkeit die zusammengestellten Konzepte der Anforderungsspezifikation in technischer Form umgestaltet und denkbare Probleme und Schwierigkeiten technischer und personeller Art entsprechende Lösungsstrategien zugeordnet.  \\
\hline
\textbf{Voraussetzung: }Anforderungsspezifikation\\
\hline 
\textbf{Ziele: }Es soll insgesamt eine solide und anwendbare Architektur, was Gruppenmanagement und Implementierungsmodelle angeht,  für unser Projekt entstehen.\\
\hline 
\end{tabular}
\begin{verbatim}

\end{verbatim}

\begin{tabular}{|p{7.43cm}|p{7.43cm}|}
\hline
\textbf{AP Titel: }Globale Analyse & \textbf{AP Nummer: }3.1\\ 
\hline
\textbf{Dauer: }18.11.13 - 22.12.13& \textbf{Aufwand: }150 Std.\\
\hline
\end{tabular}
\begin{tabular}{|p{15.3cm}|}
\hline
\textbf{AP-Teilnehmer: }alle\\
\hline
\textbf{Beschreibung: }siehe Unterpunkte\\
\hline
\textbf{Voraussetzung: }Anforderungsspezifikation\\
\hline 
\textbf{Ziele: }Eine umfassende Analyse denkbarer Einflussfaktoren und ein ausreichender Vorrat an entsprechenden Lösungsstrategien\\
\hline 
\end{tabular}
\begin{verbatim}

\end{verbatim}

\begin{tabular}{|p{7.43cm}|p{7.43cm}|}
\hline
\textbf{AP Titel: }Einflussfaktoren & \textbf{AP Nummer: }3.1.1\\ 
\hline
\textbf{Dauer: }18.11.13 - 22.12.13& \textbf{Aufwand: }150 Std.\\
\hline
\end{tabular}
\begin{tabular}{|p{15.3cm}|}
\hline
\textbf{AP-Teilnehmer: }alle\\
\hline
\textbf{Beschreibung: }Hier werden die gegebenen Umstände in technischer und personeller Hinsicht auf Abhängigkeiten untersucht, um so mögliche Einflussfaktoren herauszuarbeiten.\\
\hline
\textbf{Voraussetzung: }Anforderungsspezifikation\\
\hline 
\textbf{Ziele: }Alle Einflussfaktoren, die das Bearbeiten eines Arbeitspaketes stark einschränkt oder komplett behindern, sind herausgearbeitet.  \\
\hline 
\end{tabular}
\begin{verbatim}

\end{verbatim}

\begin{tabular}{|p{7.43cm}|p{7.43cm}|}
\hline
\textbf{AP Titel: }Problem- und Strategienkarten & \textbf{AP Nummer: }3.1.2\\ 
\hline
\textbf{Dauer: }18.11.13 - 22.12.13& \textbf{Aufwand: }150 Std.\\
\hline
\end{tabular}
\begin{tabular}{|p{15.3cm}|}
\hline
\textbf{AP-Teilnehmer: }alle\\
\hline
\textbf{Beschreibung: }Zu den Einflussfaktoren werden Lösungsstrategien entwickelt. \\
\hline
\textbf{Voraussetzung: }Anforderungsspezifikation\\
\hline 
\textbf{Ziele: Zu jedem Einflussfaktor sollen mindestens zwei Lösungsstrategien bereitgestellt werden}\\
\hline 
\end{tabular}
\begin{verbatim}

\end{verbatim}

\begin{tabular}{|p{7.43cm}|p{7.43cm}|}
\hline
\textbf{AP Titel: }Konzeptionelle Sicht & \textbf{AP Nummer: }3.2\\ 
\hline
\textbf{Dauer: }18.11.13 - 22.12.13& \textbf{Aufwand: }150 Std.\\
\hline
\end{tabular}
\begin{tabular}{|p{15.3cm}|}
\hline
\textbf{AP-Teilnehmer: }alle\\
\hline
\textbf{Beschreibung: Ein UML-Diagramm, welches auf hoher Abstraktionsebene
die Struktur unseres zu implementierendes System darstellt.}\\
\hline
\textbf{Voraussetzung: }Anforderungsspezifikation\\
\hline 
\textbf{Ziele: Das Diagramm ist in nicht mehr zusammenfassbare Module bezüglich ihrer
Aufgaben aufgeteilt}\\
\hline 
\end{tabular}
\begin{verbatim}

\end{verbatim}

\begin{tabular}{|p{7.43cm}|p{7.43cm}|}
\hline
\textbf{AP Titel: }Modulsicht & \textbf{AP Nummer: }3.3\\ 
\hline
\textbf{Dauer: }18.11.13 - 22.12.13& \textbf{Aufwand: }150 Std.\\
\hline
\end{tabular}
\begin{tabular}{|p{15.3cm}|}
\hline
\textbf{AP-Teilnehmer: }alle\\
\hline
\textbf{Beschreibung: }Eine detailliertere statische Form der konzeptionellen Sicht, indem dessen grobe Module in Teilmodule unterteilt werden, um ein Arbeitspaket im Ausmaß von höchstens einer Arbeitswoche eines Gruppenmitglieds darzustellen. Zudem werden Schnittstellen
zwischen den Teilmodulen erstellt. \\
\hline
\textbf{Voraussetzung: }Anforderungsspezifikation\\
\hline 
\textbf{Ziele: Die Module sind detailliert genug, um als Anleitung für die Programmierung
zu dienen.}\\
\hline 
\end{tabular}
\begin{verbatim}

\end{verbatim}

\begin{tabular}{|p{7.43cm}|p{7.43cm}|}
\hline
\textbf{AP Titel: }Datensicht & \textbf{AP Nummer: }3.4\\ 
\hline
\textbf{Dauer: }18.11.13 - 22.12.13& \textbf{Aufwand: }150 Std.\\
\hline
\end{tabular}
\begin{tabular}{|p{15.3cm}|}
\hline
\textbf{AP-Teilnehmer: }alle\\
\hline
\textbf{Beschreibung: Das anwendungs- und vorgangsaufzeigende Datenmodell wird bezüglich implementierungsspezifischer Details erweitert. }\\
\hline
\textbf{Voraussetzung: }Anforderungsspezifikation\\
\hline 
\textbf{Ziele: Die Datensicht stellt auf praktische Weise den Übergang vom Anwendungskonzept zur tatsächlichen Implementierungsstruktur dar.}\\
\hline 
\end{tabular}
\begin{verbatim}

\end{verbatim}

\begin{tabular}{|p{7.43cm}|p{7.43cm}|}
\hline
\textbf{AP Titel: }Ausführungssicht & \textbf{AP Nummer: }3.5\\ 
\hline
\textbf{Dauer: }18.11.13 - 22.12.13& \textbf{Aufwand: }150 Std.\\
\hline
\end{tabular}
\begin{tabular}{|p{15.3cm}|}
\hline
\textbf{AP-Teilnehmer: }alle\\
\hline
\textbf{Beschreibung: Stellt den zeitlichen Verlauf unseres Systems während der Anwendung dar.}\\
\hline
\textbf{Voraussetzung: }Anforderungsspezifikation\\
\hline 
\textbf{Ziele: Die Ausführungssicht spezifiziert den Implementierungsvorgang, indem es deutlich einzuhaltende Reihenfolgen miteinander kommunizierender Module aufzeigt.}\\
\hline 
\end{tabular}
\begin{verbatim}

\end{verbatim}

\begin{tabular}{|p{7.43cm}|p{7.43cm}|}
\hline
\textbf{AP Titel: }Zusammenhänge zw. Anwendungsfälle und Architektur & \textbf{AP Nummer: }3.6\\ 
\hline
\textbf{Dauer: }18.11.13 - 22.12.13& \textbf{Aufwand: }150 Std.\\
\hline
\end{tabular}
\begin{tabular}{|p{15.3cm}|}
\hline
\textbf{AP-Teilnehmer: }alle\\
\hline
\textbf{Beschreibung: } Die Anwendungsfälle werden im bisherigen System in Form von Sequenzdiagrammen dargestellt, um die Anwendbarkeit unseres Systems an unserem Projekt
zusätzlich zu sichern.\\
\hline
\textbf{Voraussetzung: }Anforderungsspezifikation\\
\hline 
\textbf{Ziele: Das Funktionieren des Zusammenspiels aller Module in mindestens einem Anwendungsfall muss aufgezeigt sein. Es gibt keine Fehler in der Modulstruktur.}\\
\hline 
\end{tabular}
\begin{verbatim}

\end{verbatim}

\begin{tabular}{|p{7.43cm}|p{7.43cm}|}
\hline
\textbf{AP Titel: }Evolution & \textbf{AP Nummer: }3.7\\ 
\hline
\textbf{Dauer: }18.11.13 - 22.12.13& \textbf{Aufwand: }150 Std.\\
\hline
\end{tabular}
\begin{tabular}{|p{15.3cm}|}
\hline
\textbf{AP-Teilnehmer: }alle\\
\hline
\textbf{Beschreibung: Hier werden Änderungen in unserem System, bei sich ändernden Rahmenbedingungen und Anforderungen aufgezeigt.}\\
\hline
\textbf{Voraussetzung: }Anforderungsspezifikation\\
\hline 
\textbf{Ziele: Die genannten Punkte aus dem Abschnitt Ausblick der Anforderungsspezifikation 
werden mit unserem bisherigen System konfrontiert. Am Ende soll sich das System als sehr gut Wart- und Änderbar herausstellen.}\\
\hline 
\end{tabular}
\begin{verbatim}

\end{verbatim}

\begin{tabular}{|p{7.43cm}|p{7.43cm}|}
\hline
\textbf{AP Titel: }Einführung: Zweck/Status/Definitionen/Referenzen/Übersicht & \textbf{AP Nummer: }3.8\\ 
\hline
\textbf{Dauer: }18.11.13 - 22.12.13& \textbf{Aufwand: }150 Std.\\
\hline
\end{tabular}
\begin{tabular}{|p{15.3cm}|}
\hline
\textbf{AP-Teilnehmer: }alle\\
\hline
\textbf{Beschreibung: Eine Zusammenfassung des Zwecks der Architekturbeschreibung, den Status unseres Projekts und kleinerer Unterstützungen beim Zurechtfinden in diesem Dokument.}\\
\hline
\textbf{Voraussetzung: }Anforderungsspezifikation\\
\hline 
\textbf{Ziele: Für eine nicht in das Projekt eingeweihte Person soll es möglich sein, unsere Vorhaben mit den einzelnen Arbeitsschritten nachzuvollziehen.}\\
\hline 
\end{tabular}
\begin{verbatim}

\end{verbatim}

\begin{tabular}{|p{7.43cm}|p{7.43cm}|}
\hline
\textbf{AP Titel: }Testplan & \textbf{AP Nummer: }13\\ 
\hline
\textbf{Dauer: }18.11.13 - 22.12.13& \textbf{Aufwand: }150 Std.\\
\hline
\end{tabular}
\begin{tabular}{|p{15.3cm}|}
\hline
\textbf{AP-Teilnehmer: }alle\\
\hline
\textbf{Beschreibung: } Erstellung eines Testplans zur besseren Verifizierung der Funktionalitäten. \\
\hline
\textbf{Voraussetzung: }Anforderungsspezifikation\\
\hline 
\textbf{Ziele: } Die wichtigsten Funktionen des Systems und typische Fehlerquellen sind im Testplan enthalten. \\
\hline 
\end{tabular}
\begin{verbatim}

\end{verbatim}

\begin{tabular}{|p{7.43cm}|p{7.43cm}|}
\hline
\textbf{AP Titel: }Blackbox-Tests & \textbf{AP Nummer: }13\\ 
\hline
\textbf{Dauer: }18.11.13 - 22.12.13& \textbf{Aufwand: }150 Std.\\
\hline
\end{tabular}
\begin{tabular}{|p{15.3cm}|}
\hline
\textbf{AP-Teilnehmer: }alle\\
\hline
\textbf{Beschreibung: Die Erfüllung der spezifizierten Anforderungen unseres Systems, ohne
Beachtung der Implementierungsspezifischen Details einzelner Module.}\\
\hline
\textbf{Voraussetzung: }Anforderungsspezifikation\\
\hline 
\textbf{Ziele: Bei allen für einen Anwendungsvorgang verwendeten Schnittstellen stehen BlackBox-Tests bereit.}\\
\hline 
\end{tabular}
\begin{verbatim}

\end{verbatim}

\ldots

\subsubsection{Implementierung}

\begin{tabular}{|p{7.43cm}|p{7.43cm}|}
\hline
\textbf{AP Titel: }Implementierung & \textbf{AP Nummer: }14\\ 
\hline
\textbf{Dauer: }22.12.13 - 23.02.14& \textbf{Aufwand: } 380 Std.\\
\hline
\end{tabular}
\begin{tabular}{|p{15.3cm}|}
\hline
\textbf{AP-Teilnehmer: }alle\\
\hline
\textbf{Beschreibung: }Implementieren des Projekts\\
\hline
\textbf{Voraussetzung: }Architekturbeschreibung, Schnittstellenbeschreibung, Testplan und Blackbox-Tests \\
\hline 
\textbf{Ziele: }Fertig Implementiert\\
\hline 
\end{tabular}

\ldots

\subsection{Zeitplan und Abhängigkeiten}
\newpage
\begin{figure}[H]
\centering
\includegraphics[scale=0.48, angle=90]{gantt.png}
\caption{Gantt-Diagramm}
\end{figure}

\subsection{Ressourcenanforderung}

\begin{figure}[H]
\centering
\includegraphics[width=1.2\textwidth, angle = 90]{Ressourcendiagramm.png}
\caption{Ressourcendiagramm}
\end{figure}

Die benötigten Ressourcen für das Projekt sind die Mitarbeiter. Die Arbeitspakete werden unter Berücksichtigung der Kompetenzen der Mitarbeiter verteilt.
\newline Der Aufwand eines Arbeitspaketes bestimmt gleichzeitig den Bearbeitungszeitraum sowie die Anzahl der beteiligten Mitarbeiter. So können Arbeitspakete mit überschaubarem Inhalt und Aufwand von einer Person bearbeitet werden.
\newline Die Mitarbeiter müssen die ihnen zugeteilten Arbeitspakete innerhalb des festgelegten Zeitraums abschließen. Der genaue Zeitpunkt der Bearbeitung innerhalb dieses Zeitraums ist den Mitarbeitern überlassen.
\newline Die regelmäßigen Besprechungen werden, so fern möglich, nur mit Anwesenheit aller Mitarbeiter abgehalten. Weiterhin sind alle Mitarbeiter an den Punkten 1.4 Referenzen sowie 1.5 Definitionen und Akronyme beteiligt. Referenziert ein Mitarbeiter etwas oder führt er neue Fachbegriffe ein, so aktualisiert er diese Punkte mit einer sinnvollen Erläuterung.

