\subsection{Risikomanagement}

\textit{Bearbeitet von: Olga Miloevich}\\

\textit{Folgender Abschnitt w"urde aus dem Projektplan von der Gruppe YNotZoidberg (Vorlesung Software Projekt 2, WS 2013, S. 12-14) "ubernommen}\\

Um m"ogliche Risiken zu identifizieren und zu beschreiben, haben wir hier eine sogenannte FMEA verwendet. Dies ist als ``Fehlerm"oglichkeits- und Einflu"sanalyse'' bekannt und wird in "ahnlicher Form auch bei Siemens oder Toyota durchgef"uhrt. Hier geht es um die konkreten Risiken, welche die Entwicklergruppe zu ber"ucksichtigen hat. Die FMEA-Tabelle l"asst die eintretende Risiken in Zahlen analysieren. Sie verwendet dazu die Eintrittswahrscheinlichkeit, die Schwere des Fehlers, so wie die Wahrscheinlichkeit, den Fehler fr"uhzeitig zu entdecken.\\
Die Gewichtung sagt, wie schwer das Risiko ist und die Entdeckung wie offensichtlich das Risiko zu entdecken ist. Die Tabelle zeit au"serdem welche Person f"ur welchen Bereich die Verantwortung "ubernommen hat. Zuletzt sind auch m"ogliche Ma"snahmen zur Risikosmilderung genannt. \\
Folgende Informationen sind zu beachten:\\
\begin{itemize}
 \item Die Zahlen in den Kategorie-Spalten ``Auftreten'', ``Bedeutung'' und ``Entdeckung'' repr\"{a}sentieren eine Eingliederung des Risikos entsprechend der genannten Kategorie auf einer Skala von 1 bis 10, wobei eine niedrigere Zahl positiv, eine h\"{o}here Zahl negativ zu verstehen ist.
 \item Die RPZ (die \textit{Risiko-Priorit\"{a}tszahl}) ist das Produkt aus den drei vorhergehenden Kategorien Auftreten, Bedeutung und Entdeckung.\newline
 Eine RPZ unter 100 ist als unkritisch einzustufen, eine RPZ \"{u}ber 700 als sehr kritisch. Hier ist es sehr wahrscheinlich, dass bei Eintritt dieses Fehlers oder Risikos das Projekt scheitern wird.
\end{itemize}

